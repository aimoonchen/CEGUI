\begin{DoxyAuthor}{Author}
Paul D Turner
\end{DoxyAuthor}
Copyright (c) 2006 -\/ 2011 Paul D Turner \& The C\+E\+G\+UI Development Team

Permission is granted to copy, distribute and/or modify this document under the terms of the G\+NU Free Documentation License, Version 1.\+2 or any later version published by the Free Software Foundation; with no Invariant Sections, no Front-\/\+Cover Texts, and no Back-\/\+Cover Texts. A copy of the license is included in the section entitled \mbox{\hyperlink{gnu_fdl}{G\+NU Free Documentation License}}.\hypertarget{fal_man_fal_s1}{}\section{Contents}\label{fal_man_fal_s1}
\hypertarget{fal_man_fal_s1_1}{}\subsection{Tutorial Style Introduction}\label{fal_man_fal_s1_1}

\begin{DoxyItemize}
\item \mbox{\hyperlink{fal_intro}{Introduction and overview}}
\item \mbox{\hyperlink{fal_tut1}{Introduction to Falagard \textquotesingle{}looknfeel\textquotesingle{} X\+ML}} 
\end{DoxyItemize}\hypertarget{fal_man_fal_s1_2}{}\subsection{Reference Material}\label{fal_man_fal_s1_2}

\begin{DoxyItemize}
\item \mbox{\hyperlink{fal_element_ref}{Falagard X\+ML Element Reference}}
\item \mbox{\hyperlink{fal_enum_ref}{Falagard X\+ML Enumeration Reference}}
\item \mbox{\hyperlink{fal_baseclass_ref}{C\+E\+G\+UI Widget Base Type Requirements}}
\item \mbox{\hyperlink{fal_wr_ref}{Falagard Window Renderer Requirements}} 
\end{DoxyItemize}\hypertarget{gnu_fdl}{}\section{G\+NU Free Documentation License}\label{gnu_fdl}
\begin{DoxyVerb}    GNU Free Documentation License
      Version 1.2, November 2002
\end{DoxyVerb}


Copyright (C) 2000,2001,2002 Free Software Foundation, Inc. 51 Franklin St, Fifth Floor, Boston, MA 02110-\/1301 U\+SA Everyone is permitted to copy and distribute verbatim copies of this license document, but changing it is not allowed.

0. P\+R\+E\+A\+M\+B\+LE

The purpose of this License is to make a manual, textbook, or other functional and useful document \char`\"{}free\char`\"{} in the sense of freedom\+: to assure everyone the effective freedom to copy and redistribute it, with or without modifying it, either commercially or noncommercially. Secondarily, this License preserves for the author and publisher a way to get credit for their work, while not being considered responsible for modifications made by others.

This License is a kind of \char`\"{}copyleft\char`\"{}, which means that derivative works of the document must themselves be free in the same sense. It complements the G\+NU General Public License, which is a copyleft license designed for free software.

We have designed this License in order to use it for manuals for free software, because free software needs free documentation\+: a free program should come with manuals providing the same freedoms that the software does. But this License is not limited to software manuals; it can be used for any textual work, regardless of subject matter or whether it is published as a printed book. We recommend this License principally for works whose purpose is instruction or reference.


\begin{DoxyEnumerate}
\item A\+P\+P\+L\+I\+C\+A\+B\+I\+L\+I\+TY A\+ND D\+E\+F\+I\+N\+I\+T\+I\+O\+NS
\end{DoxyEnumerate}

This License applies to any manual or other work, in any medium, that contains a notice placed by the copyright holder saying it can be distributed under the terms of this License. Such a notice grants a world-\/wide, royalty-\/free license, unlimited in duration, to use that work under the conditions stated herein. The \char`\"{}\+Document\char`\"{}, below, refers to any such manual or work. Any member of the public is a licensee, and is addressed as \char`\"{}you\char`\"{}. You accept the license if you copy, modify or distribute the work in a way requiring permission under copyright law.

A \char`\"{}\+Modified Version\char`\"{} of the Document means any work containing the Document or a portion of it, either copied verbatim, or with modifications and/or translated into another language.

A \char`\"{}\+Secondary Section\char`\"{} is a named appendix or a front-\/matter section of the Document that deals exclusively with the relationship of the publishers or authors of the Document to the Document\textquotesingle{}s overall subject (or to related matters) and contains nothing that could fall directly within that overall subject. (Thus, if the Document is in part a textbook of mathematics, a Secondary Section may not explain any mathematics.) The relationship could be a matter of historical connection with the subject or with related matters, or of legal, commercial, philosophical, ethical or political position regarding them.

The \char`\"{}\+Invariant Sections\char`\"{} are certain Secondary Sections whose titles are designated, as being those of Invariant Sections, in the notice that says that the Document is released under this License. If a section does not fit the above definition of Secondary then it is not allowed to be designated as Invariant. The Document may contain zero Invariant Sections. If the Document does not identify any Invariant Sections then there are none.

The \char`\"{}\+Cover Texts\char`\"{} are certain short passages of text that are listed, as Front-\/\+Cover Texts or Back-\/\+Cover Texts, in the notice that says that the Document is released under this License. A Front-\/\+Cover Text may be at most 5 words, and a Back-\/\+Cover Text may be at most 25 words.

A \char`\"{}\+Transparent\char`\"{} copy of the Document means a machine-\/readable copy, represented in a format whose specification is available to the general public, that is suitable for revising the document straightforwardly with generic text editors or (for images composed of pixels) generic paint programs or (for drawings) some widely available drawing editor, and that is suitable for input to text formatters or for automatic translation to a variety of formats suitable for input to text formatters. A copy made in an otherwise Transparent file format whose markup, or absence of markup, has been arranged to thwart or discourage subsequent modification by readers is not Transparent. An image format is not Transparent if used for any substantial amount of text. A copy that is not \char`\"{}\+Transparent\char`\"{} is called \char`\"{}\+Opaque\char`\"{}.

Examples of suitable formats for Transparent copies include plain A\+S\+C\+II without markup, Texinfo input format, La\+TeX input format, S\+G\+ML or X\+ML using a publicly available D\+TD, and standard-\/conforming simple H\+T\+ML, Post\+Script or P\+DF designed for human modification. Examples of transparent image formats include P\+NG, X\+CF and J\+PG. Opaque formats include proprietary formats that can be read and edited only by proprietary word processors, S\+G\+ML or X\+ML for which the D\+TD and/or processing tools are not generally available, and the machine-\/generated H\+T\+ML, Post\+Script or P\+DF produced by some word processors for output purposes only.

The \char`\"{}\+Title Page\char`\"{} means, for a printed book, the title page itself, plus such following pages as are needed to hold, legibly, the material this License requires to appear in the title page. For works in formats which do not have any title page as such, \char`\"{}\+Title Page\char`\"{} means the text near the most prominent appearance of the work\textquotesingle{}s title, preceding the beginning of the body of the text.

A section \char`\"{}\+Entitled X\+Y\+Z\char`\"{} means a named subunit of the Document whose title either is precisely X\+YZ or contains X\+YZ in parentheses following text that translates X\+YZ in another language. (Here X\+YZ stands for a specific section name mentioned below, such as \char`\"{}\+Acknowledgements\char`\"{}, \char`\"{}\+Dedications\char`\"{}, \char`\"{}\+Endorsements\char`\"{}, or \char`\"{}\+History\char`\"{}.) To \char`\"{}\+Preserve the Title\char`\"{} of such a section when you modify the Document means that it remains a section \char`\"{}\+Entitled X\+Y\+Z\char`\"{} according to this definition.

The Document may include Warranty Disclaimers next to the notice which states that this License applies to the Document. These Warranty Disclaimers are considered to be included by reference in this License, but only as regards disclaiming warranties\+: any other implication that these Warranty Disclaimers may have is void and has no effect on the meaning of this License.


\begin{DoxyEnumerate}
\item V\+E\+R\+B\+A\+T\+IM C\+O\+P\+Y\+I\+NG
\end{DoxyEnumerate}

You may copy and distribute the Document in any medium, either commercially or noncommercially, provided that this License, the copyright notices, and the license notice saying this License applies to the Document are reproduced in all copies, and that you add no other conditions whatsoever to those of this License. You may not use technical measures to obstruct or control the reading or further copying of the copies you make or distribute. However, you may accept compensation in exchange for copies. If you distribute a large enough number of copies you must also follow the conditions in section 3.

You may also lend copies, under the same conditions stated above, and you may publicly display copies.


\begin{DoxyEnumerate}
\item C\+O\+P\+Y\+I\+NG IN Q\+U\+A\+N\+T\+I\+TY
\end{DoxyEnumerate}

If you publish printed copies (or copies in media that commonly have printed covers) of the Document, numbering more than 100, and the Document\textquotesingle{}s license notice requires Cover Texts, you must enclose the copies in covers that carry, clearly and legibly, all these Cover Texts\+: Front-\/\+Cover Texts on the front cover, and Back-\/\+Cover Texts on the back cover. Both covers must also clearly and legibly identify you as the publisher of these copies. The front cover must present the full title with all words of the title equally prominent and visible. You may add other material on the covers in addition. Copying with changes limited to the covers, as long as they preserve the title of the Document and satisfy these conditions, can be treated as verbatim copying in other respects.

If the required texts for either cover are too voluminous to fit legibly, you should put the first ones listed (as many as fit reasonably) on the actual cover, and continue the rest onto adjacent pages.

If you publish or distribute Opaque copies of the Document numbering more than 100, you must either include a machine-\/readable Transparent copy along with each Opaque copy, or state in or with each Opaque copy a computer-\/network location from which the general network-\/using public has access to download using public-\/standard network protocols a complete Transparent copy of the Document, free of added material. If you use the latter option, you must take reasonably prudent steps, when you begin distribution of Opaque copies in quantity, to ensure that this Transparent copy will remain thus accessible at the stated location until at least one year after the last time you distribute an Opaque copy (directly or through your agents or retailers) of that edition to the public.

It is requested, but not required, that you contact the authors of the Document well before redistributing any large number of copies, to give them a chance to provide you with an updated version of the Document.


\begin{DoxyEnumerate}
\item M\+O\+D\+I\+F\+I\+C\+A\+T\+I\+O\+NS
\end{DoxyEnumerate}

You may copy and distribute a Modified Version of the Document under the conditions of sections 2 and 3 above, provided that you release the Modified Version under precisely this License, with the Modified Version filling the role of the Document, thus licensing distribution and modification of the Modified Version to whoever possesses a copy of it. In addition, you must do these things in the Modified Version\+:

A. Use in the Title Page (and on the covers, if any) a title distinct from that of the Document, and from those of previous versions (which should, if there were any, be listed in the History section of the Document). You may use the same title as a previous version if the original publisher of that version gives permission. B. List on the Title Page, as authors, one or more persons or entities responsible for authorship of the modifications in the Modified Version, together with at least five of the principal authors of the Document (all of its principal authors, if it has fewer than five), unless they release you from this requirement. C. State on the Title page the name of the publisher of the Modified Version, as the publisher. D. Preserve all the copyright notices of the Document. E. Add an appropriate copyright notice for your modifications adjacent to the other copyright notices. F. Include, immediately after the copyright notices, a license notice giving the public permission to use the Modified Version under the terms of this License, in the form shown in the Addendum below. G. Preserve in that license notice the full lists of Invariant Sections and required Cover Texts given in the Document\textquotesingle{}s license notice. H. Include an unaltered copy of this License. I. Preserve the section Entitled \char`\"{}\+History\char`\"{}, Preserve its Title, and add to it an item stating at least the title, year, new authors, and publisher of the Modified Version as given on the Title Page. If there is no section Entitled \char`\"{}\+History\char`\"{} in the Document, create one stating the title, year, authors, and publisher of the Document as given on its Title Page, then add an item describing the Modified Version as stated in the previous sentence. J. Preserve the network location, if any, given in the Document for public access to a Transparent copy of the Document, and likewise the network locations given in the Document for previous versions it was based on. These may be placed in the \char`\"{}\+History\char`\"{} section. You may omit a network location for a work that was published at least four years before the Document itself, or if the original publisher of the version it refers to gives permission. K. For any section Entitled \char`\"{}\+Acknowledgements\char`\"{} or \char`\"{}\+Dedications\char`\"{}, Preserve the Title of the section, and preserve in the section all the substance and tone of each of the contributor acknowledgements and/or dedications given therein. L. Preserve all the Invariant Sections of the Document, unaltered in their text and in their titles. Section numbers or the equivalent are not considered part of the section titles. M. Delete any section Entitled \char`\"{}\+Endorsements\char`\"{}. Such a section may not be included in the Modified Version. N. Do not retitle any existing section to be Entitled \char`\"{}\+Endorsements\char`\"{} or to conflict in title with any Invariant Section. O. Preserve any Warranty Disclaimers.

If the Modified Version includes new front-\/matter sections or appendices that qualify as Secondary Sections and contain no material copied from the Document, you may at your option designate some or all of these sections as invariant. To do this, add their titles to the list of Invariant Sections in the Modified Version\textquotesingle{}s license notice. These titles must be distinct from any other section titles.

You may add a section Entitled \char`\"{}\+Endorsements\char`\"{}, provided it contains nothing but endorsements of your Modified Version by various parties--for example, statements of peer review or that the text has been approved by an organization as the authoritative definition of a standard.

You may add a passage of up to five words as a Front-\/\+Cover Text, and a passage of up to 25 words as a Back-\/\+Cover Text, to the end of the list of Cover Texts in the Modified Version. Only one passage of Front-\/\+Cover Text and one of Back-\/\+Cover Text may be added by (or through arrangements made by) any one entity. If the Document already includes a cover text for the same cover, previously added by you or by arrangement made by the same entity you are acting on behalf of, you may not add another; but you may replace the old one, on explicit permission from the previous publisher that added the old one.

The author(s) and publisher(s) of the Document do not by this License give permission to use their names for publicity for or to assert or imply endorsement of any Modified Version.


\begin{DoxyEnumerate}
\item C\+O\+M\+B\+I\+N\+I\+NG D\+O\+C\+U\+M\+E\+N\+TS
\end{DoxyEnumerate}

You may combine the Document with other documents released under this License, under the terms defined in section 4 above for modified versions, provided that you include in the combination all of the Invariant Sections of all of the original documents, unmodified, and list them all as Invariant Sections of your combined work in its license notice, and that you preserve all their Warranty Disclaimers.

The combined work need only contain one copy of this License, and multiple identical Invariant Sections may be replaced with a single copy. If there are multiple Invariant Sections with the same name but different contents, make the title of each such section unique by adding at the end of it, in parentheses, the name of the original author or publisher of that section if known, or else a unique number. Make the same adjustment to the section titles in the list of Invariant Sections in the license notice of the combined work.

In the combination, you must combine any sections Entitled \char`\"{}\+History\char`\"{} in the various original documents, forming one section Entitled \char`\"{}\+History\char`\"{}; likewise combine any sections Entitled \char`\"{}\+Acknowledgements\char`\"{}, and any sections Entitled \char`\"{}\+Dedications\char`\"{}. You must delete all sections Entitled \char`\"{}\+Endorsements\char`\"{}.


\begin{DoxyEnumerate}
\item C\+O\+L\+L\+E\+C\+T\+I\+O\+NS OF D\+O\+C\+U\+M\+E\+N\+TS
\end{DoxyEnumerate}

You may make a collection consisting of the Document and other documents released under this License, and replace the individual copies of this License in the various documents with a single copy that is included in the collection, provided that you follow the rules of this License for verbatim copying of each of the documents in all other respects.

You may extract a single document from such a collection, and distribute it individually under this License, provided you insert a copy of this License into the extracted document, and follow this License in all other respects regarding verbatim copying of that document.


\begin{DoxyEnumerate}
\item A\+G\+G\+R\+E\+G\+A\+T\+I\+ON W\+I\+TH I\+N\+D\+E\+P\+E\+N\+D\+E\+NT W\+O\+R\+KS
\end{DoxyEnumerate}

A compilation of the Document or its derivatives with other separate and independent documents or works, in or on a volume of a storage or distribution medium, is called an \char`\"{}aggregate\char`\"{} if the copyright resulting from the compilation is not used to limit the legal rights of the compilation\textquotesingle{}s users beyond what the individual works permit. When the Document is included in an aggregate, this License does not apply to the other works in the aggregate which are not themselves derivative works of the Document.

If the Cover Text requirement of section 3 is applicable to these copies of the Document, then if the Document is less than one half of the entire aggregate, the Document\textquotesingle{}s Cover Texts may be placed on covers that bracket the Document within the aggregate, or the electronic equivalent of covers if the Document is in electronic form. Otherwise they must appear on printed covers that bracket the whole aggregate.


\begin{DoxyEnumerate}
\item T\+R\+A\+N\+S\+L\+A\+T\+I\+ON
\end{DoxyEnumerate}

Translation is considered a kind of modification, so you may distribute translations of the Document under the terms of section 4. Replacing Invariant Sections with translations requires special permission from their copyright holders, but you may include translations of some or all Invariant Sections in addition to the original versions of these Invariant Sections. You may include a translation of this License, and all the license notices in the Document, and any Warranty Disclaimers, provided that you also include the original English version of this License and the original versions of those notices and disclaimers. In case of a disagreement between the translation and the original version of this License or a notice or disclaimer, the original version will prevail.

If a section in the Document is Entitled \char`\"{}\+Acknowledgements\char`\"{}, \char`\"{}\+Dedications\char`\"{}, or \char`\"{}\+History\char`\"{}, the requirement (section 4) to Preserve its Title (section 1) will typically require changing the actual title.


\begin{DoxyEnumerate}
\item T\+E\+R\+M\+I\+N\+A\+T\+I\+ON
\end{DoxyEnumerate}

You may not copy, modify, sublicense, or distribute the Document except as expressly provided for under this License. Any other attempt to copy, modify, sublicense or distribute the Document is void, and will automatically terminate your rights under this License. However, parties who have received copies, or rights, from you under this License will not have their licenses terminated so long as such parties remain in full compliance.


\begin{DoxyEnumerate}
\item F\+U\+T\+U\+RE R\+E\+V\+I\+S\+I\+O\+NS OF T\+H\+IS L\+I\+C\+E\+N\+SE
\end{DoxyEnumerate}

The Free Software Foundation may publish new, revised versions of the G\+NU Free Documentation License from time to time. Such new versions will be similar in spirit to the present version, but may differ in detail to address new problems or concerns. See \href{http://www.gnu.org/copyleft/}{\texttt{ http\+://www.\+gnu.\+org/copyleft/}}.

Each version of the License is given a distinguishing version number. If the Document specifies that a particular numbered version of this License \char`\"{}or any later version\char`\"{} applies to it, you have the option of following the terms and conditions either of that specified version or of any later version that has been published (not as a draft) by the Free Software Foundation. If the Document does not specify a version number of this License, you may choose any version ever published (not as a draft) by the Free Software Foundation.

A\+D\+D\+E\+N\+D\+UM\+: How to use this License for your documents

To use this License in a document you have written, include a copy of the License in the document and put the following copyright and license notices just after the title page\+: \begin{DoxyVerb}Copyright (c)  YEAR  YOUR NAME.
Permission is granted to copy, distribute and/or modify this document
under the terms of the GNU Free Documentation License, Version 1.2
or any later version published by the Free Software Foundation;
with no Invariant Sections, no Front-Cover Texts, and no Back-Cover Texts.
A copy of the license is included in the section entitled "GNU
Free Documentation License".
\end{DoxyVerb}


If you have Invariant Sections, Front-\/\+Cover Texts and Back-\/\+Cover Texts, replace the \char`\"{}with...\+Texts.\char`\"{} line with this\+: \begin{DoxyVerb}with the Invariant Sections being LIST THEIR TITLES, with the
Front-Cover Texts being LIST, and with the Back-Cover Texts being LIST.
\end{DoxyVerb}


If you have Invariant Sections without Cover Texts, or some other combination of the three, merge those two alternatives to suit the situation.

If your document contains nontrivial examples of program code, we recommend releasing these examples in parallel under your choice of free software license, such as the G\+NU General Public License, to permit their use in free software. \hypertarget{fal_intro}{}\section{Introduction and overview}\label{fal_intro}
\hypertarget{fal_intro_fal_whatis}{}\subsection{What is the Falagard Skinning System?}\label{fal_intro_fal_whatis}
The Falagard skinning system for C\+E\+G\+UI consists partly of a set of enhancements to the C\+E\+G\+UI base library, and partly of a window renderer module called \char`\"{}\+C\+E\+G\+U\+I\+Core\+Window\+Renderer\+Set\char`\"{}. Combined, these elements are intended to make it easier to create custom skins or \textquotesingle{}looks\textquotesingle{} for C\+E\+G\+UI window and widget elements.

The Falagard system is designed to allow widget imagery specification, sub-\/widget layout, and default property initialisers to be specified via X\+ML files rather than in C++ or scripted code (which, before now, was the only way to do these things).

The system is named \char`\"{}\+Falagard\char`\"{} after the forum name of the person who initially suggested the feature (as is the trend in all things C\+E\+G\+UI), although it was designed and implemented by the core C\+E\+G\+UI team.

The Falagard extensions are not limited to the \textquotesingle{}looknfeel\textquotesingle{} X\+ML files only; there are supporting elements within the core library, as well as extensions to the G\+UI scheme system to allow you to create what are essentially new widget types. This is achieved by mapping a named widget \textquotesingle{}look\textquotesingle{} and a named window renderer to a named widget base type. I\textquotesingle{}ve probably just about lost you now, the best thing is to not worry about all these details for the time being!

Once your new type has been defined in a scheme and loaded, you can specify the name of that new type name when creating windows or widgets via the Window\+Manager singleton as you would for any other widget type. There are no additional issues to be considered when using a \textquotesingle{}skinned\textquotesingle{} widget than when using one of the old \textquotesingle{}programmed\textquotesingle{} widget types.\hypertarget{fal_intro_fal_intro_contents}{}\subsection{Section Contents}\label{fal_intro_fal_intro_contents}

\begin{DoxyItemize}
\item \mbox{\hyperlink{fal_intro_fal_unifiedsystem}{The Unified Co-\/ordinate System}}
\item \mbox{\hyperlink{fal_intro_fal_udim}{The U\+Dim type}}
\item \mbox{\hyperlink{fal_intro_fal_uvector2}{The U\+Vector2 type}}
\item \mbox{\hyperlink{fal_intro_fal_urect}{The U\+Rect type}}
\item \mbox{\hyperlink{fal_intro_fal_window_alignments}{Window Alignments}}
\item \mbox{\hyperlink{fal_intro_fal_schemes}{Falagard in Schemes}}
\item \mbox{\hyperlink{fal_intro_fal_conclusion}{Conclusion}}
\end{DoxyItemize}\hypertarget{fal_intro_fal_unifiedsystem}{}\subsection{The Unified Co-\/ordinate System}\label{fal_intro_fal_unifiedsystem}
As part of the Falagard system, C\+E\+G\+UI has effectively replaced the old either/or approach to relative and absolute co-\/ordinates with a new \textquotesingle{}unified\textquotesingle{} co-\/ordinate system. Using this new system, each co-\/ordinate can specify both a parent-\/relative and absolute-\/pixel component. Since most people baulk at the idea of this, I\textquotesingle{}ll use examples to introduce these concepts.\hypertarget{fal_intro_fal_udim}{}\subsubsection{U\+Dim}\label{fal_intro_fal_udim}
\hypertarget{fal_intro_fal_udim_def}{}\paragraph{U\+Dim Definition}\label{fal_intro_fal_udim_def}
The basic building block of the unified system is the U\+Dim. This type represents a single dimension of some kind, and is defined as\+: 
\begin{DoxyCode}{0}
\DoxyCodeLine{UDim(scale, offset)}
\end{DoxyCode}


where\+: \begin{DoxyItemize}
\item {\ttfamily \textquotesingle{}scale\textquotesingle{}} represents some proportion of the parent element, where the parent element is either some other Window or the total available display, and is usually a value between 0 and 1.\+0. The scale value corresponds to relative coordinates under the pre-\/unified system. \item {\ttfamily \textquotesingle{}offset\textquotesingle{}} represents an arbitrary number of pixels. For positional values, offset represents a pixel offset, for size values, offset represents a number of additional pixels (basically like a padding value). The offset value corresponds to absolute coordinates under the pre-\/unified system.\end{DoxyItemize}
Still confused? On to the examples!\hypertarget{fal_intro_fal_udim_examples}{}\paragraph{Simple U\+Dim Examples}\label{fal_intro_fal_udim_examples}
\hypertarget{fal_intro_fal_udim_example1}{}\subparagraph{Example 1}\label{fal_intro_fal_udim_example1}

\begin{DoxyCode}{0}
\DoxyCodeLine{UDim(0, 10)}
\end{DoxyCode}
 Here we see a U\+Dim with a scale of 0, and an offset of 10. This simply represents an absolute value of 10, if you used such a U\+Dim to set a window width, then under the old system it\textquotesingle{}s the equivalent of\+: 
\begin{DoxyCode}{0}
\DoxyCodeLine{myWindow->setWidth(Absolute, 10);}
\end{DoxyCode}
\hypertarget{fal_intro_fal_udim_example2}{}\subparagraph{Example 2}\label{fal_intro_fal_udim_example2}

\begin{DoxyCode}{0}
\DoxyCodeLine{UDim(0.25f, 0)}
\end{DoxyCode}
 Here we see a U\+Dim with a scale of 0.\+25 and an offset of 0. This represents a simple relative co-\/ordinate. If you were to set the y position of a window using this U\+Dim, then the window would be a quarter of the way down it\textquotesingle{}s parent, and it\textquotesingle{}s the same as the following under the old system\+: 
\begin{DoxyCode}{0}
\DoxyCodeLine{myWindow->setYPosition(Relative, 0.25f);}
\end{DoxyCode}
\hypertarget{fal_intro_fal_udim_example3}{}\subparagraph{Example 3}\label{fal_intro_fal_udim_example3}

\begin{DoxyCode}{0}
\DoxyCodeLine{UDim(0.33f, -15)}
\end{DoxyCode}
 Here we see the power of U\+Dim. We have a scale of 0.\+33 and an offset of -\/15. If we used this as the height of a window, you would get a height that is approximately one third of the height of the window\textquotesingle{}s parent, minus 15 pixels. There is no simple equivalent for this under the old system.\hypertarget{fal_intro_fal_udim_prop_fmt}{}\paragraph{U\+Dim Property Format}\label{fal_intro_fal_udim_prop_fmt}
The format of a U\+Dim to be used in the window property strings is as follows\+: 
\begin{DoxyCode}{0}
\DoxyCodeLine{\{s,o\}}
\end{DoxyCode}


where\+: \begin{DoxyItemize}
\item {\ttfamily \textquotesingle{}s\textquotesingle{}} is the scale value. \item {\ttfamily \textquotesingle{}o\textquotesingle{}} is the pixel offset.\end{DoxyItemize}
\hypertarget{fal_intro_fal_uvector2}{}\subsubsection{U\+Vector2}\label{fal_intro_fal_uvector2}
\hypertarget{fal_intro_fal_uvector2_def}{}\paragraph{U\+Vector2 Definition}\label{fal_intro_fal_uvector2_def}
There is a U\+Vector2 type which consists of two U\+Dim elements; one for the x axis, and one for the y axis. Note that the U\+Vector2 is used to specify both positional points and also sizes; that is, there is no such thing as U\+Size to correspond to the C\+E\+G\+U\+I\+::\+Size type that used to be used for specifying a size.

The U\+Vector2 is defined as\+: 
\begin{DoxyCode}{0}
\DoxyCodeLine{UVector2(x\_udim, y\_udim)}
\end{DoxyCode}
 where\+: \begin{DoxyItemize}
\item {\ttfamily \textquotesingle{}x\+\_\+udim\textquotesingle{}} is a U\+Dim value that specifies the x co-\/ordinate or width. \item {\ttfamily \textquotesingle{}y\+\_\+udim\textquotesingle{}} is a U\+Dim value that specifies the y co-\/ordinate or height.\end{DoxyItemize}
\hypertarget{fal_intro_fal_uvector2_examples}{}\paragraph{Simple U\+Vector2 Examples}\label{fal_intro_fal_uvector2_examples}
\hypertarget{fal_intro_fal_uvec2_example1}{}\subparagraph{Example 1}\label{fal_intro_fal_uvec2_example1}

\begin{DoxyCode}{0}
\DoxyCodeLine{UVector2( UDim(0, 25), UDim(0.2f, 12) )}
\end{DoxyCode}
 The above example specifies a point that is 25 pixels along the x-\/axis and one fifth of the way down the parent window plus twelve pixels.\hypertarget{fal_intro_fal_uvec2_example2}{}\subparagraph{Example 2}\label{fal_intro_fal_uvec2_example2}

\begin{DoxyCode}{0}
\DoxyCodeLine{UVector2( UDim(1.0f, -25), UDim(1.0f, -25) )}
\end{DoxyCode}
 This example, intended as a size for a window, would give the window the same width as its parent, minus 25 pixels, and the same height as its parent, minus 25 pixels.\hypertarget{fal_intro_fal_uvector2_prop_fmt}{}\paragraph{U\+Vector2 Property Format}\label{fal_intro_fal_uvector2_prop_fmt}
The format of a U\+Vector2 to be used in the window property strings is as follows\+: 
\begin{DoxyCode}{0}
\DoxyCodeLine{\{\{sx,ox\},\{sy,oy\}\}}
\end{DoxyCode}


where\+: \begin{DoxyItemize}
\item {\ttfamily \textquotesingle{}sx\textquotesingle{}} is the scale value for the x-\/axis \item {\ttfamily \textquotesingle{}ox\textquotesingle{}} is the pixel offset for the x-\/axis. \item {\ttfamily \textquotesingle{}sy\textquotesingle{}} is the scale value for the y-\/axis \item {\ttfamily \textquotesingle{}oy\textquotesingle{}} is the pixel offset for the y-\/axis.\end{DoxyItemize}
\hypertarget{fal_intro_fal_urect}{}\subsubsection{U\+Rect}\label{fal_intro_fal_urect}
\hypertarget{fal_intro_fal_urect_def}{}\paragraph{U\+Rect Definition}\label{fal_intro_fal_urect_def}
The last of the Unified co-\/ordinate types is U\+Rect. The U\+Rect defines four sides of a rectangle using U\+Dim elements. You generally access the U\+Rect as you would the normal \textquotesingle{}Rect\textquotesingle{} type, except that each edge of the rectangle is represented by a U\+Dim rather than a float, or any other type you may be used to seeing! 
\begin{DoxyCode}{0}
\DoxyCodeLine{URect(left\_udim, top\_udim, right\_udim, bottom\_udim)}
\end{DoxyCode}


where\+: \begin{DoxyItemize}
\item {\ttfamily \textquotesingle{}left\+\_\+udim\textquotesingle{}} is a U\+Dim defining the left edge. \item {\ttfamily \textquotesingle{}top\+\_\+udim\textquotesingle{}} is a U\+Dim defining the top edge. \item {\ttfamily \textquotesingle{}right\+\_\+udim\textquotesingle{}} is a U\+Dim defining the right edge. \item {\ttfamily \textquotesingle{}bottom\+\_\+udim\textquotesingle{}} is a U\+Dim defining the bottom edge.\end{DoxyItemize}
It is also possible to define a U\+Rect with two U\+Vector2 objects; the first describes the top-\/left corner, and the second the bottom-\/right corner\+: 
\begin{DoxyCode}{0}
\DoxyCodeLine{URect(tl\_uvec2, br\_uvec2)}
\end{DoxyCode}


where\+: \begin{DoxyItemize}
\item {\ttfamily \textquotesingle{}tl\+\_\+uvec2\textquotesingle{}} is a U\+Vector2 that describes the top-\/left point of the rect area. \item {\ttfamily \textquotesingle{}br\+\_\+uvec2\textquotesingle{}} is a U\+Vector2 that describes the bottom-\/right point of the rect area. Don\textquotesingle{}t confuse this with the size of the area.\end{DoxyItemize}
\hypertarget{fal_intro_fal_urect_examples}{}\paragraph{Simple U\+Rect Example}\label{fal_intro_fal_urect_examples}

\begin{DoxyCode}{0}
\DoxyCodeLine{URect( UDim(0, 25),}
\DoxyCodeLine{       UDim(0, 25),}
\DoxyCodeLine{       UDim(1.0f, -25),}
\DoxyCodeLine{       UDim(1.0f, -25)}
\DoxyCodeLine{     )}
\end{DoxyCode}


If we used the U\+Rect defined here to specify the area for a window, we would get a window that was 25 pixels smaller than its parent on each edge.\hypertarget{fal_intro_fal_urect_prop_fmt}{}\paragraph{U\+Rect Property format}\label{fal_intro_fal_urect_prop_fmt}
The format of a U\+Rect to be used in the window property strings is as follows\+: 
\begin{DoxyCode}{0}
\DoxyCodeLine{\{\{sl,ol\},\{st,ot\},\{sr,or\},\{sb,ob\}\}}
\end{DoxyCode}


where\+: \begin{DoxyItemize}
\item {\ttfamily \textquotesingle{}sl\textquotesingle{}} is the scale value for the left edge. \item {\ttfamily \textquotesingle{}ol\textquotesingle{}} is the pixel offset for the left edge. \item {\ttfamily \textquotesingle{}st\textquotesingle{}} is the scale value for the top edge. \item {\ttfamily \textquotesingle{}ot\textquotesingle{}} is the pixel offset for the top edge. \item {\ttfamily \textquotesingle{}sr\textquotesingle{}} is the scale value for the right edge. \item {\ttfamily \textquotesingle{}or\textquotesingle{}} is the pixel offset for the right edge. \item {\ttfamily \textquotesingle{}sb\textquotesingle{}} is the scale value for the bottom edge. \item {\ttfamily \textquotesingle{}ob\textquotesingle{}} is the pixel offset for the bottom edge.\end{DoxyItemize}
\hypertarget{fal_intro_fal_window_alignments}{}\subsection{Window Alignments}\label{fal_intro_fal_window_alignments}
The Falagard enhancements also include settings to specify alignments for windows. This gives the possibility to position child windows from the right edge, bottom edge and centre positions of their parents, as well as the previous left edge and top edge possibilities.

It is possible to set the alignment options in C++ code by using methods in the Window class, and also via the properties system which is used by X\+ML layouts system.\hypertarget{fal_intro_fal_vert_alignments}{}\subsubsection{Vertical Alignments}\label{fal_intro_fal_vert_alignments}
To set the vertical alignment use the Window class member function\+: 
\begin{DoxyCode}{0}
\DoxyCodeLine{\textcolor{keywordtype}{void} setVerticalAlignment(\textcolor{keyword}{const} VerticalAlignment alignment);}
\end{DoxyCode}


This function takes one of the Vertical\+Alignment enumerated values as its input. The Vertical\+Alignment enumeration is defined as\+: 
\begin{DoxyCode}{0}
\DoxyCodeLine{\textcolor{keyword}{enum} VerticalAlignment}
\DoxyCodeLine{\{}
\DoxyCodeLine{    VA\_TOP,}
\DoxyCodeLine{    VA\_CENTRE,}
\DoxyCodeLine{    VA\_BOTTOM}
\DoxyCodeLine{\};}
\end{DoxyCode}


Where\+: \begin{DoxyItemize}
\item {\ttfamily V\+A\+\_\+\+T\+OP} specifies that y-\/axis positions specify an offset for a window\textquotesingle{}s top edge from the top edge of it\textquotesingle{}s parent window. \item {\ttfamily V\+A\+\_\+\+C\+E\+N\+T\+RE} specifies that y-\/axis positions specify an offset for a window\textquotesingle{}s centre point from the centre point of it\textquotesingle{}s parent window. \item {\ttfamily V\+A\+\_\+\+B\+O\+T\+T\+OM} specifies that y-\/axis positions specify an offset for a window\textquotesingle{}s bottom edge from the bottom edge of it\textquotesingle{}s parent window.\end{DoxyItemize}
The window property to access the vertical alignment setting is\+: 
\begin{DoxyCode}{0}
\DoxyCodeLine{\textcolor{stringliteral}{"VerticalAlignment"}}
\end{DoxyCode}
 This property takes a simple string as its value, which should be one of the following options\+: 
\begin{DoxyCode}{0}
\DoxyCodeLine{\textcolor{stringliteral}{"Top"}}
\DoxyCodeLine{\textcolor{stringliteral}{"Centre"}}
\DoxyCodeLine{\textcolor{stringliteral}{"Bottom"}}
\end{DoxyCode}


Where these setting values correspond to the similar values in the Vertical\+Alignment enumeration.\hypertarget{fal_intro_fal_horz_alignments}{}\subsubsection{Horizontal Alignments}\label{fal_intro_fal_horz_alignments}
To set the horizontal alignment use the Window class member function\+: 
\begin{DoxyCode}{0}
\DoxyCodeLine{\textcolor{keywordtype}{void} setHorizontalAlignment(\textcolor{keyword}{const} HorizontalAlignment alignment);}
\end{DoxyCode}


This function takes one of the Horizontal\+Alignment enumerated values as its input. The Horizontal\+Alignment enumeration is defined as\+: 
\begin{DoxyCode}{0}
\DoxyCodeLine{\textcolor{keyword}{enum} HorizontalAlignment}
\DoxyCodeLine{\{}
\DoxyCodeLine{    HA\_LEFT,}
\DoxyCodeLine{    HA\_CENTRE,}
\DoxyCodeLine{    HA\_RIGHT}
\DoxyCodeLine{\};}
\end{DoxyCode}


Where\+: \begin{DoxyItemize}
\item {\ttfamily H\+A\+\_\+\+L\+E\+FT} specifies that x-\/axis positions specify an offset for a window\textquotesingle{}s left edge from the left edge of it\textquotesingle{}s parent window. \item {\ttfamily H\+A\+\_\+\+C\+E\+N\+T\+RE} specifies that x-\/axis positions specify an offset for a window\textquotesingle{}s centre point from the centre point of it\textquotesingle{}s parent window. \item {\ttfamily H\+A\+\_\+\+R\+I\+G\+HT} specifies that x-\/axis positions specify an offset for a window\textquotesingle{}s right edge from the right edge of it\textquotesingle{}s parent window.\end{DoxyItemize}
The window property to access the horizontal alignment setting is\+: 
\begin{DoxyCode}{0}
\DoxyCodeLine{\textcolor{stringliteral}{"HorizontalAlignment"}}
\end{DoxyCode}


This property takes a simple string as its value, which should be one of the following options\+: 
\begin{DoxyCode}{0}
\DoxyCodeLine{\textcolor{stringliteral}{"Left"}}
\DoxyCodeLine{\textcolor{stringliteral}{"Centre"}}
\DoxyCodeLine{\textcolor{stringliteral}{"Right"}}
\end{DoxyCode}


Where these setting values correspond to the similar values in the Horizontal\+Alignment enumeration.\hypertarget{fal_intro_fal_schemes}{}\subsection{Falagard in Schemes}\label{fal_intro_fal_schemes}
The C\+E\+G\+UI scheme system is the means by which you to specify how the system is to load your X\+ML skin definition files, known as \textquotesingle{}looknfeel\textquotesingle{} files, and how these skins are to be mapped to window renderers and widget base classes to create new concrete widget types.\hypertarget{fal_intro_fal_wr_modules}{}\subsubsection{The C\+E\+G\+U\+I\+Core\+Window\+Renderer\+Set module}\label{fal_intro_fal_wr_modules}
One of the main parts of the Falagard system is the window renderer module known as C\+E\+G\+U\+I\+Core\+Window\+Renderer\+Set (which will be named lib\+C\+E\+G\+U\+I\+Core\+Window\+Renderer\+Set.\+so on linux style systems and C\+E\+G\+U\+I\+Core\+Window\+Renderer\+Set.\+dll on Win32 systems). This module contains a set of predefined window renderer classes that take actions to transform skinning data loaded from skin definition X\+ML files into the rendering operations and layout adjustments required to output the widget visual representation to the display.

Before you can make use of the C\+E\+G\+U\+I\+Core\+Window\+Renderer\+Set module it must be loaded into the system. To achieve this, you will usually specify it in one of your scheme X\+ML files so that it\textquotesingle{}s available to the system. This can be done with a single line of X\+ML in a scheme file, such as\+: 
\begin{DoxyCode}{0}
\DoxyCodeLine{<WindowRendererSet Filename=\textcolor{stringliteral}{"CEGUICoreWindowRendererSet"} />}
\end{DoxyCode}


Some users, having previously employed the Window\+Set \textquotesingle{}look\textquotesingle{} modules, may be used to specifying a list of widgets which are to be made available from the module, this is not required when loading a Window\+Renderer module (actually, such lists of widgets are no longer needed for the old style \textquotesingle{}look\textquotesingle{} modules either, as long as the module has been updated to provide the required entry point); by employing X\+ML such as that shown above, the module will register all widget types it has available.

The key thing about the C\+E\+G\+U\+I\+Core\+Window\+Renderer\+Set module is that for each widget base type, it defines various required elements and states. These required items need to be defined within the widget look definitions of your looknfeel X\+ML files; they enable the system to make use of your skin imagery and related data in a logical fashion. All of the required elements for each widget can be found in the reference sections \mbox{\hyperlink{fal_baseclass_ref}{C\+E\+G\+UI Widget Base Type Requirements}} and \mbox{\hyperlink{fal_wr_ref}{Falagard Window Renderer Requirements}}.\hypertarget{fal_intro_fal_looknfeel_elements}{}\subsubsection{Look\+N\+Feel Elements}\label{fal_intro_fal_looknfeel_elements}
The {\ttfamily $<$Look\+N\+Feel$>$} X\+ML element for schemes is the means by which you will usually get C\+E\+G\+UI to load the X\+ML \textquotesingle{}looknfeel\textquotesingle{} files containing your widget skin definitions. It is possible to load these files manually via code, but it is expected that the majority of users will be using the scheme system. The Look\+N\+Feel element should appear after any Font or Imageset elements, but before any Window\+Set elements.

The following is an example of how to use the Look\+N\+Feel element\+: 
\begin{DoxyCode}{0}
\DoxyCodeLine{<LookNFeel Filename=\textcolor{stringliteral}{"FunkyWidgets.looknfeel"} />}
\end{DoxyCode}


Here we can see a single \textquotesingle{}Filename\textquotesingle{} attribute which specifies the name the file to be loaded.

It is acceptable to specify as many Look\+N\+Feel elements as is required. This allows you to configure your X\+ML files in the way that best suits your application. This might mean that all skin definitions for all widget elements will go into a single file, it might mean that you have multiple files with a single widget skin definition in each, or it could be any place in between the two extremes -\/ it\textquotesingle{}s up to you.\hypertarget{fal_intro_fal_mappings}{}\subsubsection{Falagard\+Mapping Elements}\label{fal_intro_fal_mappings}
The C\+E\+G\+UI scheme system supports a {\ttfamily $<$Falagard\+Mapping$>$} element that creates a new concrete window or widget type within the system. This is achieved by creating a named alias that ties together a base widget type, a window renderer type, and a named widget \textquotesingle{}Look\+N\+Feel\textquotesingle{}. Here, \textquotesingle{}Look\+N\+Feel\textquotesingle{} refers to an individual widget skin as opposed to an entire \textquotesingle{}looknfeel\textquotesingle{} X\+ML file. The base widget type will generally be one of the core system widgets provided by the C\+E\+G\+UI library, although any window type that has a concrete Window\+Factory registered in the system is a candidate, which allows the system to be extended with custom widgets. The window renderer type will usually be the name of one of the window renderers registered when the C\+E\+G\+U\+I\+Core\+Window\+Renderer\+Set module was loaded, again this is not a requirement -\/ the window renderer used could just as easily be one you have written yourself. The named \textquotesingle{}Look\+N\+Feel\textquotesingle{} is what you specify in your X\+ML looknfeel files (via Widget\+Look elements).

An example mapping\+: 
\begin{DoxyCode}{0}
\DoxyCodeLine{<FalagardMapping}
\DoxyCodeLine{    WindowType=\textcolor{stringliteral}{"FunkyLook/Button"}}
\DoxyCodeLine{    TargetType=\textcolor{stringliteral}{"CEGUI/PushButton"}}
\DoxyCodeLine{    Renderer=\textcolor{stringliteral}{"Falagard/Button"}}
\DoxyCodeLine{    LookNFeel=\textcolor{stringliteral}{"MyButtonSkin"}}
\DoxyCodeLine{/>}
\end{DoxyCode}


In this example, a new widget type named \char`\"{}\+Funky\+Look/\+Button\char`\"{} is being created. The new widget is based upon the \char`\"{}\+C\+E\+G\+U\+I/\+Push\+Button\char`\"{} base type, uses the window renderer named \char`\"{}\+Falagard/\+Button\char`\"{} and applies the skin defined by the loaded Widget\+Look named \char`\"{}\+My\+Button\+Skin\char`\"{}. Once the scheme with this mapping has been loaded, you can then use the new type within the system\+: 
\begin{DoxyCode}{0}
\DoxyCodeLine{\textcolor{comment}{// Get access to the main window manager}}
\DoxyCodeLine{CEGUI::WindowManager wMgr\& = CEGUI::WindowManager::getSingleton();}
\DoxyCodeLine{}
\DoxyCodeLine{\textcolor{comment}{// Create a new widget}}
\DoxyCodeLine{Window* wnd = wMgr.createWindow(\textcolor{stringliteral}{"FunkyLook/Button"}, \textcolor{stringliteral}{"myFunkyButton"});}
\end{DoxyCode}


Here we create an instance of the new widget, and name it \char`\"{}my\+Funky\+Button\char`\"{}. The widget can now be attached to other windows and generally used as you would any \textquotesingle{}normal\textquotesingle{} widget.\hypertarget{fal_intro_fal_conclusion}{}\subsection{Conclusion}\label{fal_intro_fal_conclusion}
This concludes the overview of the new parts of the C\+E\+G\+UI system.

You have seen how the new unified coordinate system works, and how to make use of the new window alignment options.

You have also learned the basics of how to set up your scheme files to initialise the Falagard window renderer module, and how to map skins defined in X\+ML files to the Falagard to create new widget types.

The next section of this document will introduce the X\+ML format and elements used in the \textquotesingle{}looknfeel\textquotesingle{} files. \hypertarget{fal_tut1}{}\section{Introduction to Falagard \textquotesingle{}looknfeel\textquotesingle{} X\+ML}\label{fal_tut1}
Before we get to the good stuff, I\textquotesingle{}d just like to point out that this section (or, indeed, the entire document) is not intended to teach you anything about X\+ML in general. It is assumed the reader has some familiarity with X\+ML and how to use it properly.\hypertarget{fal_tut1_fal_tut1_empty_skin}{}\subsection{Before we begin\+: An empty skin}\label{fal_tut1_fal_tut1_empty_skin}
Before we can start adding widget skins, or Widget\+Looks as they are known in the system, to our X\+ML file, we need the basic file outline initialised. This is extremely trivial, and looks like this\+: 
\begin{DoxyCode}{0}
\DoxyCodeLine{<?xml version=\textcolor{stringliteral}{"1.0"} ?>}
\DoxyCodeLine{<Falagard version=\textcolor{stringliteral}{"7"}>}
\DoxyCodeLine{</Falagard>}
\end{DoxyCode}


We will be placing our Widget\+Look definitions between the {\ttfamily $<$Falagard$>$$<$/\+Falagard$>$} pair. It is possible to specify as many sub-\/elements as we require within these tags, so all of our definitions can go into a single file (in most cases this ends up being a very large file!)\hypertarget{fal_tut1_fal_tut1_button}{}\subsection{Starting Simple\+: A Button}\label{fal_tut1_fal_tut1_button}
Without a doubt, the humble push button is the most common widget we\textquotesingle{}re ever likely to see; without this workhorse, any UI would be virtually useless. So, this is where we will start.

To define any widget skin, you use the Widget\+Look element and specify a name for the widget type that you\textquotesingle{}re defining by using the name attribute. So we\textquotesingle{}ll start off by adding the following empty Widget\+Look to our initial skin file\+: 
\begin{DoxyCode}{0}
\DoxyCodeLine{<WidgetLook name=\textcolor{stringliteral}{"TaharezLook/Button"}>}
\DoxyCodeLine{</WidgetLook>}
\end{DoxyCode}


As you can see from the reference for the Falagard/\+Button window renderer, we are required to specify imagery for numerous states, namely these are\+: \begin{DoxyItemize}
\item {\ttfamily Normal} \item {\ttfamily Hover} \item {\ttfamily Pushed} \item {\ttfamily Disabled} \end{DoxyItemize}
Since we now know what states are required for the widget, it\textquotesingle{}s a good idea to add the framework for these first; this effectively makes the Widget\+Look complete and usable, although obviously nothing would be drawn for it at this stage since we have not defined any imagery. So, we add empty State\+Imagery elements for the required states, and we end up with this\+: 
\begin{DoxyCode}{0}
\DoxyCodeLine{<WidgetLook name=\textcolor{stringliteral}{"TaharezLook/Button"}>}
\DoxyCodeLine{}
\DoxyCodeLine{  <StateImagery name=\textcolor{stringliteral}{"Normal"}>}
\DoxyCodeLine{  </StateImagery>}
\DoxyCodeLine{}
\DoxyCodeLine{  <StateImagery name=\textcolor{stringliteral}{"Hover"}>}
\DoxyCodeLine{  </StateImagery>}
\DoxyCodeLine{}
\DoxyCodeLine{  <StateImagery name=\textcolor{stringliteral}{"Pushed"}>}
\DoxyCodeLine{  </StateImagery>}
\DoxyCodeLine{}
\DoxyCodeLine{  <StateImagery name=\textcolor{stringliteral}{"PushedOff"}>}
\DoxyCodeLine{  </StateImagery>}
\DoxyCodeLine{}
\DoxyCodeLine{  <StateImagery name=\textcolor{stringliteral}{"Disabled"}>}
\DoxyCodeLine{  </StateImagery>}
\DoxyCodeLine{}
\DoxyCodeLine{</WidgetLook>}
\end{DoxyCode}


To specify rendering to be used for a widget, we use the Imagery\+Section element. Each imagery section is given a name; this name is used later to \textquotesingle{}include\textquotesingle{} the imagery section within layers defined for each of the state imagery definitions.

For our button, we will have an imagery section for each of the button states. We can add the outline of these to our existing, work-\/in-\/progress, widget-\/look\+: 
\begin{DoxyCode}{0}
\DoxyCodeLine{<WidgetLook name=\textcolor{stringliteral}{"TaharezLook/Button"}>}
\DoxyCodeLine{}
\DoxyCodeLine{  <ImagerySection name=\textcolor{stringliteral}{"normal\_imagery"}>}
\DoxyCodeLine{  </ImagerySection>}
\DoxyCodeLine{}
\DoxyCodeLine{  <ImagerySection name=\textcolor{stringliteral}{"hover\_imagery"}>}
\DoxyCodeLine{  </ImagerySection>}
\DoxyCodeLine{}
\DoxyCodeLine{  <ImagerySection name=\textcolor{stringliteral}{"pushed\_imagery"}>}
\DoxyCodeLine{  </ImagerySection>}
\DoxyCodeLine{}
\DoxyCodeLine{  <StateImagery name=\textcolor{stringliteral}{"Normal"}>}
\DoxyCodeLine{  </StateImagery>}
\DoxyCodeLine{}
\DoxyCodeLine{  <StateImagery name=\textcolor{stringliteral}{"Hover"}>}
\DoxyCodeLine{  </StateImagery>}
\DoxyCodeLine{}
\DoxyCodeLine{  <StateImagery name=\textcolor{stringliteral}{"Pushed"}>}
\DoxyCodeLine{  </StateImagery>}
\DoxyCodeLine{}
\DoxyCodeLine{  <StateImagery name=\textcolor{stringliteral}{"PushedOff"}>}
\DoxyCodeLine{  </StateImagery>}
\DoxyCodeLine{}
\DoxyCodeLine{  <StateImagery name=\textcolor{stringliteral}{"Disabled"}>}
\DoxyCodeLine{  </StateImagery>}
\DoxyCodeLine{}
\DoxyCodeLine{</WidgetLook>}
\end{DoxyCode}


Now we can start to define the Imagey\+Components for each section; this will tell the system how we want our button to appear on screen.

The imagery for Taharez\+Look gives us three sections for each button state (except Disabled; for this we\textquotesingle{}ll just re-\/use the \textquotesingle{}normal\+\_\+imagery\textquotesingle{} and use some different colours!). The available imagery sections give us a left end, a right end, and a middle section.

There are various ways that we can approach applying these image sections to the widget; although the intended use is to have the end pieces drawn at their \textquotesingle{}natural\textquotesingle{} horizontal size and the middle section stretched to fill the space in between the two ends. This all sounds simple enough, although there is one issue; the actual pixel sizes of the imagery is not fixed. The Taharez\+Look imageset uses the auto-\/scaling feature, which means that the source images will have variable sizes dependant upon the display resolution. All this needs to be taken into account when specifying the imagery; this way we ensure the results will be what we expect -\/ at all resolutions.

To specify an image to be drawn, we use the Imagery\+Component element. This should be added as a sub-\/element of Imagery\+Section. So we\textquotesingle{}ll start by adding an empty imagery component to the definition for \textquotesingle{}normal\+\_\+imagery\textquotesingle{}\+: 
\begin{DoxyCode}{0}
\DoxyCodeLine{...}
\DoxyCodeLine{<ImagerySection name=\textcolor{stringliteral}{"normal\_imagery"}>}
\DoxyCodeLine{}
\DoxyCodeLine{  <ImageryComponent>}
\DoxyCodeLine{  </ImageryComponent>}
\DoxyCodeLine{}
\DoxyCodeLine{</ImagerySection>}
\DoxyCodeLine{...}
\end{DoxyCode}


The first thing we need to add to the Imagery\+Component is an area definition telling the system where this image should be drawn\+: 
\begin{DoxyCode}{0}
\DoxyCodeLine{<ImageryComponent>}
\DoxyCodeLine{}
\DoxyCodeLine{  <Area>}
\DoxyCodeLine{  </Area>}
\DoxyCodeLine{}
\DoxyCodeLine{</ImageryComponent>}
\end{DoxyCode}


We\textquotesingle{}ll start by placing the image for the left end of the button. This is the simplest component to place, since its position is known as being (0, 0). To specify these absolute values, we use the Absolute\+Dim element.

We start defining the required dimensions for our image area by using the Dim element, and using Absolute\+Dim sub-\/element to indicate values to be used\+: 
\begin{DoxyCode}{0}
\DoxyCodeLine{<ImageryComponent>}
\DoxyCodeLine{  <Area>}
\DoxyCodeLine{}
\DoxyCodeLine{    <Dim type=\textcolor{stringliteral}{"LeftEdge"}>}
\DoxyCodeLine{      <AbsoluteDim value=\textcolor{stringliteral}{"0"} />}
\DoxyCodeLine{    </Dim>}
\DoxyCodeLine{}
\DoxyCodeLine{    <Dim type=\textcolor{stringliteral}{"TopEdge"}>}
\DoxyCodeLine{      <AbsoluteDim value=\textcolor{stringliteral}{"0"} />}
\DoxyCodeLine{    </Dim>}
\DoxyCodeLine{}
\DoxyCodeLine{  </Area>}
\DoxyCodeLine{</ImageryComponent>}
\end{DoxyCode}


We have defined the left and top edges which gives our image its position. Next we will specify dimensions to establish the area size.

We want the width of the area to come from the source image itself, to do this we use the Image\+Dim element and tell it to access the image that we will be using for this component\+: 
\begin{DoxyCode}{0}
\DoxyCodeLine{<Dim type=\textcolor{stringliteral}{"Width"}>}
\DoxyCodeLine{}
\DoxyCodeLine{  <ImageDim}
\DoxyCodeLine{    imageset=\textcolor{stringliteral}{"TaharezLook"}}
\DoxyCodeLine{    image=\textcolor{stringliteral}{"ButtonLeftNormal"}}
\DoxyCodeLine{    dimension=\textcolor{stringliteral}{"Width"}}
\DoxyCodeLine{  />}
\DoxyCodeLine{}
\DoxyCodeLine{</Dim>}
\end{DoxyCode}


This tells the system that for the width of the area being defined, use the width of the image named Button\+Left\+Normal from the Taharez\+Look imageset.

The last part of our area is the height. This is another simple thing to specify, since we want the height to be the same as the full height of the widget being defined. We could use either the Unified\+Dim element or the Widget\+Dim element for this purpose; we\textquotesingle{}ll use the Unified\+Dim here as it does not need to look up the widget by name and so is likely more economical\+:


\begin{DoxyCode}{0}
\DoxyCodeLine{<Dim type=\textcolor{stringliteral}{"Height"}>}
\DoxyCodeLine{}
\DoxyCodeLine{  <UnifiedDim scale=\textcolor{stringliteral}{"1.0"} type=\textcolor{stringliteral}{"Height"} />}
\DoxyCodeLine{}
\DoxyCodeLine{</Dim>}
\end{DoxyCode}


Here we use a scale value of 1.\+0 to indicate we want the full height of the widget.

Now we have completed our area definition for this first image, and it looks like this\+: 
\begin{DoxyCode}{0}
\DoxyCodeLine{<ImageryComponent>}
\DoxyCodeLine{  <Area>}
\DoxyCodeLine{    <Dim type=\textcolor{stringliteral}{"LeftEdge"}>}
\DoxyCodeLine{      <AbsoluteDim value=\textcolor{stringliteral}{"0"} />}
\DoxyCodeLine{    </Dim>}
\DoxyCodeLine{    <Dim type=\textcolor{stringliteral}{"TopEdge"}>}
\DoxyCodeLine{      <AbsoluteDim value=\textcolor{stringliteral}{"0"} />}
\DoxyCodeLine{    </Dim>}
\DoxyCodeLine{    <Dim type=\textcolor{stringliteral}{"Width"}>}
\DoxyCodeLine{      <ImageDim}
\DoxyCodeLine{        imageset=\textcolor{stringliteral}{"TaharezLook"}}
\DoxyCodeLine{        image=\textcolor{stringliteral}{"ButtonLeftNormal"}}
\DoxyCodeLine{        dimension=\textcolor{stringliteral}{"Width"}}
\DoxyCodeLine{      />}
\DoxyCodeLine{    </Dim>}
\DoxyCodeLine{    <Dim type=\textcolor{stringliteral}{"Height"}>}
\DoxyCodeLine{      <UnifiedDim scale=\textcolor{stringliteral}{"1.0"} type=\textcolor{stringliteral}{"Height"} />}
\DoxyCodeLine{    </Dim>}
\DoxyCodeLine{  </Area>}
\DoxyCodeLine{</ImageryComponent>}
\end{DoxyCode}


The next thing we need to do here is tell the system which image it should draw, this is done by using the Image element, and this should be placed immediately after the area definition\+: 
\begin{DoxyCode}{0}
\DoxyCodeLine{...}
\DoxyCodeLine{<Image imageset=\textcolor{stringliteral}{"TaharezLook"} image=\textcolor{stringliteral}{"ButtonLeftNormal"} />}
\DoxyCodeLine{...}
\end{DoxyCode}


The final element that we need to add to this Imagery\+Component definition is the Vert\+Format element. Using this we will tell the system to stretch the image vertically so that it covers the full height of our defined area\+: 
\begin{DoxyCode}{0}
\DoxyCodeLine{...}
\DoxyCodeLine{<VertFormat type=\textcolor{stringliteral}{"Stretched"} />}
\DoxyCodeLine{...}
\end{DoxyCode}


This completes the definition for the left end of the button, and the final xml for this component looks like this\+: 
\begin{DoxyCode}{0}
\DoxyCodeLine{<ImageryComponent>}
\DoxyCodeLine{  <Area>}
\DoxyCodeLine{    <Dim type=\textcolor{stringliteral}{"LeftEdge"}>}
\DoxyCodeLine{      <AbsoluteDim value=\textcolor{stringliteral}{"0"} />}
\DoxyCodeLine{    </Dim>}
\DoxyCodeLine{    <Dim type=\textcolor{stringliteral}{"TopEdge"}>}
\DoxyCodeLine{      <AbsoluteDim value=\textcolor{stringliteral}{"0"} />}
\DoxyCodeLine{    </Dim>}
\DoxyCodeLine{    <Dim type=\textcolor{stringliteral}{"Width"}>}
\DoxyCodeLine{      <ImageDim}
\DoxyCodeLine{        imageset=\textcolor{stringliteral}{"TaharezLook"}}
\DoxyCodeLine{        image=\textcolor{stringliteral}{"ButtonLeftNormal"}}
\DoxyCodeLine{        dimension=\textcolor{stringliteral}{"Width"}}
\DoxyCodeLine{      />}
\DoxyCodeLine{    </Dim>}
\DoxyCodeLine{    <Dim type=\textcolor{stringliteral}{"Height"}>}
\DoxyCodeLine{      <UnifiedDim scale=\textcolor{stringliteral}{"1.0"} type=\textcolor{stringliteral}{"Height"} />}
\DoxyCodeLine{    </Dim>}
\DoxyCodeLine{  </Area>}
\DoxyCodeLine{  <Image imageset=\textcolor{stringliteral}{"TaharezLook"} image=\textcolor{stringliteral}{"ButtonLeftNormal"} />}
\DoxyCodeLine{  <VertFormat type=\textcolor{stringliteral}{"Stretched"} />}
\DoxyCodeLine{</ImageryComponent>}
\end{DoxyCode}


The next image we will set up is the right end. To show another approach for image placement, rather than precisely defining the area where the image will appear, here we will define the target area as covering the entire widget and use the image alignment formatting to draw the image on the right hand side of the widget.

The area definition that specifies the entire widget is something that you\textquotesingle{}ll likely use a lot, and looks like this\+: 
\begin{DoxyCode}{0}
\DoxyCodeLine{<Area>}
\DoxyCodeLine{  <Dim type=\textcolor{stringliteral}{"LeftEdge"}><AbsoluteDim value=\textcolor{stringliteral}{"0"} /></Dim>}
\DoxyCodeLine{  <Dim type=\textcolor{stringliteral}{"TopEdge"}><AbsoluteDim value=\textcolor{stringliteral}{"0"} /></Dim>}
\DoxyCodeLine{  <Dim type=\textcolor{stringliteral}{"Width"}><UnifiedDim scale=\textcolor{stringliteral}{"1"} type=\textcolor{stringliteral}{"Width"} /></Dim>}
\DoxyCodeLine{  <Dim type=\textcolor{stringliteral}{"Height"}><UnifiedDim scale=\textcolor{stringliteral}{"1"} type=\textcolor{stringliteral}{"Height"} /></Dim>}
\DoxyCodeLine{</Area>}
\end{DoxyCode}


Next comes comes the image specification\+: 
\begin{DoxyCode}{0}
\DoxyCodeLine{<Image imageset=\textcolor{stringliteral}{"TaharezLook"} image=\textcolor{stringliteral}{"ButtonRightNormal"} />}
\end{DoxyCode}


Then the vertical formatting option\+: 
\begin{DoxyCode}{0}
\DoxyCodeLine{<VertFormat type=\textcolor{stringliteral}{"Stretched"} />}
\end{DoxyCode}


Finally, we add the horizontal formatting option which tells the system to align this image on the right edge of the defined area\+: 
\begin{DoxyCode}{0}
\DoxyCodeLine{<HorzFormat type=\textcolor{stringliteral}{"RightAligned"} />}
\end{DoxyCode}


The completed definition for the right end image now looks like this\+: 
\begin{DoxyCode}{0}
\DoxyCodeLine{<ImageryComponent>}
\DoxyCodeLine{}
\DoxyCodeLine{  <Area>}
\DoxyCodeLine{    <Dim type=\textcolor{stringliteral}{"LeftEdge"}><AbsoluteDim value=\textcolor{stringliteral}{"0"} /></Dim>}
\DoxyCodeLine{    <Dim type=\textcolor{stringliteral}{"TopEdge"}><AbsoluteDim value=\textcolor{stringliteral}{"0"} /></Dim>}
\DoxyCodeLine{    <Dim type=\textcolor{stringliteral}{"Width"}><UnifiedDim scale=\textcolor{stringliteral}{"1"} type=\textcolor{stringliteral}{"Width"} /></Dim>}
\DoxyCodeLine{    <Dim type=\textcolor{stringliteral}{"Height"}><UnifiedDim scale=\textcolor{stringliteral}{"1"} type=\textcolor{stringliteral}{"Height"} /></Dim>}
\DoxyCodeLine{  </Area>}
\DoxyCodeLine{}
\DoxyCodeLine{  <Image imageset=\textcolor{stringliteral}{"TaharezLook"} image=\textcolor{stringliteral}{"ButtonRightNormal"} />}
\DoxyCodeLine{  <VertFormat type=\textcolor{stringliteral}{"Stretched"} />}
\DoxyCodeLine{  <HorzFormat type=\textcolor{stringliteral}{"RightAligned"} />}
\DoxyCodeLine{}
\DoxyCodeLine{</ImageryComponent>}
\end{DoxyCode}


The last image we need to place for the \char`\"{}normal\+\_\+imagery\char`\"{} section is the middle section. Remember that we want this image to occupy the space between to two end pieces. The main part of achieving this is to correctly define the destination area for the image.

The vertical aspects of the image definition for the middle section will be the same as for the two ends, and as such these will not be discussed any further.

The first thing we need is to tell the system where the left edge of the middle section should appear. We know that the left edge of the image for the middle section needs to join to the right edge of the image for the left section. To achieve this we can make use of the Image\+Dim element to get the width of the left end image, and use this as the co-\/ordinate for the left edge of the middle section area\+: 
\begin{DoxyCode}{0}
\DoxyCodeLine{<Area>}
\DoxyCodeLine{  <Dim type=\textcolor{stringliteral}{"LeftEdge"}>}
\DoxyCodeLine{    <ImageDim}
\DoxyCodeLine{      imageset=\textcolor{stringliteral}{"TaharezLook"}}
\DoxyCodeLine{      image=\textcolor{stringliteral}{"ButtonLeftNormal"}}
\DoxyCodeLine{      dimension=\textcolor{stringliteral}{"Width"}}
\DoxyCodeLine{    />}
\DoxyCodeLine{  </Dim>}
\DoxyCodeLine{  ...}
\DoxyCodeLine{</Area>}
\end{DoxyCode}


Now comes the fun part. Due to the fact we want the skin to operate correctly without knowing ahead of time how large the images are, we must use mathematical calculations in order to establish the required width of the middle section. If we knew for sure the image sizes, this could all be pre-\/calculated and we could simply use Absolute\+Dim to tell the system the width we require. Unfortunately we are not this lucky. We are lucky, however, in the fact that the system provides a means for us to specify, within the X\+ML, what calculations should be done to arrive at the final value for a dimension. The Operator\+Dim element is what provides this ability.

Before going further we should look at what we need to calculate. The width of the middle section is basically the width of the widget, minus the combined width of the two end sections\+:

$ middleWidth=widgetWidth-(leftEndWidth+rightEndWidth) $

However, due to the fact that the area can accept either a width or right edge co-\/ordinate, we can simplify this a little by specifying the right edge co-\/ordinate instead of the width. The right edge location for this middle image will be equal to the width of the widget, minus the width of the right end image. So the final calculation we need to do is this\+:

$ rightEdge=widgetWidth-rightEndWidth $

The result from both calculations is the same, so wherever possible we will use the simpler option. Since we need to perform some calculation, to specify this in X\+ML we start off with an Operator\+Dim element that specifies the mathematical operation that we need to perform\+: 
\begin{DoxyCode}{0}
\DoxyCodeLine{<Dim type=\textcolor{stringliteral}{"RightEdge"}>}
\DoxyCodeLine{}
\DoxyCodeLine{  <OperatorDim op=\textcolor{stringliteral}{"Subtract"}>}
\DoxyCodeLine{  </OperatorDim>}
\DoxyCodeLine{}
\DoxyCodeLine{</Dim>}
\end{DoxyCode}


Next we will specify the first operand, which will be the left side of the operation being defined. In this example this will be the width of the widget\+: 
\begin{DoxyCode}{0}
\DoxyCodeLine{<Dim type=\textcolor{stringliteral}{"RightEdge"}>}
\DoxyCodeLine{  <OperatorDim op=\textcolor{stringliteral}{"Subtract"}>}
\DoxyCodeLine{}
\DoxyCodeLine{  <UnifiedDim scale=\textcolor{stringliteral}{"1"} type=\textcolor{stringliteral}{"Width"} />}
\DoxyCodeLine{}
\DoxyCodeLine{  </OperatorDim>}
\DoxyCodeLine{</Dim>}
\end{DoxyCode}


To complete the dimension specification we just insert a second $\ast$\+Dim element to tell the system what is to be subtracted. In this case it\textquotesingle{}s the width of the image for the right end, so we will use the Image\+Dim element for this purpose. The final specification for this dimension looks as follows\+: 
\begin{DoxyCode}{0}
\DoxyCodeLine{<Dim type=\textcolor{stringliteral}{"RightEdge"}>}
\DoxyCodeLine{  <OperatorDim op=\textcolor{stringliteral}{"Subtract"}>}
\DoxyCodeLine{    <UnifiedDim scale=\textcolor{stringliteral}{"1"} type=\textcolor{stringliteral}{"Width"} />}
\DoxyCodeLine{}
\DoxyCodeLine{    <ImageDim}
\DoxyCodeLine{      imageset=\textcolor{stringliteral}{"TaharezLook"}}
\DoxyCodeLine{      image=\textcolor{stringliteral}{"ButtonRightNormal"}}
\DoxyCodeLine{      dimension=\textcolor{stringliteral}{"Width"}}
\DoxyCodeLine{    />}
\DoxyCodeLine{}
\DoxyCodeLine{  </OperatorDim>}
\DoxyCodeLine{</Dim>}
\end{DoxyCode}


It is possible to chain further mathematical operations within the dimension specification. It would have been possible to do our original width calculation using two Operator\+Dim elements chained together to form an expression tree. The system is flexible enough that any expression using the supported operators can be expressed in this manner.

Anyway, I digress slightly. Lets get back to our button imagery. We now have enough information to define the middle section of the button, which looks like this\+: 
\begin{DoxyCode}{0}
\DoxyCodeLine{<ImageryComponent>}
\DoxyCodeLine{  <Area>}
\DoxyCodeLine{    <Dim type=\textcolor{stringliteral}{"LeftEdge"}>}
\DoxyCodeLine{      <ImageDim}
\DoxyCodeLine{        imageset=\textcolor{stringliteral}{"TaharezLook"}}
\DoxyCodeLine{        image=\textcolor{stringliteral}{"ButtonLeftNormal"}}
\DoxyCodeLine{        dimension=\textcolor{stringliteral}{"Width"}}
\DoxyCodeLine{      />}
\DoxyCodeLine{    </Dim>}
\DoxyCodeLine{    <Dim type=\textcolor{stringliteral}{"TopEdge"}><AbsoluteDim value=\textcolor{stringliteral}{"0"} /></Dim>}
\DoxyCodeLine{    <Dim type=\textcolor{stringliteral}{"RightEdge"}>}
\DoxyCodeLine{      <OperatorDim op=\textcolor{stringliteral}{"Subtract"}>}
\DoxyCodeLine{        <UnifiedDim scale=\textcolor{stringliteral}{"1"} type=\textcolor{stringliteral}{"Width"} />}
\DoxyCodeLine{        <ImageDim}
\DoxyCodeLine{          imageset=\textcolor{stringliteral}{"TaharezLook"}}
\DoxyCodeLine{          image=\textcolor{stringliteral}{"ButtonRightNormal"}}
\DoxyCodeLine{          dimension=\textcolor{stringliteral}{"Width"}}
\DoxyCodeLine{        />}
\DoxyCodeLine{      </OperatorDim>}
\DoxyCodeLine{    </Dim>}
\DoxyCodeLine{    <Dim type=\textcolor{stringliteral}{"Height"}><UnifiedDim scale=\textcolor{stringliteral}{"1"} type=\textcolor{stringliteral}{"Height"} /></Dim>}
\DoxyCodeLine{  </Area>}
\DoxyCodeLine{  <Image imageset=\textcolor{stringliteral}{"TaharezLook"} image=\textcolor{stringliteral}{"ButtonMiddleNormal"} />}
\DoxyCodeLine{  <VertFormat type=\textcolor{stringliteral}{"Stretched"} />}
\DoxyCodeLine{  <HorzFormat type=\textcolor{stringliteral}{"Stretched"} />}
\DoxyCodeLine{</ImageryComponent>}
\end{DoxyCode}


This completes the imagery within the normal\+\_\+imagery section. You can now add in the other two sections in the same manner, just replacing the image names used for the hover and pushed imagery as appropriate -\/ everything else will be exactly the same as what you\textquotesingle{}ve done for the normal imagery.

Now we can add references to the imagery sections within the elements that define the various states. The imagery for state imagery elements must be specified in layers. It is possible to specify multiple imagery sections to use within each layer, though for most simple cases, you\textquotesingle{}ll only need one layer.

Here we\textquotesingle{}ve added the imagery specification for the Normal state\+: 
\begin{DoxyCode}{0}
\DoxyCodeLine{<StateImagery name=\textcolor{stringliteral}{"Normal"}>}
\DoxyCodeLine{}
\DoxyCodeLine{  <Layer>}
\DoxyCodeLine{    <Section section=\textcolor{stringliteral}{"normal\_imagery"} />}
\DoxyCodeLine{  </Layer>}
\DoxyCodeLine{}
\DoxyCodeLine{</StateImagery>}
\end{DoxyCode}


The Hover and Pushed states are defined in a similar fashion; just replace the name \char`\"{}normal\+\_\+imagery\char`\"{} with the name of the appropriate imagery section for the state.

The Disabled state is somewhat different though; we do not have any specific imagery for this state, so instead we will re-\/use the normal\+\_\+imagery but specify some colours that will be applied to make the button appear darker. This is done by embedding a Colours element within the Section element, as demonstrated here\+: 
\begin{DoxyCode}{0}
\DoxyCodeLine{<StateImagery name=\textcolor{stringliteral}{"Disabled"}>}
\DoxyCodeLine{  <Layer>}
\DoxyCodeLine{    <Section section=\textcolor{stringliteral}{"normal\_imagery"}>}
\DoxyCodeLine{}
\DoxyCodeLine{      <Colours}
\DoxyCodeLine{        topLeft=\textcolor{stringliteral}{"FF7F7F7F"}}
\DoxyCodeLine{        topRight=\textcolor{stringliteral}{"FF7F7F7F"}}
\DoxyCodeLine{        bottomLeft=\textcolor{stringliteral}{"FF7F7F7F"}}
\DoxyCodeLine{        bottomRight=\textcolor{stringliteral}{"FF7F7F7F"}}
\DoxyCodeLine{      />}
\DoxyCodeLine{}
\DoxyCodeLine{    </Section>}
\DoxyCodeLine{  </Layer>}
\DoxyCodeLine{</StateImagery>}
\end{DoxyCode}


Now we have a nice button with imagery defined for all the required states. There\textquotesingle{}s just one thing missing -\/ we need to put some label text on the button.

To specify text, you use the Text\+Component element, which goes in an Imagery\+Section the same as the Imagery\+Component elements do. We could have put a Text\+Component in each of the imagery sections we defined to display the label, however this is wasteful repetition. A better approach is to define a imagery section what contains the label by itself, then we can re-\/use that for all of the states.

So, start by defining the containing Imagery\+Section\+: 
\begin{DoxyCode}{0}
\DoxyCodeLine{...}
\DoxyCodeLine{<ImagerySection name=\textcolor{stringliteral}{"label"}>}
\DoxyCodeLine{}
\DoxyCodeLine{  <TextComponent>}
\DoxyCodeLine{  </TextComponent>}
\DoxyCodeLine{}
\DoxyCodeLine{</ImagerySection>}
\DoxyCodeLine{...}
\end{DoxyCode}


The definition of the Text\+Component is extremely similar to that of Imagery\+Component. We specify an area for the text and the formatting that we require. We can also optionally specify a Text element which is used to explicitly set the font and / or text string to be drawn. Without these explicit settings, these items will be obtained from the base widget itself.

We want our label to be centred within the entire area of the widget, so we need to use the area that defines the entire widget (this was shown above, so will not be repeated here.)

We also need to set the formatting options for the text. For the vertical formatting we will use\+: 
\begin{DoxyCode}{0}
\DoxyCodeLine{<VertFormat type=\textcolor{stringliteral}{"CentreAligned"} />}
\end{DoxyCode}


And for the horizontal formatting\+: 
\begin{DoxyCode}{0}
\DoxyCodeLine{<HorzFormat type=\textcolor{stringliteral}{"WordWrapCentreAligned"} />}
\end{DoxyCode}


The final definition for our label imagery section looks like this\+: 
\begin{DoxyCode}{0}
\DoxyCodeLine{<ImagerySection name=\textcolor{stringliteral}{"label"}>}
\DoxyCodeLine{  <TextComponent>}
\DoxyCodeLine{    <Area>}
\DoxyCodeLine{      <Dim type=\textcolor{stringliteral}{"LeftEdge"}><AbsoluteDim value=\textcolor{stringliteral}{"0"} /></Dim>}
\DoxyCodeLine{      <Dim type=\textcolor{stringliteral}{"TopEdge"}><AbsoluteDim value=\textcolor{stringliteral}{"0"} /></Dim>}
\DoxyCodeLine{      <Dim type=\textcolor{stringliteral}{"Width"}><UnifiedDim scale=\textcolor{stringliteral}{"1"} type=\textcolor{stringliteral}{"Width"} /></Dim>}
\DoxyCodeLine{      <Dim type=\textcolor{stringliteral}{"Height"}><UnifiedDim scale=\textcolor{stringliteral}{"1"} type=\textcolor{stringliteral}{"Height"} /></Dim>}
\DoxyCodeLine{    </Area>}
\DoxyCodeLine{    <VertFormat type=\textcolor{stringliteral}{"CentreAligned"} />}
\DoxyCodeLine{    <HorzFormat type=\textcolor{stringliteral}{"WordWrapCentreAligned"} />}
\DoxyCodeLine{  </TextComponent>}
\DoxyCodeLine{</ImagerySection>}
\end{DoxyCode}


Now all that is left is to add this to the layer specification for the state imagery. Normal state now looks like this (with Hover and Pushed being very similar)\+: 
\begin{DoxyCode}{0}
\DoxyCodeLine{<StateImagery name=\textcolor{stringliteral}{"Normal"}>}
\DoxyCodeLine{  <Layer>}
\DoxyCodeLine{    <Section section=\textcolor{stringliteral}{"normal\_imagery"} />}
\DoxyCodeLine{    <Section section=\textcolor{stringliteral}{"label"} />}
\DoxyCodeLine{  </Layer>}
\DoxyCodeLine{</StateImagery>}
\end{DoxyCode}


And for Disabled we again specify some additional colours\+: 
\begin{DoxyCode}{0}
\DoxyCodeLine{<StateImagery name=\textcolor{stringliteral}{"Disabled"}>}
\DoxyCodeLine{  <Layer>}
\DoxyCodeLine{    <Section section=\textcolor{stringliteral}{"normal\_imagery"}>}
\DoxyCodeLine{      <Colours}
\DoxyCodeLine{        topLeft=\textcolor{stringliteral}{"FF7F7F7F"}}
\DoxyCodeLine{        topRight=\textcolor{stringliteral}{"FF7F7F7F"}}
\DoxyCodeLine{        bottomLeft=\textcolor{stringliteral}{"FF7F7F7F"}}
\DoxyCodeLine{        bottomRight=\textcolor{stringliteral}{"FF7F7F7F"}}
\DoxyCodeLine{      />}
\DoxyCodeLine{    </Section>}
\DoxyCodeLine{    <Section section=\textcolor{stringliteral}{"label"}>}
\DoxyCodeLine{      <Colours}
\DoxyCodeLine{        topLeft=\textcolor{stringliteral}{"FF7F7F7F"}}
\DoxyCodeLine{        topRight=\textcolor{stringliteral}{"FF7F7F7F"}}
\DoxyCodeLine{        bottomLeft=\textcolor{stringliteral}{"FF7F7F7F"}}
\DoxyCodeLine{        bottomRight=\textcolor{stringliteral}{"FF7F7F7F"}}
\DoxyCodeLine{      />}
\DoxyCodeLine{    </Section>}
\DoxyCodeLine{  </Layer>}
\DoxyCodeLine{</StateImagery>}
\end{DoxyCode}


This concludes the introduction tutorial. For full examples of this, and all the other Widget\+Look specifications, see the example \textquotesingle{}looknfeel\textquotesingle{} files in the C\+E\+G\+UI distribution, in the directory\+: {\itshape cegui/datafiles/looknfeel/} \hypertarget{fal_element_ref}{}\section{Falagard X\+ML Element Reference}\label{fal_element_ref}
The following pages contain reference material for the X\+ML elements defined for the Falagard skin definition files.\hypertarget{fal_element_ref_fal_elem_ref_sec_0}{}\subsection{Section Contents}\label{fal_element_ref_fal_elem_ref_sec_0}
\mbox{\hyperlink{fal_element_ref_fal_elem_ref_sec_1}{Overview}} ~\newline
 \mbox{\hyperlink{fal_element_ref_fal_elem_ref_sec_2}{$<$\+Absolute\+Dim$>$ Element}} ~\newline
 \mbox{\hyperlink{fal_element_ref_fal_elem_ref_sec_3}{$<$\+Area$>$ Element}} ~\newline
 \mbox{\hyperlink{fal_element_ref_fal_elem_ref_sec_4}{$<$\+Area\+Property$>$ Element}} ~\newline
 \mbox{\hyperlink{fal_element_ref_fal_elem_ref_sec_5}{$<$\+Child$>$ Element}} ~\newline
 \mbox{\hyperlink{fal_element_ref_fal_elem_ref_sec_6}{$<$\+Colour\+Property$>$ Element}} ~\newline
 \mbox{\hyperlink{fal_element_ref_fal_elem_ref_sec_7}{$<$\+Colour\+Rect\+Property$>$ Element}} ~\newline
 \mbox{\hyperlink{fal_element_ref_fal_elem_ref_sec_8}{$<$\+Colours$>$ Element}} ~\newline
 \mbox{\hyperlink{fal_element_ref_fal_elem_ref_sec_9}{$<$\+Dim$>$ Element}} ~\newline
 \mbox{\hyperlink{fal_element_ref_fal_elem_ref_eventaction}{$<$\+Event\+Action$>$ Element}} ~\newline
 \mbox{\hyperlink{fal_element_ref_fal_elem_ref_eventlinkdefintion}{$<$\+Event\+Link\+Definition$>$ Element}} ~\newline
 \mbox{\hyperlink{fal_element_ref_fal_elem_ref_eventlinktarget}{$<$\+Event\+Link\+Target$>$ Element}} ~\newline
 \mbox{\hyperlink{fal_element_ref_fal_elem_ref_sec_11}{$<$\+Falagard$>$ Element}} ~\newline
 \mbox{\hyperlink{fal_element_ref_fal_elem_ref_sec_12}{$<$\+Font\+Dim$>$ Element}} ~\newline
 \mbox{\hyperlink{fal_element_ref_fal_elem_ref_sec_13}{$<$\+Font\+Property$>$ Element}} ~\newline
 \mbox{\hyperlink{fal_element_ref_fal_elem_ref_sec_14}{$<$\+Frame\+Component$>$ Element}} ~\newline
 \mbox{\hyperlink{fal_element_ref_fal_elem_ref_sec_15}{$<$\+Horz\+Alignment$>$ Element}} ~\newline
 \mbox{\hyperlink{fal_element_ref_fal_elem_ref_sec_16}{$<$\+Horz\+Format$>$ Element}} ~\newline
 \mbox{\hyperlink{fal_element_ref_fal_elem_ref_sec_17}{$<$\+Horz\+Format\+Property$>$ Element}} ~\newline
 \mbox{\hyperlink{fal_element_ref_fal_elem_ref_sec_18}{$<$\+Image$>$ Element}} ~\newline
 \mbox{\hyperlink{fal_element_ref_fal_elem_ref_sec_19}{$<$\+Image\+Dim$>$ Element}} ~\newline
 \mbox{\hyperlink{fal_element_ref_fal_elem_ref_sec_imagepropertydim}{$<$\+Image\+Property\+Dim$>$ Element}} ~\newline
 \mbox{\hyperlink{fal_element_ref_fal_elem_ref_sec_20}{$<$\+Imagery\+Component$>$ Element}} ~\newline
 \mbox{\hyperlink{fal_element_ref_fal_elem_ref_sec_21}{$<$\+Image\+Property$>$ Element}} ~\newline
 \mbox{\hyperlink{fal_element_ref_fal_elem_ref_sec_22}{$<$\+Imagery\+Section$>$ Element}} ~\newline
 \mbox{\hyperlink{fal_element_ref_fal_elem_ref_sec_23}{$<$\+Layer$>$ Element}} ~\newline
 \mbox{\hyperlink{fal_element_ref_fal_elem_ref_sec_24}{$<$\+Named\+Area$>$ Element}} ~\newline
 \mbox{\hyperlink{fal_element_ref_fal_elem_ref_sec_operatordim}{$<$\+Operator\+Dim$>$ Element}} ~\newline
 \mbox{\hyperlink{fal_element_ref_fal_elem_ref_sec_25}{$<$\+Property$>$ Element}} ~\newline
 \mbox{\hyperlink{fal_element_ref_fal_elem_ref_sec_26}{$<$\+Property\+Definition$>$ Element}} ~\newline
 \mbox{\hyperlink{fal_element_ref_fal_elem_ref_sec_27}{$<$\+Property\+Link\+Definition$>$ Element}} ~\newline
 \mbox{\hyperlink{fal_element_ref_fal_elem_propertylinktarget}{$<$\+Property\+Link\+Target$>$ Element}} ~\newline
 \mbox{\hyperlink{fal_element_ref_fal_elem_ref_sec_28}{$<$\+Property\+Dim$>$ Element}} ~\newline
 \mbox{\hyperlink{fal_element_ref_fal_elem_ref_sec_29}{$<$\+Section$>$ Element}} ~\newline
 \mbox{\hyperlink{fal_element_ref_fal_elem_ref_sec_30}{$<$\+State\+Imagery$>$ Element}} ~\newline
 \mbox{\hyperlink{fal_element_ref_fal_elem_ref_sec_31}{$<$\+Text$>$ Element}} ~\newline
 \mbox{\hyperlink{fal_element_ref_fal_elem_ref_sec_32}{$<$\+Text\+Component$>$ Element}} ~\newline
 \mbox{\hyperlink{fal_element_ref_fal_elem_ref_sec_33}{$<$\+Text\+Property$>$ Element}} ~\newline
 \mbox{\hyperlink{fal_element_ref_fal_elem_ref_sec_34}{$<$\+Unified\+Dim$>$ Element}} ~\newline
 \mbox{\hyperlink{fal_element_ref_fal_elem_ref_sec_35}{$<$\+Vert\+Alignment$>$ Element}} ~\newline
 \mbox{\hyperlink{fal_element_ref_fal_elem_ref_sec_36}{$<$\+Vert\+Format$>$ Element}} ~\newline
 \mbox{\hyperlink{fal_element_ref_fal_elem_ref_sec_37}{$<$\+Vert\+Format\+Property$>$ Element}} ~\newline
 \mbox{\hyperlink{fal_element_ref_fal_elem_ref_sec_38}{$<$\+Widget\+Dim$>$ Element}} ~\newline
 \mbox{\hyperlink{fal_element_ref_fal_elem_ref_sec_39}{$<$\+Widget\+Look$>$ Element}} ~\newline
\hypertarget{fal_element_ref_fal_elem_ref_sec_1}{}\subsection{Overview}\label{fal_element_ref_fal_elem_ref_sec_1}
The reference for each element is arranged into sections, as described below\+:\hypertarget{fal_element_ref_fal_elem_ref_sec_1_1}{}\subsubsection{Purpose\+:}\label{fal_element_ref_fal_elem_ref_sec_1_1}
This section describes what the elements general purpose is within the specifications.\hypertarget{fal_element_ref_fal_elem_ref_sec_1_2}{}\subsubsection{Attributes\+:}\label{fal_element_ref_fal_elem_ref_sec_1_2}
This section describes available attributes for the elements, and whether they are required or optional.\hypertarget{fal_element_ref_fal_elem_ref_sec_1_3}{}\subsubsection{Usage\+:}\label{fal_element_ref_fal_elem_ref_sec_1_3}
Describes where the element may appear, whether the element may have sub-\/elements, and other important usage information.\hypertarget{fal_element_ref_fal_elem_ref_sec_1_4}{}\subsubsection{Examples\+:}\label{fal_element_ref_fal_elem_ref_sec_1_4}
For many elements, this section will contain brief examples showing the element used in context.\hypertarget{fal_element_ref_fal_elem_ref_sec_2}{}\subsection{$<$\+Absolute\+Dim$>$ Element}\label{fal_element_ref_fal_elem_ref_sec_2}
\hypertarget{fal_element_ref_fal_elem_ref_sec_2_1}{}\subsubsection{Purpose\+:}\label{fal_element_ref_fal_elem_ref_sec_2_1}
The {\ttfamily $<$Absolute\+Dim$>$} element is used to define a component dimension for an area rectangle. {\ttfamily $<$Absolute\+Dim$>$} is used to specify absolute pixel values for a dimension.\hypertarget{fal_element_ref_fal_elem_ref_sec_2_2}{}\subsubsection{Attributes\+:}\label{fal_element_ref_fal_elem_ref_sec_2_2}
\begin{DoxyItemize}
\item {\ttfamily value} specifies the a number of pixels. Required attribute.\end{DoxyItemize}
\hypertarget{fal_element_ref_fal_elem_ref_sec_2_3}{}\subsubsection{Usage\+:}\label{fal_element_ref_fal_elem_ref_sec_2_3}
\begin{DoxyItemize}
\item The {\ttfamily $<$Absolute\+Dim$>$} element can appear as a sub-\/element in {\ttfamily $<$Dim$>$} to form a dimension specification for an area. \item The {\ttfamily $<$Absolute\+Dim$>$} element can appear as a sub-\/element of {\ttfamily $<$Operator\+Dim$>$} to specify an operand for a dimension calculation.\end{DoxyItemize}
\hypertarget{fal_element_ref_fal_elem_ref_sec_2_4}{}\subsubsection{Examples\+:}\label{fal_element_ref_fal_elem_ref_sec_2_4}
The following shows {\ttfamily $<$Absolute\+Dim$>$} used to define an area rectangle. In the example, all four component dimensions of the area rectangle are specified using {\ttfamily $<$Absolute\+Dim$>$}\+: 
\begin{DoxyCode}{0}
\DoxyCodeLine{<Area>}
\DoxyCodeLine{  <Dim type=\textcolor{stringliteral}{"LeftEdge"} >}
\DoxyCodeLine{    <AbsoluteDim value=\textcolor{stringliteral}{"10"} />}
\DoxyCodeLine{  </Dim>}
\DoxyCodeLine{  <Dim type=\textcolor{stringliteral}{"TopEdge"} >}
\DoxyCodeLine{    <AbsoluteDim value=\textcolor{stringliteral}{"50"} />}
\DoxyCodeLine{  </Dim>}
\DoxyCodeLine{  <Dim type=\textcolor{stringliteral}{"Width"} >}
\DoxyCodeLine{    <AbsoluteDim value=\textcolor{stringliteral}{"290"} />}
\DoxyCodeLine{  </Dim>}
\DoxyCodeLine{  <Dim type=\textcolor{stringliteral}{"Height"} >}
\DoxyCodeLine{    <AbsoluteDim value=\textcolor{stringliteral}{"250"} />}
\DoxyCodeLine{  </Dim>}
\DoxyCodeLine{</Area>}
\end{DoxyCode}


The following shows {\ttfamily $<$Absolute\+Dim$>$} in use as part of a dimension calculation sequence. In the example the left edge is being set to the width of the child widget \textquotesingle{}my\+Widget\textquotesingle{} minus two pixels\+: 
\begin{DoxyCode}{0}
\DoxyCodeLine{<Area>}
\DoxyCodeLine{  <Dim type=\textcolor{stringliteral}{"LeftEdge"} >}
\DoxyCodeLine{    <OperatorDim op=\textcolor{stringliteral}{"Subtract"} >}
\DoxyCodeLine{      <WidgetDim widget=\textcolor{stringliteral}{"myWidget"} dimension=\textcolor{stringliteral}{"Width"} />}
\DoxyCodeLine{      <AbsoluteDim value=\textcolor{stringliteral}{"2"} />}
\DoxyCodeLine{    </OperatorDim>}
\DoxyCodeLine{  </Dim>}
\DoxyCodeLine{  ...}
\DoxyCodeLine{</Area>}
\end{DoxyCode}


Finally, we see {\ttfamily $<$Absolute\+Dim$>$} as part of a dimension calculation sequence. In the example, we are adding the value of some window property to the starting absolute value of six\+: 
\begin{DoxyCode}{0}
\DoxyCodeLine{<Area>}
\DoxyCodeLine{  ...}
\DoxyCodeLine{  <Dim type=\textcolor{stringliteral}{"Height"} >}
\DoxyCodeLine{    <OperatorDim op=\textcolor{stringliteral}{"Add"} >}
\DoxyCodeLine{      <AbsoluteDim value=\textcolor{stringliteral}{"6"} />}
\DoxyCodeLine{      <PropertyDim name=\textcolor{stringliteral}{"someHeightProperty"} />}
\DoxyCodeLine{    </OperatorDim>}
\DoxyCodeLine{  </Dim>}
\DoxyCodeLine{</Area>}
\end{DoxyCode}
\hypertarget{fal_element_ref_fal_elem_ref_sec_3}{}\subsection{$<$\+Area$>$ Element}\label{fal_element_ref_fal_elem_ref_sec_3}
\hypertarget{fal_element_ref_fal_elem_ref_sec_3_1}{}\subsubsection{Purpose\+:}\label{fal_element_ref_fal_elem_ref_sec_3_1}
The {\ttfamily $<$Area$>$} element is a simple container element for the {\ttfamily $<$Dim$>$} dimension elements, or a single {\ttfamily $<$Area\+Property$>$} element, in order to form a rectangular area. {\ttfamily $<$Area$>$} is generally used to define target regions which are to be used for rendering imagery, text, to place a component child widget, or to form \textquotesingle{}named\textquotesingle{} areas required by the base widget.\hypertarget{fal_element_ref_fal_elem_ref_sec_3_2}{}\subsubsection{Attributes\+:}\label{fal_element_ref_fal_elem_ref_sec_3_2}
{\ttfamily $<$Area$>$} has no attributes.\hypertarget{fal_element_ref_fal_elem_ref_sec_3_3}{}\subsubsection{Usage\+:}\label{fal_element_ref_fal_elem_ref_sec_3_3}

\begin{DoxyItemize}
\item The {\ttfamily $<$Area$>$} element must contain either\+: 
\begin{DoxyItemize}
\item A single {\ttfamily $<$Area\+Property$>$} element that describes a U\+Rect type property where the final area can be obtained.


\item Four {\ttfamily $<$Dim$>$} elements\+: 
\begin{DoxyItemize}
\item One {\ttfamily $<$Dim$>$} element must define the left edge or x position. 
\item One {\ttfamily $<$Dim$>$} element must define the top edge or y position. 
\item One {\ttfamily $<$Dim$>$} element must define either the right edge or width. 
\item One {\ttfamily $<$Dim$>$} element must define either the bottom edge or height. 
\end{DoxyItemize}
\end{DoxyItemize}


\item The {\ttfamily $<$Area$>$} element may appear in any of the following elements\+: 
\begin{DoxyItemize}
\item {\ttfamily $<$Child$>$} to define the target area to be occupied by a child widget. 
\item {\ttfamily $<$Imagery\+Component$>$} to define the target rendering area of an image. 
\item {\ttfamily $<$Named\+Area$>$} to define an area which can be retrieved by name. 
\item {\ttfamily $<$Text\+Component$>$} to define the target rendering area of some text. 
\item {\ttfamily $<$Frame\+Component$>$} to define the target rendering area for a frame. 
\end{DoxyItemize}
\end{DoxyItemize}\hypertarget{fal_element_ref_fal_elem_ref_sec_3_4}{}\subsubsection{Examples\+:}\label{fal_element_ref_fal_elem_ref_sec_3_4}
In this example we can see a named area being defined\+: 
\begin{DoxyCode}{0}
\DoxyCodeLine{<NamedArea name=\textcolor{stringliteral}{"exampleArea"} >}
\DoxyCodeLine{  <Area>}
\DoxyCodeLine{    <Dim type=\textcolor{stringliteral}{"LeftEdge"}><AbsoluteDim value=\textcolor{stringliteral}{"0"} /></Dim>}
\DoxyCodeLine{    <Dim type=\textcolor{stringliteral}{"TopEdge"}><AbsoluteDim value=\textcolor{stringliteral}{"0"} /></Dim>}
\DoxyCodeLine{    <Dim type=\textcolor{stringliteral}{"Width"}><UnifiedDim scale=\textcolor{stringliteral}{"1.0"} /></Dim>}
\DoxyCodeLine{    <Dim type=\textcolor{stringliteral}{"Height"}><UnifiedDim scale=\textcolor{stringliteral}{"1.0"} /></Dim>}
\DoxyCodeLine{  </Area>}
\DoxyCodeLine{</NamedArea>}
\end{DoxyCode}
\hypertarget{fal_element_ref_fal_elem_ref_sec_4}{}\subsection{$<$\+Area\+Property$>$ Element}\label{fal_element_ref_fal_elem_ref_sec_4}
\hypertarget{fal_element_ref_fal_elem_ref_sec_4_1}{}\subsubsection{Purpose\+:}\label{fal_element_ref_fal_elem_ref_sec_4_1}
The {\ttfamily $<$Area\+Property$>$} element is intended to allow the system to access a property on the target window to obtain the final target area of a component being defined.\hypertarget{fal_element_ref_fal_elem_ref_sec_4_2}{}\subsubsection{Attributes\+:}\label{fal_element_ref_fal_elem_ref_sec_4_2}
\begin{DoxyItemize}
\item {\ttfamily name} specifies the name of the property to access. The named property must access a U\+Rect value. Required attribute.\end{DoxyItemize}
\hypertarget{fal_element_ref_fal_elem_ref_sec_4_3}{}\subsubsection{Usage\+:}\label{fal_element_ref_fal_elem_ref_sec_4_3}
\begin{DoxyItemize}
\item The {\ttfamily $<$Area\+Property$>$} element may not contain sub-\/elements. \item The {\ttfamily $<$Area\+Property$>$} element may appear as a sub-\/element only within the main {\ttfamily $<$Area$>$} element.\end{DoxyItemize}
\hypertarget{fal_element_ref_fal_elem_ref_sec_4_4}{}\subsubsection{Examples\+:}\label{fal_element_ref_fal_elem_ref_sec_4_4}
\hypertarget{fal_element_ref_fal_elem_ref_sec_5}{}\subsection{$<$\+Child$>$ Element}\label{fal_element_ref_fal_elem_ref_sec_5}
\hypertarget{fal_element_ref_fal_elem_ref_sec_5_1}{}\subsubsection{Purpose\+:}\label{fal_element_ref_fal_elem_ref_sec_5_1}
The {\ttfamily $<$Child$>$} element defines a component widget that will be created and added to each instance of any window using the {\ttfamily $<$Widget\+Look$>$} being defined. Some base widgets have requirements for {\ttfamily $<$Child$>$} element definition that must be provided.\hypertarget{fal_element_ref_fal_elem_ref_sec_5_2}{}\subsubsection{Attributes\+:}\label{fal_element_ref_fal_elem_ref_sec_5_2}
\begin{DoxyItemize}
\item {\ttfamily type} specifies the widget type to create. Required attribute. \item {\ttfamily name\+Suffix} specifies a suffix which will be used when naming the widget. The final name of the child widget will be that of the parent with this suffix appended. Required attribute. \item {\ttfamily look} specifies the name of a widget look to apply to the child widget. You should only use this if \textquotesingle{}type\textquotesingle{} specifies a Falagard base widget type. Optional attribute. \item {\ttfamily auto\+Window} specifies whether this child window is to be flagged as an auto-\/window. Optional attribute, default value is {\ttfamily true}.\end{DoxyItemize}
\hypertarget{fal_element_ref_fal_elem_ref_sec_5_3}{}\subsubsection{Usage\+:}\label{fal_element_ref_fal_elem_ref_sec_5_3}
Note\+: the sub-\/elements should appear in the order that they are defined here.


\begin{DoxyItemize}
\item The {\ttfamily $<$Child$>$} element may contain any number of {\ttfamily $<$Event\+Action$>$} elements that define actions to be taken by the containing widget in response to events being fired on the child widget being defined. 
\item The {\ttfamily $<$Child$>$} element must contain an {\ttfamily $<$Area$>$} element that defines the location of the child widget in relation to the component being defined. 
\item You may optionally specify a single {\ttfamily $<$Vert\+Alignment$>$} element to set the vertical alignment for the child. 
\item You may optionally specify a single {\ttfamily $<$Horz\+Alignment$>$} element to set the horizontal alignment for the child. 
\item You may specify any number of {\ttfamily $<$Property$>$} elements to set default values for any property supported by the widget type being used for the child. 
\item The {\ttfamily $<$Child$>$} element may only appear within the {\ttfamily $<$Widget\+Look$>$} element. 
\end{DoxyItemize}\hypertarget{fal_element_ref_fal_elem_ref_sec_5_4}{}\subsubsection{Examples\+:}\label{fal_element_ref_fal_elem_ref_sec_5_4}
In this example, taken from Taharez\+Look.\+looknfeel, we see how the title bar child widget required by the frame window type is defined\+: 
\begin{DoxyCode}{0}
\DoxyCodeLine{<WidgetLook name=\textcolor{stringliteral}{"TaharezLook/FrameWindow"}>}
\DoxyCodeLine{  ...}
\DoxyCodeLine{  <Child type=\textcolor{stringliteral}{"TaharezLook/Titlebar"} nameSuffix=\textcolor{stringliteral}{"\_\_auto\_titlebar\_\_"}>}
\DoxyCodeLine{    <Area>}
\DoxyCodeLine{      <Dim type=\textcolor{stringliteral}{"LeftEdge"} ><AbsoluteDim value=\textcolor{stringliteral}{"0"} /></Dim>}
\DoxyCodeLine{      <Dim type=\textcolor{stringliteral}{"TopEdge"} ><AbsoluteDim value=\textcolor{stringliteral}{"0"} /></Dim>}
\DoxyCodeLine{      <Dim type=\textcolor{stringliteral}{"Width"} ><UnifiedDim scale=\textcolor{stringliteral}{"1"} type=\textcolor{stringliteral}{"Width"} /></Dim>}
\DoxyCodeLine{      <Dim type=\textcolor{stringliteral}{"Height"} >}
\DoxyCodeLine{        <OperatorDim op=\textcolor{stringliteral}{"Multiply"}>}
\DoxyCodeLine{          <FontDim type=\textcolor{stringliteral}{"LineSpacing"} />}
\DoxyCodeLine{          <AbsoluteDim value=\textcolor{stringliteral}{"1.5"} />}
\DoxyCodeLine{        </OperatorDim>}
\DoxyCodeLine{      </Dim>}
\DoxyCodeLine{    </Area>}
\DoxyCodeLine{    <Property name=\textcolor{stringliteral}{"AlwaysOnTop"} value=\textcolor{stringliteral}{"False"} />}
\DoxyCodeLine{  </Child>}
\DoxyCodeLine{  ...}
\DoxyCodeLine{</WidgetLook>}
\end{DoxyCode}
\hypertarget{fal_element_ref_fal_elem_ref_sec_6}{}\subsection{$<$\+Colour\+Property$>$ Element}\label{fal_element_ref_fal_elem_ref_sec_6}
\hypertarget{fal_element_ref_fal_elem_ref_sec_6_1}{}\subsubsection{Purpose\+:}\label{fal_element_ref_fal_elem_ref_sec_6_1}
The {\ttfamily $<$Colour\+Property$>$} element is intended to allow the system to access a property on the target window to obtain colour information to be used when drawing some part of the component being defined.\hypertarget{fal_element_ref_fal_elem_ref_sec_6_2}{}\subsubsection{Attributes\+:}\label{fal_element_ref_fal_elem_ref_sec_6_2}
\begin{DoxyItemize}
\item {\ttfamily name} specifies the name of the property to access. The named property must access a single colour value. Required attribute.\end{DoxyItemize}
\hypertarget{fal_element_ref_fal_elem_ref_sec_6_3}{}\subsubsection{Usage\+:}\label{fal_element_ref_fal_elem_ref_sec_6_3}

\begin{DoxyItemize}
\item The {\ttfamily $<$Colour\+Property$>$} element may not contain sub-\/elements.


\item The {\ttfamily $<$Colour\+Property$>$} element may appear as a sub-\/element within any of the following elements\+: 
\begin{DoxyItemize}
\item {\ttfamily $<$Imagery\+Component$>$} to specify a modulating colour to be applied when rendering the image. 
\item {\ttfamily $<$Imagery\+Section$>$} to specify a modulating colour to be applied to all imagery components within the imagery section as it is rendered. 
\item {\ttfamily $<$Section$>$} to specify a modulating colour to be applied to all imagery in the named section as it is rendered. 
\item {\ttfamily $<$Text\+Component$>$} to specify a colour to use when rendering the text component. 
\item {\ttfamily $<$Frame\+Component$>$} to specify a colour to use when rendering the text frame. 
\end{DoxyItemize}
\end{DoxyItemize}\hypertarget{fal_element_ref_fal_elem_ref_sec_6_4}{}\subsubsection{Examples\+:}\label{fal_element_ref_fal_elem_ref_sec_6_4}
The following example, listing imagery for a button in the \char`\"{}\+Normal\char`\"{} state, shows the {\ttfamily $<$Colour\+Property$>$} element in use to specify a property where colours to be used when rendering the Imagery\+Section named \textquotesingle{}label\textquotesingle{} can be found\+: 
\begin{DoxyCode}{0}
\DoxyCodeLine{<StateImagery name=\textcolor{stringliteral}{"Normal"}>}
\DoxyCodeLine{  <Layer>}
\DoxyCodeLine{    <Section section=\textcolor{stringliteral}{"normal"} />}
\DoxyCodeLine{    <Section section=\textcolor{stringliteral}{"label"}>}
\DoxyCodeLine{      <ColourProperty name=\textcolor{stringliteral}{"NormalTextColour"} />}
\DoxyCodeLine{    </Section>}
\DoxyCodeLine{  </Layer>}
\DoxyCodeLine{</StateImagery>}
\end{DoxyCode}
\hypertarget{fal_element_ref_fal_elem_ref_sec_7}{}\subsection{$<$\+Colour\+Rect\+Property$>$ Element}\label{fal_element_ref_fal_elem_ref_sec_7}
\hypertarget{fal_element_ref_fal_elem_ref_sec_7_1}{}\subsubsection{Purpose\+:}\label{fal_element_ref_fal_elem_ref_sec_7_1}
The {\ttfamily $<$Colour\+Rect\+Property} $>$ element is intended to allow the system to access a property on the target window to obtain colour information to be used when drawing some part of the component being defined.\hypertarget{fal_element_ref_fal_elem_ref_sec_7_2}{}\subsubsection{Attributes\+:}\label{fal_element_ref_fal_elem_ref_sec_7_2}
\begin{DoxyItemize}
\item {\ttfamily name} specifies the name of the property to access. The named property must access a Colour\+Rect value. Required attribute.\end{DoxyItemize}
\hypertarget{fal_element_ref_fal_elem_ref_sec_7_3}{}\subsubsection{Usage\+:}\label{fal_element_ref_fal_elem_ref_sec_7_3}

\begin{DoxyItemize}
\item The {\ttfamily $<$Colour\+Rect\+Property$>$} element may not contain sub-\/elements.


\item The {\ttfamily $<$Colour\+Rect\+Property$>$} element may appear as a sub-\/element within any of the following elements\+: 
\begin{DoxyItemize}
\item {\ttfamily $<$Imagery\+Component$>$} to specify a modulating Colour\+Rect to be applied when rendering the image. 
\item {\ttfamily $<$Imagery\+Section$>$} to specify a modulating Colour\+Rect to be applied to all imagery components within the imagery section as it is rendered. 
\item {\ttfamily $<$Section$>$} to specify a modulating Colour\+Rect to be applied to all imagery in the named section as it is rendered. 
\item {\ttfamily $<$Text\+Component$>$} to specify a Colour\+Rect to use when rendering the text component. 
\item {\ttfamily $<$Frame\+Component$>$} to specify a colour to use when rendering the text frame. 
\end{DoxyItemize}
\end{DoxyItemize}\hypertarget{fal_element_ref_fal_elem_ref_sec_7_4}{}\subsubsection{Examples\+:}\label{fal_element_ref_fal_elem_ref_sec_7_4}

\begin{DoxyCode}{0}
\DoxyCodeLine{...}
\DoxyCodeLine{<StateImagery name=\textcolor{stringliteral}{"SpecialState"}>}
\DoxyCodeLine{  <Layer>}
\DoxyCodeLine{    <Section section=\textcolor{stringliteral}{"special\_main"}>}
\DoxyCodeLine{      <ColourRectProperty name=\textcolor{stringliteral}{"SpecialColours"} />}
\DoxyCodeLine{    </Section>}
\DoxyCodeLine{  </Layer>}
\DoxyCodeLine{</StateImagery>}
\DoxyCodeLine{...}
\end{DoxyCode}
\hypertarget{fal_element_ref_fal_elem_ref_sec_8}{}\subsection{$<$\+Colours$>$ Element}\label{fal_element_ref_fal_elem_ref_sec_8}
\hypertarget{fal_element_ref_fal_elem_ref_sec_8_1}{}\subsubsection{Purpose\+:}\label{fal_element_ref_fal_elem_ref_sec_8_1}
The {\ttfamily $<$Colours$>$} element is used to explicitly specify values for a Colour\+Rect that should be used when rendering some part of the component being defined.\hypertarget{fal_element_ref_fal_elem_ref_sec_8_2}{}\subsubsection{Attributes\+:}\label{fal_element_ref_fal_elem_ref_sec_8_2}
\begin{DoxyItemize}
\item {\ttfamily top\+Left} specifies a hex colour value, of the form \char`\"{}\+A\+A\+R\+R\+G\+G\+B\+B\char`\"{}, to be used for the top-\/left corner of the Colour\+Rect. Required attribute. \item {\ttfamily top\+Right} specifies a hex colour value, of the form \char`\"{}\+A\+A\+R\+R\+G\+G\+B\+B\char`\"{}, to be used for the top-\/right corner of the Colour\+Rect. Required attribute. \item {\ttfamily bottom\+Left} specifies a hex colour value, of the form \char`\"{}\+A\+A\+R\+R\+G\+G\+B\+B\char`\"{}, to be used for the bottom-\/left corner of the Colour\+Rect. Required attribute. \item {\ttfamily bottom\+Right} specifies a hex colour value of the form \char`\"{}\+A\+A\+R\+R\+G\+G\+B\+B\char`\"{}, to be used for the bottom-\/right corner of the Colour\+Rect. Required attribute.\end{DoxyItemize}
\hypertarget{fal_element_ref_fal_elem_ref_sec_8_3}{}\subsubsection{Usage\+:}\label{fal_element_ref_fal_elem_ref_sec_8_3}

\begin{DoxyItemize}
\item The {\ttfamily $<$Colours$>$} element may not contain sub-\/elements.


\item The {\ttfamily $<$Colours$>$} element may appear as a sub-\/element within any of the following elements\+: 
\begin{DoxyItemize}
\item {\ttfamily $<$Imagery\+Component$>$} to specify a modulating Colour\+Rect to be applied when rendering the image. 
\item {\ttfamily $<$Imagery\+Section$>$} to specify a modulating Colour\+Rect to be applied to all imagery components within the imagery section as it is rendered. 
\item {\ttfamily $<$Section$>$} to specify a modulating Colour\+Rect to be applied to all imagery in the named section as it is rendered. 
\item {\ttfamily $<$Text\+Component$>$} to specify a Colour\+Rect to use when rendering the text component. 
\item {\ttfamily $<$Frame\+Component$>$} to specify a colour to use when rendering the text frame. 
\end{DoxyItemize}
\end{DoxyItemize}\hypertarget{fal_element_ref_fal_elem_ref_sec_8_4}{}\subsubsection{Examples\+:}\label{fal_element_ref_fal_elem_ref_sec_8_4}
In this example, we see the {\ttfamily $<$Colours$>$} element used to specify the value \textquotesingle{}F\+F\+F\+F\+F\+F00\textquotesingle{} as the colour for all four corners of the colour rect to be used when rendering the image being defined\+: 
\begin{DoxyCode}{0}
\DoxyCodeLine{...}
\DoxyCodeLine{<ImageryComponent>}
\DoxyCodeLine{  <Area>}
\DoxyCodeLine{    <Dim type=\textcolor{stringliteral}{"LeftEdge"} ><AbsoluteDim value=\textcolor{stringliteral}{"0"} /></Dim>}
\DoxyCodeLine{    <Dim type=\textcolor{stringliteral}{"TopEdge"} ><AbsoluteDim value=\textcolor{stringliteral}{"0"} /></Dim>}
\DoxyCodeLine{    <Dim type=\textcolor{stringliteral}{"Width"} ><AbsoluteDim value=\textcolor{stringliteral}{"12"} /></Dim>}
\DoxyCodeLine{    <Dim type=\textcolor{stringliteral}{"Height"} ><AbsoluteDim value=\textcolor{stringliteral}{"24"} /></Dim>}
\DoxyCodeLine{  </Area>}
\DoxyCodeLine{  <Image imageset=\textcolor{stringliteral}{"newImageset"} image=\textcolor{stringliteral}{"FunkyComponent"} />}
\DoxyCodeLine{  <Colours}
\DoxyCodeLine{    topLeft=\textcolor{stringliteral}{"FFFFFF00"}}
\DoxyCodeLine{    topRight=\textcolor{stringliteral}{"FFFFFF00"}}
\DoxyCodeLine{    bottomLeft=\textcolor{stringliteral}{"FFFFFF00"}}
\DoxyCodeLine{    bottomRight=\textcolor{stringliteral}{"FFFFFF00"}}
\DoxyCodeLine{  />}
\DoxyCodeLine{  <VertFormat type=\textcolor{stringliteral}{"Stretched"} />}
\DoxyCodeLine{  <HorzFormat type=\textcolor{stringliteral}{"Stretched"} />}
\DoxyCodeLine{</ImageryComponent>}
\DoxyCodeLine{...}
\end{DoxyCode}
\hypertarget{fal_element_ref_fal_elem_ref_sec_9}{}\subsection{$<$\+Dim$>$ Element}\label{fal_element_ref_fal_elem_ref_sec_9}
\hypertarget{fal_element_ref_fal_elem_ref_sec_9_1}{}\subsubsection{Purpose\+:}\label{fal_element_ref_fal_elem_ref_sec_9_1}
The {\ttfamily $<$Dim$>$} element is intended as a container element for a single dimension of an area rectangle.\hypertarget{fal_element_ref_fal_elem_ref_sec_9_2}{}\subsubsection{Attributes\+:}\label{fal_element_ref_fal_elem_ref_sec_9_2}
\begin{DoxyItemize}
\item {\ttfamily type} specifies what the dimension being defined represents. This attribute should be set to one of the values defined for the Dimension\+Type enumeration (see below). Required attribute.\end{DoxyItemize}
\hypertarget{fal_element_ref_fal_elem_ref_sec_9_3}{}\subsubsection{Usage\+:}\label{fal_element_ref_fal_elem_ref_sec_9_3}

\begin{DoxyItemize}
\item The {\ttfamily $<$Dim$>$} element may only appear within the {\ttfamily $<$Area$>$} element.


\item The {\ttfamily $<$Dim$>$} element may contain any of the following specialised dimension elements\+: 
\begin{DoxyItemize}
\item {\ttfamily $<$Absolute\+Dim$>$} 
\item {\ttfamily $<$Font\+Dim$>$} 
\item {\ttfamily $<$Image\+Dim$>$} 
\item {\ttfamily $<$Image\+Property\+Dim$>$} 
\item {\ttfamily $<$Property\+Dim$>$} 
\item {\ttfamily $<$Unified\+Dim$>$} 
\item {\ttfamily $<$Widget\+Dim$>$} 
\item {\ttfamily $<$Operator\+Dim$>$} 
\end{DoxyItemize}
\end{DoxyItemize}\hypertarget{fal_element_ref_fal_elem_ref_sec_9_4}{}\subsubsection{Examples\+:}\label{fal_element_ref_fal_elem_ref_sec_9_4}
\hypertarget{fal_element_ref_fal_elem_ref_eventaction}{}\subsection{$<$\+Event\+Action$>$ Element}\label{fal_element_ref_fal_elem_ref_eventaction}
\hypertarget{fal_element_ref_fal_elem_ref_eventaction_1}{}\subsubsection{Purpose\+:}\label{fal_element_ref_fal_elem_ref_eventaction_1}
The {\ttfamily $<$Event\+Action$>$} element is used to specify a predefined action that should be performed by the containing widget for the child widget being defined, in response to a specified event being fired by the child widget. \begin{DoxyParagraph}{This is useful in situations where the containing widget has imagery or}
other content that is tied to some state on the child wiget. By using the facilities provided by $<$Event\+Action$>$ the containing widget is updated on demand in response to changes on the child.
\end{DoxyParagraph}
\hypertarget{fal_element_ref_fal_elem_ref_eventaction_2}{}\subsubsection{Attributes\+:}\label{fal_element_ref_fal_elem_ref_eventaction_2}
\begin{DoxyItemize}
\item {\ttfamily event} specifies the name of the event on the Child component being defined that will trigger the action. Required attribute. \item {\ttfamily action} specifies one of the values defined for the \mbox{\hyperlink{fal_enum_ref_fal_enum_ref_sec_12}{Child\+Event\+Action}} enumeration. Required attribute.\end{DoxyItemize}
\hypertarget{fal_element_ref_fal_elem_ref_eventaction_3}{}\subsubsection{Usage\+:}\label{fal_element_ref_fal_elem_ref_eventaction_3}

\begin{DoxyItemize}
\item The {\ttfamily $<$Event\+Action$>$} element may not conatin any sub-\/elements. 
\item The {\ttfamily $<$Event\+Action$>$} element may only appear as a sub-\/element within {\ttfamily $<$Child$>$} elements. 
\end{DoxyItemize}\hypertarget{fal_element_ref_fal_elem_ref_eventaction_4}{}\subsubsection{Examples\+:}\label{fal_element_ref_fal_elem_ref_eventaction_4}
In this example, an Event\+Action is defined for the thumb of a slider widget. The definition indicates that when the {\ttfamily Moved} is fired for the thumb widget, the containing slider should redraw itself. This is useful in scenarios where the slider has imagery whose position is dependent upon the position of the thumb -\/ by using the $<$Event\+Action$>$ system, the required updates are fully automated. 
\begin{DoxyCode}{0}
\DoxyCodeLine{<Falagard>}
\DoxyCodeLine{  <WidgetLook name=\textcolor{stringliteral}{"ANewLook/Slider"}>}
\DoxyCodeLine{  ...}
\DoxyCodeLine{    <Child  type=\textcolor{stringliteral}{"ANewLook/SliderThumb"} nameSuffix=\textcolor{stringliteral}{"\_\_auto\_thumb\_\_"}>}
\DoxyCodeLine{      <EventAction \textcolor{keyword}{event}=\textcolor{stringliteral}{"Moved"} action=\textcolor{stringliteral}{"Redraw"} />}
\DoxyCodeLine{      <Area>}
\DoxyCodeLine{        <Dim type=\textcolor{stringliteral}{"LeftEdge"} ><AbsoluteDim value=\textcolor{stringliteral}{"0"} /></Dim>}
\DoxyCodeLine{        ...}
\DoxyCodeLine{      </Area>}
\DoxyCodeLine{      ...}
\DoxyCodeLine{  </WidgetLook>}
\DoxyCodeLine{  ...}
\DoxyCodeLine{</Falagard>}
\end{DoxyCode}
\hypertarget{fal_element_ref_fal_elem_ref_eventlinkdefintion}{}\subsection{$<$\+Event\+Link\+Definition$>$ Element}\label{fal_element_ref_fal_elem_ref_eventlinkdefintion}
\hypertarget{fal_element_ref_fal_elem_ref_eventlinkdefintion_1}{}\subsubsection{Purpose\+:}\label{fal_element_ref_fal_elem_ref_eventlinkdefintion_1}
The {\ttfamily $<$Event\+Link\+Definition$>$} element is used to define an event on a window using the {\ttfamily $<$Widget\+Look$>$} being defined. The new event can be linked to other events on the window, or to events on child windows. For child windows, the window must exist, and the facility is largely intended to be able to link to events on windows defined via the {\ttfamily $<$Child$>$} element.\hypertarget{fal_element_ref_fal_elem_ref_eventlinkdefintion_2}{}\subsubsection{Attributes\+:}\label{fal_element_ref_fal_elem_ref_eventlinkdefintion_2}
\begin{DoxyItemize}
\item {\ttfamily name} specifies the unique name to use for the new event. Required attribute. \item {\ttfamily widget} specifies the name suffix of the child widget containing an event that will be linked to the new event being defined. Optional attribute. \item {\ttfamily event} specifies the name of an event on widget identified via @cwidget that is to be linked to the new event being defined. If this is omitted but {\ttfamily widget} is specified, it will be assumed that the event to link to has the same name as the event being defined. Optional attribute.\end{DoxyItemize}
\hypertarget{fal_element_ref_fal_elem_ref_eventlinkdefintion_3}{}\subsubsection{Usage\+:}\label{fal_element_ref_fal_elem_ref_eventlinkdefintion_3}

\begin{DoxyItemize}
\item The {\ttfamily $<$Event\+Link\+Definition$>$} element may contain any number of {\ttfamily $<$Event\+Link\+Target$>$} sub-\/elements. 
\item The {\ttfamily $<$Event\+Link\+Definition$>$} element may only appear as a sub-\/element within {\ttfamily $<$Widget\+Look$>$} elements. 
\end{DoxyItemize}\hypertarget{fal_element_ref_fal_elem_ref_eventlinkdefintion_4}{}\subsubsection{Examples\+:}\label{fal_element_ref_fal_elem_ref_eventlinkdefintion_4}
T\+O\+DO\hypertarget{fal_element_ref_fal_elem_ref_eventlinktarget}{}\subsection{$<$\+Event\+Link\+Target$>$ Element}\label{fal_element_ref_fal_elem_ref_eventlinktarget}
\hypertarget{fal_element_ref_fal_elem_ref_eventlinktarget_1}{}\subsubsection{Purpose\+:}\label{fal_element_ref_fal_elem_ref_eventlinktarget_1}
The {\ttfamily $<$Event\+Link\+Target$>$} element specifies a target widget suffix and event name to be used with a {\ttfamily $<$Event\+Link\+Definition$>$}. Whenever the event specified here fires, the event defined by the enclosing {\ttfamily $<$Event\+Link\+Definition$>$} will also fire.\hypertarget{fal_element_ref_fal_elem_ref_eventlinktarget_2}{}\subsubsection{Attributes\+:}\label{fal_element_ref_fal_elem_ref_eventlinktarget_2}
\begin{DoxyItemize}
\item {\ttfamily widget} specifies the name suffix of the child widget containing an event that will be linked to the new event being defined by the enclosing {\ttfamily $<$Event\+Link\+Definition$>$}. Optional attribute. \item {\ttfamily event} specifies the name of an event on the widget identified via @cwidget that is to be linked to the event being defined by the enclosing {\ttfamily $<$Event\+Link\+Definition$>$}. If this is omitted but {\ttfamily widget} is specified, it will be assumed that the event to link to has the same name as the event being defined. Optional attribute.\end{DoxyItemize}
\hypertarget{fal_element_ref_fal_elem_ref_eventlinktarget_3}{}\subsubsection{Usage\+:}\label{fal_element_ref_fal_elem_ref_eventlinktarget_3}

\begin{DoxyItemize}
\item The {\ttfamily $<$Event\+Link\+Target$>$} element may not contain sub-\/elements. 
\item The {\ttfamily $<$Event\+Link\+Target$>$} element may only appear as a sub-\/element within {\ttfamily $<$Event\+Link\+Definition$>$} elements. 
\end{DoxyItemize}\hypertarget{fal_element_ref_fal_elem_ref_eventlinktarget_4}{}\subsubsection{Examples\+:}\label{fal_element_ref_fal_elem_ref_eventlinktarget_4}
T\+O\+DO\hypertarget{fal_element_ref_fal_elem_ref_sec_11}{}\subsection{$<$\+Falagard$>$ Element}\label{fal_element_ref_fal_elem_ref_sec_11}
\hypertarget{fal_element_ref_fal_elem_ref_sec_11_1}{}\subsubsection{Purpose\+:}\label{fal_element_ref_fal_elem_ref_sec_11_1}
The {\ttfamily $<$Falagard$>$} element is the root element in Falagard X\+ML skin definition files. The element serves mainly as a container for {\ttfamily $<$Widget\+Look$>$} elements\hypertarget{fal_element_ref_fal_elem_ref_sec_11_2}{}\subsubsection{Attributes\+:}\label{fal_element_ref_fal_elem_ref_sec_11_2}
\begin{DoxyItemize}
\item {\ttfamily version} specifies the version of the resource file. Should be specified for all files, current C\+E\+G\+UI falagard version is\+: 7\end{DoxyItemize}
\hypertarget{fal_element_ref_fal_elem_ref_sec_11_3}{}\subsubsection{Usage\+:}\label{fal_element_ref_fal_elem_ref_sec_11_3}

\begin{DoxyItemize}
\item The {\ttfamily $<$Falagard$>$} element is the root element for Falagard skin files. 
\item The {\ttfamily $<$Falagard$>$} element may contain any number of {\ttfamily $<$Widget\+Look$>$} elements. 
\item No element may contain {\ttfamily $<$Falagard$>$} elements as a sub-\/element. 
\end{DoxyItemize}\hypertarget{fal_element_ref_fal_elem_ref_sec_11_4}{}\subsubsection{Examples\+:}\label{fal_element_ref_fal_elem_ref_sec_11_4}
Here we just see the general structure of a Falagard X\+ML file, notice that the {\ttfamily $<$Falagard$>$} element just serves as a container for multiple {\ttfamily $<$Widget\+Look$>$} elements\+: 
\begin{DoxyCode}{0}
\DoxyCodeLine{<?xml version=\textcolor{stringliteral}{"1.0"} ?>}
\DoxyCodeLine{<Falagard>}
\DoxyCodeLine{  <WidgetLook name=\textcolor{stringliteral}{"TaharezLook/Button"}>}
\DoxyCodeLine{  ...}
\DoxyCodeLine{  </WidgetLook>}
\DoxyCodeLine{  <WidgetLook ... >}
\DoxyCodeLine{  ...}
\DoxyCodeLine{  </WidgetLook>}
\DoxyCodeLine{  ...}
\DoxyCodeLine{</Falagard>}
\end{DoxyCode}
\hypertarget{fal_element_ref_fal_elem_ref_sec_12}{}\subsection{$<$\+Font\+Dim$>$ Element}\label{fal_element_ref_fal_elem_ref_sec_12}
\hypertarget{fal_element_ref_fal_elem_ref_sec_12_1}{}\subsubsection{Purpose\+:}\label{fal_element_ref_fal_elem_ref_sec_12_1}
The {\ttfamily $<$Font\+Dim$>$} element is used to take some measurement of a Font, and use it as a dimension component of an area rectangle.\hypertarget{fal_element_ref_fal_elem_ref_sec_12_2}{}\subsubsection{Attributes\+:}\label{fal_element_ref_fal_elem_ref_sec_12_2}
\begin{DoxyItemize}
\item {\ttfamily widget} specifies the name suffix of a child window to access when automatically obtaining the font or text string to be used when calculating the dimension\textquotesingle{}s value. The final name used to access the widget will be that of the target window with this suffix appended. If this suffix is not specified, the target window itself is used. Optional attribute. \item {\ttfamily type} specifies the type of font metric / measurement to use for this dimension. This should be set to one of the values from the Font\+Metric\+Type enumeration. Required attribute. \item {\ttfamily font} specifies the name of a font. If no font is given, the font will be taken from the target window at the time the dimension\textquotesingle{}s value is taken. Optional attribute. \item {\ttfamily string} For horizontal extents measurement, specifies the string to be measured. If no explicit string is given, the window text for the target window at the time the dimension\textquotesingle{}s value is taken will be used instead. Optional attribute. \item {\ttfamily padding} an absolute pixel \textquotesingle{}padding\textquotesingle{} value to be added to the font metric value. Optional attribute.\end{DoxyItemize}
\hypertarget{fal_element_ref_fal_elem_ref_sec_12_3}{}\subsubsection{Usage\+:}\label{fal_element_ref_fal_elem_ref_sec_12_3}

\begin{DoxyItemize}
\item The {\ttfamily $<$Font\+Dim$>$} element can appear as a sub-\/element in {\ttfamily $<$Dim$>$} to form a dimension specification for an area. 
\item The {\ttfamily $<$Font\+Dim$>$} element can appear as a sub-\/element of {\ttfamily $<$Operator\+Dim$>$} to specify one of the operands for a dimension calculation. 
\end{DoxyItemize}\hypertarget{fal_element_ref_fal_elem_ref_sec_12_4}{}\subsubsection{Examples\+:}\label{fal_element_ref_fal_elem_ref_sec_12_4}
This first example just gets the line spacing for the window\textquotesingle{}s current font\+: 
\begin{DoxyCode}{0}
\DoxyCodeLine{<Dim type=\textcolor{stringliteral}{"Height"}>}
\DoxyCodeLine{  <FontDim type=\textcolor{stringliteral}{"LineSpacing"} />}
\DoxyCodeLine{</Dim>}
\end{DoxyCode}


Now we take an extents measurement of the windows current text, using a specified font, and pad the result by ten pixels\+: 
\begin{DoxyCode}{0}
\DoxyCodeLine{<Dim type=\textcolor{stringliteral}{"Width"}>}
\DoxyCodeLine{  <FontDim type=\textcolor{stringliteral}{"HorzExtent"} font=\textcolor{stringliteral}{"Roman-14"} padding=\textcolor{stringliteral}{"10"} />}
\DoxyCodeLine{</Dim>}
\end{DoxyCode}
\hypertarget{fal_element_ref_fal_elem_ref_sec_13}{}\subsection{$<$\+Font\+Property$>$ Element}\label{fal_element_ref_fal_elem_ref_sec_13}
\hypertarget{fal_element_ref_fal_elem_ref_sec_13_1}{}\subsubsection{Purpose\+:}\label{fal_element_ref_fal_elem_ref_sec_13_1}
The {\ttfamily $<$Font\+Property$>$} element is intended to allow the system to access a property on the target window to obtain the font to be used when rendering the Text\+Component being defined.\hypertarget{fal_element_ref_fal_elem_ref_sec_13_2}{}\subsubsection{Attributes\+:}\label{fal_element_ref_fal_elem_ref_sec_13_2}
\begin{DoxyItemize}
\item {\ttfamily name} specifies the name of the property to access. Required attribute. The value of the named property is taken as being the name of a Font.\end{DoxyItemize}
\hypertarget{fal_element_ref_fal_elem_ref_sec_13_3}{}\subsubsection{Usage\+:}\label{fal_element_ref_fal_elem_ref_sec_13_3}

\begin{DoxyItemize}
\item The {\ttfamily $<$Font\+Property$>$} element may not contain sub-\/elements. 
\item The {\ttfamily $<$Font\+Property$>$} element may appear as a sub-\/element only within the {\ttfamily $<$Text\+Component$>$} element. 
\end{DoxyItemize}\hypertarget{fal_element_ref_fal_elem_ref_sec_13_4}{}\subsubsection{Examples\+:}\label{fal_element_ref_fal_elem_ref_sec_13_4}
\hypertarget{fal_element_ref_fal_elem_ref_sec_14}{}\subsection{$<$\+Frame\+Component$>$ Element}\label{fal_element_ref_fal_elem_ref_sec_14}
\hypertarget{fal_element_ref_fal_elem_ref_sec_14_1}{}\subsubsection{Purpose\+:}\label{fal_element_ref_fal_elem_ref_sec_14_1}
The {\ttfamily $<$Frame\+Component$>$} element is used to define an imagery frame using a maximum of eight images for the corners and edges, and a single, formatted, image for the background. Any of the images may be omitted if not required.\hypertarget{fal_element_ref_fal_elem_ref_sec_14_2}{}\subsubsection{Attributes\+:}\label{fal_element_ref_fal_elem_ref_sec_14_2}
No attributes are currently defined for the {\ttfamily $<$Frame\+Component$>$} element.\hypertarget{fal_element_ref_fal_elem_ref_sec_14_3}{}\subsubsection{Usage\+:}\label{fal_element_ref_fal_elem_ref_sec_14_3}
Note\+: the sub-\/elements should appear in the order that they are defined here.


\begin{DoxyItemize}
\item {\ttfamily $<$Area$>$} defining the target area for this frame. 
\item Up to nine {\ttfamily $<$Image$>$} or {\ttfamily $<$Image\+Property$>$} elements specifying the images to be drawn and in what positions. Note that it is acceptable to freely mix {\ttfamily $<$Image$>$} and {\ttfamily $<$Image\+Property$>$} in a single {\ttfamily $<$Frame\+Component$>$} definition.


\item Optionally specifying the colours for the entire frame, one of the colour elements\+: 
\begin{DoxyItemize}
\item {\ttfamily $<$Colours$>$} 
\item {\ttfamily $<$Colour\+Property$>$} 
\item {\ttfamily $<$Colour\+Rect\+Property$>$} 
\end{DoxyItemize}


\item Optionally, to specify the vertical formatting to use for the frame background, either of\+: 
\begin{DoxyItemize}
\item {\ttfamily $<$Vert\+Format$>$} 
\item {\ttfamily $<$Vert\+Format\+Property$>$} 
\end{DoxyItemize}


\item Optionally, to specify the horizontal formatting to use for the frame background, either of\+: 
\begin{DoxyItemize}
\item {\ttfamily $<$Horz\+Format$>$} 
\item {\ttfamily $<$Horz\+Format\+Property$>$} 
\end{DoxyItemize}


\item The {\ttfamily $<$Frame\+Component$>$} element may only appear as a sub-\/element of the element {\ttfamily $<$Imagery\+Section$>$}. 
\end{DoxyItemize}\hypertarget{fal_element_ref_fal_elem_ref_sec_14_4}{}\subsubsection{Examples\+:}\label{fal_element_ref_fal_elem_ref_sec_14_4}
The following defines a full frame and background. It is taken from the Taharez\+Look skin specification for the List\+View widget\+:


\begin{DoxyCode}{0}
\DoxyCodeLine{<FrameComponent>}
\DoxyCodeLine{  <Area>}
\DoxyCodeLine{    <Dim type=\textcolor{stringliteral}{"LeftEdge"} ><AbsoluteDim value=\textcolor{stringliteral}{"0"} /></Dim>}
\DoxyCodeLine{    <Dim type=\textcolor{stringliteral}{"TopEdge"} ><AbsoluteDim value=\textcolor{stringliteral}{"0"} /></Dim>}
\DoxyCodeLine{    <Dim type=\textcolor{stringliteral}{"Width"} ><UnifiedDim scale=\textcolor{stringliteral}{"1"} type=\textcolor{stringliteral}{"Width"} /></Dim>}
\DoxyCodeLine{    <Dim type=\textcolor{stringliteral}{"Height"} ><UnifiedDim scale=\textcolor{stringliteral}{"1"} type=\textcolor{stringliteral}{"Height"} /></Dim>}
\DoxyCodeLine{  </Area>}
\DoxyCodeLine{  <Image type=\textcolor{stringliteral}{"TopLeftCorner"}}
\DoxyCodeLine{    imageset=\textcolor{stringliteral}{"TaharezLook"} image=\textcolor{stringliteral}{"ListboxTopLeft"}}
\DoxyCodeLine{  />}
\DoxyCodeLine{  <Image type=\textcolor{stringliteral}{"TopRightCorner"}}
\DoxyCodeLine{    imageset=\textcolor{stringliteral}{"TaharezLook"} image=\textcolor{stringliteral}{"ListboxTopRight"}}
\DoxyCodeLine{  />}
\DoxyCodeLine{  <Image type=\textcolor{stringliteral}{"BottomLeftCorner"}}
\DoxyCodeLine{    imageset=\textcolor{stringliteral}{"TaharezLook"} image=\textcolor{stringliteral}{"ListboxBottomLeft"}}
\DoxyCodeLine{  />}
\DoxyCodeLine{  <Image type=\textcolor{stringliteral}{"BottomRightCorner"}}
\DoxyCodeLine{    imageset=\textcolor{stringliteral}{"TaharezLook"} image=\textcolor{stringliteral}{"ListboxBottomRight"}}
\DoxyCodeLine{  />}
\DoxyCodeLine{  <Image type=\textcolor{stringliteral}{"LeftEdge"}}
\DoxyCodeLine{    imageset=\textcolor{stringliteral}{"TaharezLook"} image=\textcolor{stringliteral}{"ListboxLeft"}}
\DoxyCodeLine{  />}
\DoxyCodeLine{  <Image type=\textcolor{stringliteral}{"RightEdge"}}
\DoxyCodeLine{    imageset=\textcolor{stringliteral}{"TaharezLook"} image=\textcolor{stringliteral}{"ListboxRight"}}
\DoxyCodeLine{  />}
\DoxyCodeLine{  <Image type=\textcolor{stringliteral}{"TopEdge"}}
\DoxyCodeLine{    imageset=\textcolor{stringliteral}{"TaharezLook"} image=\textcolor{stringliteral}{"ListboxTop"}}
\DoxyCodeLine{  />}
\DoxyCodeLine{  <Image type=\textcolor{stringliteral}{"BottomEdge"}}
\DoxyCodeLine{    imageset=\textcolor{stringliteral}{"TaharezLook"} image=\textcolor{stringliteral}{"ListboxBottom"}}
\DoxyCodeLine{  />}
\DoxyCodeLine{  <Image type=\textcolor{stringliteral}{"Background"}}
\DoxyCodeLine{    imageset=\textcolor{stringliteral}{"TaharezLook"} image=\textcolor{stringliteral}{"ListboxBackdrop"}}
\DoxyCodeLine{  />}
\DoxyCodeLine{</FrameComponent>}
\end{DoxyCode}
\hypertarget{fal_element_ref_fal_elem_ref_sec_15}{}\subsection{$<$\+Horz\+Alignment$>$ Element}\label{fal_element_ref_fal_elem_ref_sec_15}
\hypertarget{fal_element_ref_fal_elem_ref_sec_15_1}{}\subsubsection{Purpose\+:}\label{fal_element_ref_fal_elem_ref_sec_15_1}
The {\ttfamily $<$Horz\+Alignment$>$} element is used to specify the horizontal alignment option required for a child window element.\hypertarget{fal_element_ref_fal_elem_ref_sec_15_2}{}\subsubsection{Attributes\+:}\label{fal_element_ref_fal_elem_ref_sec_15_2}
\begin{DoxyItemize}
\item {\ttfamily type} specifies one of the values from the Horizontal\+Alignment enumeration indicating the desired horizontal alignment.\end{DoxyItemize}
\hypertarget{fal_element_ref_fal_elem_ref_sec_15_3}{}\subsubsection{Usage\+:}\label{fal_element_ref_fal_elem_ref_sec_15_3}

\begin{DoxyItemize}
\item The {\ttfamily $<$Horz\+Alignment$>$} element may only appear as a sub-\/element of the {\ttfamily $<$Child$>$} element. 
\item The {\ttfamily $<$Horz\+Alignment$>$} element may not contain any sub-\/elements. 
\end{DoxyItemize}\hypertarget{fal_element_ref_fal_elem_ref_sec_15_4}{}\subsubsection{Examples\+:}\label{fal_element_ref_fal_elem_ref_sec_15_4}
This example defines a scrollbar type child widget. We have used the {\ttfamily $<$Horz\+Alignment$>$} element to specify that the scrollbar appear on the far right edge of the component being defined\+: 
\begin{DoxyCode}{0}
\DoxyCodeLine{...}
\DoxyCodeLine{<Child type=\textcolor{stringliteral}{"MyLook/VertScrollbar"} nameSuffix=\textcolor{stringliteral}{"\_\_auto\_vscrollbar\_\_"}>}
\DoxyCodeLine{  <Area>}
\DoxyCodeLine{    <Dim type=\textcolor{stringliteral}{"LeftEdge"} ><AbsoluteDim value=\textcolor{stringliteral}{"0"} /></Dim>}
\DoxyCodeLine{    <Dim type=\textcolor{stringliteral}{"TopEdge"} ><AbsoluteDim value=\textcolor{stringliteral}{"0"} /></Dim>}
\DoxyCodeLine{    <Dim type=\textcolor{stringliteral}{"Width"} ><AbsoluteDim value=\textcolor{stringliteral}{"15"} /></Dim>}
\DoxyCodeLine{    <Dim type=\textcolor{stringliteral}{"Height"} ><UnifiedDim scale=\textcolor{stringliteral}{"1"} type=\textcolor{stringliteral}{"Height"} /></Dim>}
\DoxyCodeLine{  </Area>}
\DoxyCodeLine{  <HorzAlignment type=\textcolor{stringliteral}{"RightAligned"} />}
\DoxyCodeLine{</Child>}
\DoxyCodeLine{...}
\end{DoxyCode}
\hypertarget{fal_element_ref_fal_elem_ref_sec_16}{}\subsection{$<$\+Horz\+Format$>$ Element}\label{fal_element_ref_fal_elem_ref_sec_16}
\hypertarget{fal_element_ref_fal_elem_ref_sec_16_1}{}\subsubsection{Purpose\+:}\label{fal_element_ref_fal_elem_ref_sec_16_1}
The {\ttfamily $<$Horz\+Format$>$} element is used to specify the required horizontal formatting for an image, frame, or text component.\hypertarget{fal_element_ref_fal_elem_ref_sec_16_2}{}\subsubsection{Attributes\+:}\label{fal_element_ref_fal_elem_ref_sec_16_2}

\begin{DoxyItemize}
\item {\ttfamily type} specifies the required horizontal formatting option. 
\begin{DoxyItemize}
\item For use in Imagery\+Components or Frame\+Components, this will be one of the values from the Horizontal\+Format enumeration. 
\item For use in Text\+Components, this will one of the values form the Horizontal\+Text\+Format enumeration. 
\end{DoxyItemize}
\item {\ttfamily component} Only for Frame\+Component. Specifies the part of the frame that this formatting is to be used for. Should be \char`\"{}\+Top\+Edge\char`\"{}, \char`\"{}\+Bottom\+Edge\char`\"{} or \char`\"{}\+Background\char`\"{} from the Frame\+Image\+Component enumeration. Optional attribute, defaults to \char`\"{}\+Background\char`\"{}. 
\end{DoxyItemize}\hypertarget{fal_element_ref_fal_elem_ref_sec_16_3}{}\subsubsection{Usage\+:}\label{fal_element_ref_fal_elem_ref_sec_16_3}

\begin{DoxyItemize}
\item The {\ttfamily $<$Horz\+Format$>$} element may only appear as a sub-\/element of the following elements\+: 
\begin{DoxyItemize}
\item {\ttfamily $<$Imagery\+Component$>$} 
\item {\ttfamily $<$Frame\+Component$>$} 
\item {\ttfamily $<$Text\+Component$>$} 
\end{DoxyItemize}


\item The {\ttfamily $<$Horz\+Format$>$} element may not contain any sub-\/elements. 
\end{DoxyItemize}\hypertarget{fal_element_ref_fal_elem_ref_sec_16_4}{}\subsubsection{Examples\+:}\label{fal_element_ref_fal_elem_ref_sec_16_4}
This first example shows an Imagery\+Component definition. We use {\ttfamily $<$Horz\+Format$>$} to specify that we want the image stretched to cover the entire width of the designated target area\+: 
\begin{DoxyCode}{0}
\DoxyCodeLine{...}
\DoxyCodeLine{<ImageryComponent>}
\DoxyCodeLine{  <Area>}
\DoxyCodeLine{    <Dim type=\textcolor{stringliteral}{"LeftEdge"} ><AbsoluteDim value=\textcolor{stringliteral}{"0"} /></Dim>}
\DoxyCodeLine{    <Dim type=\textcolor{stringliteral}{"TopEdge"} ><AbsoluteDim value=\textcolor{stringliteral}{"0"} /></Dim>}
\DoxyCodeLine{    <Dim type=\textcolor{stringliteral}{"Width"} ><AbsoluteDim value=\textcolor{stringliteral}{"25"} /></Dim>}
\DoxyCodeLine{    <Dim type=\textcolor{stringliteral}{"Height"} ><AbsoluteDim value=\textcolor{stringliteral}{"25"} /></Dim>}
\DoxyCodeLine{  </Area>}
\DoxyCodeLine{  <Image imageset=\textcolor{stringliteral}{"myImageset"} image=\textcolor{stringliteral}{"coolImage"} />}
\DoxyCodeLine{  <VertFormat type=\textcolor{stringliteral}{"Stretched"} />}
\DoxyCodeLine{  <HorzFormat type=\textcolor{stringliteral}{"Stretched"} />}
\DoxyCodeLine{</ImageryComponent>}
\DoxyCodeLine{...}
\end{DoxyCode}


This second example is for a Text\+Component. You can see {\ttfamily $<$Horz\+Format$>$} used here to specify that we want the text centred within the target area, and word-\/wrapped where required\+: 
\begin{DoxyCode}{0}
\DoxyCodeLine{<TextComponent>}
\DoxyCodeLine{  <Area>}
\DoxyCodeLine{    <Dim type=\textcolor{stringliteral}{"LeftEdge"} ><AbsoluteDim value=\textcolor{stringliteral}{"0"} /></Dim>}
\DoxyCodeLine{    <Dim type=\textcolor{stringliteral}{"TopEdge"} ><AbsoluteDim value=\textcolor{stringliteral}{"0"} /></Dim>}
\DoxyCodeLine{    <Dim type=\textcolor{stringliteral}{"RightEdge"} ><UnifiedDim scale=\textcolor{stringliteral}{"1"} type=\textcolor{stringliteral}{"Width"} /></Dim>}
\DoxyCodeLine{    <Dim type=\textcolor{stringliteral}{"Height"} ><UnifiedDim scale=\textcolor{stringliteral}{"1"} type=\textcolor{stringliteral}{"Height"} /></Dim>}
\DoxyCodeLine{  </Area>}
\DoxyCodeLine{  <HorzFormat type=\textcolor{stringliteral}{"WordWrapLeftAligned"} />}
\DoxyCodeLine{</TextComponent>}
\end{DoxyCode}
\hypertarget{fal_element_ref_fal_elem_ref_sec_17}{}\subsection{$<$\+Horz\+Format\+Property$>$ Element}\label{fal_element_ref_fal_elem_ref_sec_17}
\hypertarget{fal_element_ref_fal_elem_ref_sec_17_1}{}\subsubsection{Purpose\+:}\label{fal_element_ref_fal_elem_ref_sec_17_1}
The {\ttfamily $<$Horz\+Format\+Property$>$} element is intended to allow the system to access a property on the target window to obtain the horizontal formatting to be used when drawing the component being defined.\hypertarget{fal_element_ref_fal_elem_ref_sec_17_2}{}\subsubsection{Attributes\+:}\label{fal_element_ref_fal_elem_ref_sec_17_2}
\begin{DoxyItemize}
\item {\ttfamily name} specifies the name of the property to access. The named property must access a string value that will be set to one of the enumeration values appropriate for the component being defined (Horizontal\+Text\+Format for Text\+Component, and Horizontal\+Format for either Frame\+Component or Imagery\+Component). Required attribute. \item {\ttfamily component} Only for Frame\+Component. Specifies the part of the frame that this formatting is to be used for. Should be \char`\"{}\+Top\+Edge\char`\"{}, \char`\"{}\+Bottom\+Edge\char`\"{} or \char`\"{}\+Background\char`\"{} from the Frame\+Image\+Component enumeration. Optional attribute, defaults to \char`\"{}\+Background\char`\"{}.\end{DoxyItemize}
\hypertarget{fal_element_ref_fal_elem_ref_sec_17_3}{}\subsubsection{Usage\+:}\label{fal_element_ref_fal_elem_ref_sec_17_3}

\begin{DoxyItemize}
\item The {\ttfamily $<$Horz\+Format\+Property$>$} element may not contain sub-\/elements.


\item The {\ttfamily $<$Horz\+Format\+Property$>$} element may appear as a sub-\/element within any of the following elements\+: 
\begin{DoxyItemize}
\item {\ttfamily $<$Imagery\+Component$>$} to specify a horizontal formatting to be used the the image. 
\item {\ttfamily $<$Frame\+Component$>$} to specify a horizontal formatting to be used for the frame background. 
\item {\ttfamily $<$Text\+Component$>$} to specify a horizontal formatting to be used for the text. 
\end{DoxyItemize}
\end{DoxyItemize}\hypertarget{fal_element_ref_fal_elem_ref_sec_17_4}{}\subsubsection{Examples\+:}\label{fal_element_ref_fal_elem_ref_sec_17_4}
\hypertarget{fal_element_ref_fal_elem_ref_sec_18}{}\subsection{$<$\+Image$>$ Element}\label{fal_element_ref_fal_elem_ref_sec_18}
\hypertarget{fal_element_ref_fal_elem_ref_sec_18_1}{}\subsubsection{Purpose\+:}\label{fal_element_ref_fal_elem_ref_sec_18_1}
The {\ttfamily $<$Image$>$} element is used to specify an Imageset and Image pair, and for Frame\+Component images, how the image is to be used.\hypertarget{fal_element_ref_fal_elem_ref_sec_18_2}{}\subsubsection{Attributes\+:}\label{fal_element_ref_fal_elem_ref_sec_18_2}
\begin{DoxyItemize}
\item {\ttfamily imageset} specifies the name of an Imageset which contains the image to be used. Required attribute. \item {\ttfamily image} specifies the name of the image from the specified Imageset to be used. Required attribute. \item {\ttfamily component} Only for Frame\+Component. Specifies the part of the frame that this image is to be used for. One of the values from the Frame\+Image\+Component enumeration. Required attribute.\end{DoxyItemize}
\hypertarget{fal_element_ref_fal_elem_ref_sec_18_3}{}\subsubsection{Usage\+:}\label{fal_element_ref_fal_elem_ref_sec_18_3}

\begin{DoxyItemize}
\item The {\ttfamily $<$Image$>$} element may only appear as a sub-\/element of the {\ttfamily $<$Imagery\+Component$>$} or {\ttfamily $<$Frame\+Component$>$} elements. 
\item The {\ttfamily $<$Image$>$} element may not contain any sub-\/elements. 
\end{DoxyItemize}\hypertarget{fal_element_ref_fal_elem_ref_sec_18_4}{}\subsubsection{Examples\+:}\label{fal_element_ref_fal_elem_ref_sec_18_4}
Here you can see the {\ttfamily $<$Image$>$} element used to specify the image to render for an Imagery\+Component being defined\+: 
\begin{DoxyCode}{0}
\DoxyCodeLine{...}
\DoxyCodeLine{<ImageryComponent>}
\DoxyCodeLine{  <Area>}
\DoxyCodeLine{    <Dim type=\textcolor{stringliteral}{"LeftEdge"} ><AbsoluteDim value=\textcolor{stringliteral}{"0"} /></Dim>}
\DoxyCodeLine{    <Dim type=\textcolor{stringliteral}{"TopEdge"} ><AbsoluteDim value=\textcolor{stringliteral}{"0"} /></Dim>}
\DoxyCodeLine{    <Dim type=\textcolor{stringliteral}{"Width"} ><AbsoluteDim value=\textcolor{stringliteral}{"15"} /></Dim>}
\DoxyCodeLine{    <Dim type=\textcolor{stringliteral}{"Height"} ><UnifiedDim scale=\textcolor{stringliteral}{"1.0"} type=\textcolor{stringliteral}{"Height"} /></Dim>}
\DoxyCodeLine{  </Area>}
\DoxyCodeLine{  <Image imageset=\textcolor{stringliteral}{"FunkyLook"} image=\textcolor{stringliteral}{"ButtonIcon"} />}
\DoxyCodeLine{  <VertFormat type=\textcolor{stringliteral}{"CentreAligned"} />}
\DoxyCodeLine{  <HorzFormat type=\textcolor{stringliteral}{"CentreAligned"} />}
\DoxyCodeLine{</ImageryComponent>}
\DoxyCodeLine{...}
\end{DoxyCode}
\hypertarget{fal_element_ref_fal_elem_ref_sec_19}{}\subsection{$<$\+Image\+Dim$>$ Element}\label{fal_element_ref_fal_elem_ref_sec_19}
\hypertarget{fal_element_ref_fal_elem_ref_sec_19_1}{}\subsubsection{Purpose\+:}\label{fal_element_ref_fal_elem_ref_sec_19_1}
The {\ttfamily $<$Image\+Dim$>$} element is used to define a component dimension for an area rectangle. {\ttfamily $<$Image\+Dim$>$} is used to specify some dimension of an image for use as an area dimension.\hypertarget{fal_element_ref_fal_elem_ref_sec_19_2}{}\subsubsection{Attributes\+:}\label{fal_element_ref_fal_elem_ref_sec_19_2}
\begin{DoxyItemize}
\item {\ttfamily name} specifies the name of the image to be used. Required attribute. \item {\ttfamily dimension} specifies the image dimension to be used. This should be set to one of the values defined in the Dimension\+Type enumeration. Required attribute.\end{DoxyItemize}
\hypertarget{fal_element_ref_fal_elem_ref_sec_19_3}{}\subsubsection{Usage\+:}\label{fal_element_ref_fal_elem_ref_sec_19_3}

\begin{DoxyItemize}
\item The {\ttfamily $<$Image\+Dim$>$} element can appear as a sub-\/element in {\ttfamily $<$Dim$>$} to form a dimension specification for an area. 
\item The {\ttfamily $<$Image\+Dim$>$} element can appear as a sub-\/element of {\ttfamily $<$Operator\+Dim$>$} to specify one of the operands for a dimension calculation. 
\end{DoxyItemize}\hypertarget{fal_element_ref_fal_elem_ref_sec_19_4}{}\subsubsection{Examples\+:}\label{fal_element_ref_fal_elem_ref_sec_19_4}
This example shows a dimension that uses {\ttfamily $<$Image\+Dim$>$} to fetch the width of a specified image for use as the dimensions value\+: 
\begin{DoxyCode}{0}
\DoxyCodeLine{...}
\DoxyCodeLine{<Area>}
\DoxyCodeLine{  <Dim type=\textcolor{stringliteral}{"LeftEdge"}>}
\DoxyCodeLine{    <ImageDim name=\textcolor{stringliteral}{"myImages/leftImage"} dimension=\textcolor{stringliteral}{"Width"} />}
\DoxyCodeLine{  </Dim>}
\DoxyCodeLine{  ...}
\DoxyCodeLine{</Area>}
\DoxyCodeLine{...}
\end{DoxyCode}
\hypertarget{fal_element_ref_fal_elem_ref_sec_imagepropertydim}{}\subsection{$<$\+Image\+Property\+Dim$>$ Element}\label{fal_element_ref_fal_elem_ref_sec_imagepropertydim}
\hypertarget{fal_element_ref_fal_elem_ref_sec_imagepropertydim_1}{}\subsubsection{Purpose\+:}\label{fal_element_ref_fal_elem_ref_sec_imagepropertydim_1}
The {\ttfamily $<$Image\+Property\+Dim$>$} element is used to define a component dimension for an area rectangle. {\ttfamily $<$Image\+Property\+Dim$>$} is used to specify some dimension of an image that is named by a property for use as an area dimension.\hypertarget{fal_element_ref_fal_elem_ref_sec_imagepropertydim_2}{}\subsubsection{Attributes\+:}\label{fal_element_ref_fal_elem_ref_sec_imagepropertydim_2}
\begin{DoxyItemize}
\item {\ttfamily name} specifies the name of the property that will fetch the name of the Image to be used. Required attribute. \item {\ttfamily dimension} specifies the image dimension to be used. This should be set to one of the values defined in the Dimension\+Type enumeration. Required attribute.\end{DoxyItemize}
\hypertarget{fal_element_ref_fal_elem_ref_sec_imagepropertydim_3}{}\subsubsection{Usage\+:}\label{fal_element_ref_fal_elem_ref_sec_imagepropertydim_3}

\begin{DoxyItemize}
\item The {\ttfamily $<$Image\+Property\+Dim$>$} element can appear as a sub-\/element in {\ttfamily $<$Dim$>$} to form a dimension specification for an area. 
\item The {\ttfamily $<$Image\+Property\+Dim$>$} element can appear as a sub-\/element of {\ttfamily $<$Operator\+Dim$>$} to specify one of the operands for a dimension calculation. 
\end{DoxyItemize}\hypertarget{fal_element_ref_fal_elem_ref_sec_imagepropertydim_4}{}\subsubsection{Examples\+:}\label{fal_element_ref_fal_elem_ref_sec_imagepropertydim_4}
This example shows a dimension that uses {\ttfamily $<$Image\+Property\+Dim$>$} to fetch the Height of the image specified in the property {\ttfamily Frame\+Top\+Image} for use as the dimension\textquotesingle{}s value\+: 
\begin{DoxyCode}{0}
\DoxyCodeLine{...}
\DoxyCodeLine{<Area>}
\DoxyCodeLine{  <Dim type=\textcolor{stringliteral}{"TopEdge"}>}
\DoxyCodeLine{    <ImagePropertyDim name=\textcolor{stringliteral}{"FrameTopImage"} dimension=\textcolor{stringliteral}{"Height"} />}
\DoxyCodeLine{  </Dim>}
\DoxyCodeLine{  ...}
\DoxyCodeLine{</Area>}
\DoxyCodeLine{...}
\end{DoxyCode}
\hypertarget{fal_element_ref_fal_elem_ref_sec_20}{}\subsection{$<$\+Imagery\+Component$>$ Element}\label{fal_element_ref_fal_elem_ref_sec_20}
\hypertarget{fal_element_ref_fal_elem_ref_sec_20_1}{}\subsubsection{Purpose\+:}\label{fal_element_ref_fal_elem_ref_sec_20_1}
The {\ttfamily $<$Imagery\+Component$>$} element defines a single image to be drawn within a given Imagery\+Section. The Imagery\+Component contains all information about which image is to be drawn, where it should be drawn, which colours are to be used and how the image should be formatted.\hypertarget{fal_element_ref_fal_elem_ref_sec_20_2}{}\subsubsection{Attributes\+:}\label{fal_element_ref_fal_elem_ref_sec_20_2}
No attributes are defined for the {\ttfamily $<$Imagery\+Component$>$} element.\hypertarget{fal_element_ref_fal_elem_ref_sec_20_3}{}\subsubsection{Usage\+:}\label{fal_element_ref_fal_elem_ref_sec_20_3}
Note\+: the sub-\/elements should appear in the order that they are defined here.


\begin{DoxyItemize}
\item {\ttfamily $<$Area$>$} defining the target area for this image.


\item Either one of\+: 
\begin{DoxyItemize}
\item {\ttfamily $<$Image$>$} element specifying the image to be drawn. 
\item {\ttfamily $<$Image\+Property$>$} element specifying a property defining the image to be drawn. 
\end{DoxyItemize}


\item Optionally specifying the colours for this single image, one of the colour elements\+: 
\begin{DoxyItemize}
\item {\ttfamily $<$Colours$>$} 
\item {\ttfamily $<$Colour\+Property$>$} 
\item {\ttfamily $<$Colour\+Rect\+Property$>$} 
\end{DoxyItemize}


\item Optionally, to specify the vertical formatting to use, either of\+: 
\begin{DoxyItemize}
\item {\ttfamily $<$Vert\+Format$>$} 
\item {\ttfamily $<$Vert\+Format\+Property$>$} 
\end{DoxyItemize}


\item Optionally, to specify the horizontal formatting to use, either of\+: 
\begin{DoxyItemize}
\item {\ttfamily $<$Horz\+Format$>$} 
\item {\ttfamily $<$Horz\+Format\+Property$>$} 
\end{DoxyItemize}


\item The {\ttfamily $<$Imagery\+Component$>$} element may only appear as a sub-\/element of the element {\ttfamily $<$Imagery\+Section$>$}. 
\end{DoxyItemize}\hypertarget{fal_element_ref_fal_elem_ref_sec_20_4}{}\subsubsection{Examples\+:}\label{fal_element_ref_fal_elem_ref_sec_20_4}
The following was taken from Taharez\+Look.\+looknfeel and shows a full Imagery\+Component definition\+: 
\begin{DoxyCode}{0}
\DoxyCodeLine{<ImageryComponent>}
\DoxyCodeLine{  <Area>}
\DoxyCodeLine{    <Dim type=\textcolor{stringliteral}{"LeftEdge"} ><UnifiedDim scale=\textcolor{stringliteral}{"0"} type=\textcolor{stringliteral}{"LeftEdge"} /></Dim>}
\DoxyCodeLine{    <Dim type=\textcolor{stringliteral}{"TopEdge"} ><UnifiedDim scale=\textcolor{stringliteral}{"0.2"} type=\textcolor{stringliteral}{"TopEdge"} /></Dim>}
\DoxyCodeLine{    <Dim type=\textcolor{stringliteral}{"Width"} ><UnifiedDim scale=\textcolor{stringliteral}{"1"} type=\textcolor{stringliteral}{"Width"} /></Dim>}
\DoxyCodeLine{    <Dim type=\textcolor{stringliteral}{"Height"} ><UnifiedDim scale=\textcolor{stringliteral}{"0.3"} type=\textcolor{stringliteral}{"Height"} /></Dim>}
\DoxyCodeLine{  </Area>}
\DoxyCodeLine{  <Image imageset=\textcolor{stringliteral}{"TaharezLook"} image=\textcolor{stringliteral}{"TextSelectionBrush"} />}
\DoxyCodeLine{  <Colours}
\DoxyCodeLine{    topLeft=\textcolor{stringliteral}{"FFFFFF00"}}
\DoxyCodeLine{    topRight=\textcolor{stringliteral}{"FFFFFF00"}}
\DoxyCodeLine{    bottomLeft=\textcolor{stringliteral}{"FFFFFF00"}}
\DoxyCodeLine{    bottomRight=\textcolor{stringliteral}{"FFFFFF00"}}
\DoxyCodeLine{  />}
\DoxyCodeLine{  <VertFormat type=\textcolor{stringliteral}{"Tiled"} />}
\DoxyCodeLine{  <HorzFormat type=\textcolor{stringliteral}{"Stretched"} />}
\DoxyCodeLine{</ImageryComponent>}
\end{DoxyCode}
\hypertarget{fal_element_ref_fal_elem_ref_sec_21}{}\subsection{$<$\+Image\+Property$>$ Element}\label{fal_element_ref_fal_elem_ref_sec_21}
\hypertarget{fal_element_ref_fal_elem_ref_sec_21_1}{}\subsubsection{Purpose\+:}\label{fal_element_ref_fal_elem_ref_sec_21_1}
The {\ttfamily $<$Image\+Property$>$} element is intended to allow the system to access a property on the target window to obtain the final image to be used when rendering the component being defined.\hypertarget{fal_element_ref_fal_elem_ref_sec_21_2}{}\subsubsection{Attributes\+:}\label{fal_element_ref_fal_elem_ref_sec_21_2}
\begin{DoxyItemize}
\item {\ttfamily name} specifies the name of the property to access that will provide the name of the Image to be used. Required attribute. \item {\ttfamily component} Only for {\ttfamily $<$Frame\+Component$>$}. Specifies the part of the frame that this image is to be used for. One of the values from the Frame\+Image\+Component enumeration. Required attribute.\end{DoxyItemize}
\hypertarget{fal_element_ref_fal_elem_ref_sec_21_3}{}\subsubsection{Usage\+:}\label{fal_element_ref_fal_elem_ref_sec_21_3}

\begin{DoxyItemize}
\item The {\ttfamily $<$Image\+Property$>$} element may not contain sub-\/elements. 
\item The {\ttfamily $<$Image\+Property$>$} element may appear as a sub-\/element within either {\ttfamily $<$Imagery\+Component$>$} or {\ttfamily $<$Frame\+Component$>$} elements. 
\end{DoxyItemize}\hypertarget{fal_element_ref_fal_elem_ref_sec_21_4}{}\subsubsection{Examples\+:}\label{fal_element_ref_fal_elem_ref_sec_21_4}
\hypertarget{fal_element_ref_fal_elem_ref_sec_22}{}\subsection{$<$\+Imagery\+Section$>$ Element}\label{fal_element_ref_fal_elem_ref_sec_22}
\hypertarget{fal_element_ref_fal_elem_ref_sec_22_1}{}\subsubsection{Purpose\+:}\label{fal_element_ref_fal_elem_ref_sec_22_1}
The {\ttfamily $<$Imagery\+Section$>$} element is used to group multiple {\ttfamily $<$Imagery\+Component$>$} and {\ttfamily $<$Text\+Component$>$} definitions into named sections which can then be specified for use as imagery in state definitions.\hypertarget{fal_element_ref_fal_elem_ref_sec_22_2}{}\subsubsection{Attributes\+:}\label{fal_element_ref_fal_elem_ref_sec_22_2}
\begin{DoxyItemize}
\item {\ttfamily name} specifies the name to be given to this Imagery\+Section. Names are per-\/\+Widget\+Look, and specifying the same name more than once will replace the previous definition. Required attribute.\end{DoxyItemize}
\hypertarget{fal_element_ref_fal_elem_ref_sec_22_3}{}\subsubsection{Usage\+:}\label{fal_element_ref_fal_elem_ref_sec_22_3}
Note\+: the sub-\/elements should appear in the order that they are defined here.


\begin{DoxyItemize}
\item To optionally specify colours to be modulated with the individual component\textquotesingle{}s colours, the {\ttfamily $<$Imagery\+Section$>$} may contain one of the colour definition elements\+: 
\begin{DoxyItemize}
\item {\ttfamily $<$Colours$>$} 
\item {\ttfamily $<$Colour\+Property$>$} 
\item {\ttfamily $<$Colour\+Rect\+Property$>$} 
\end{DoxyItemize}


\item Any number of {\ttfamily $<$Frame\+Component$>$} elements may then follow. 
\item Followed by any number of {\ttfamily $<$Imagery\+Component$>$} elements. 
\item Finally, any number of {\ttfamily $<$Text\+Component$>$} elements may be given. 
\item The {\ttfamily $<$Imagery\+Section$>$} element may only appear as a sub-\/element of the {\ttfamily $<$Widget\+Look$>$} element. 
\end{DoxyItemize}\hypertarget{fal_element_ref_fal_elem_ref_sec_22_4}{}\subsubsection{Examples\+:}\label{fal_element_ref_fal_elem_ref_sec_22_4}

\begin{DoxyCode}{0}
\DoxyCodeLine{<ImagerySection name=\textcolor{stringliteral}{"example"}>}
\DoxyCodeLine{  <ImageryComponent>}
\DoxyCodeLine{    <Area>}
\DoxyCodeLine{      <Dim type=\textcolor{stringliteral}{"LeftEdge"} ><AbsoluteDim value=\textcolor{stringliteral}{"0"} /></Dim>}
\DoxyCodeLine{      <Dim type=\textcolor{stringliteral}{"TopEdge"} ><AbsoluteDim value=\textcolor{stringliteral}{"0"} /></Dim>}
\DoxyCodeLine{      <Dim type=\textcolor{stringliteral}{"Width"} ><AbsoluteDim value=\textcolor{stringliteral}{"15"} /></Dim>}
\DoxyCodeLine{      <Dim type=\textcolor{stringliteral}{"Height"} ><UnifiedDim scale=\textcolor{stringliteral}{"1.0"} type=\textcolor{stringliteral}{"Height"} /></Dim>}
\DoxyCodeLine{    </Area>}
\DoxyCodeLine{    <Image imageset=\textcolor{stringliteral}{"sillyImages"} image=\textcolor{stringliteral}{"anotherImage"} />}
\DoxyCodeLine{    <VertFormat type=\textcolor{stringliteral}{"Stretched"} />}
\DoxyCodeLine{    <HorzFormat type=\textcolor{stringliteral}{"Stretched"} />}
\DoxyCodeLine{  </ImageryComponent>}
\DoxyCodeLine{  <TextComponent>}
\DoxyCodeLine{    <Area>}
\DoxyCodeLine{      <Dim type=\textcolor{stringliteral}{"LeftEdge"} ><AbsoluteDim value=\textcolor{stringliteral}{"0"} /></Dim>}
\DoxyCodeLine{      <Dim type=\textcolor{stringliteral}{"TopEdge"} ><AbsoluteDim value=\textcolor{stringliteral}{"0"} /></Dim>}
\DoxyCodeLine{      <Dim type=\textcolor{stringliteral}{"Width"} ><UnifiedDim scale=\textcolor{stringliteral}{"1"} type=\textcolor{stringliteral}{"Width"} /></Dim>}
\DoxyCodeLine{      <Dim type=\textcolor{stringliteral}{"Height"} ><UnifiedDim scale=\textcolor{stringliteral}{"1"} type=\textcolor{stringliteral}{"Height"} /></Dim>}
\DoxyCodeLine{    </Area>}
\DoxyCodeLine{  </TextComponent>}
\DoxyCodeLine{</ImagerySection>}
\end{DoxyCode}
\hypertarget{fal_element_ref_fal_elem_ref_sec_23}{}\subsection{$<$\+Layer$>$ Element}\label{fal_element_ref_fal_elem_ref_sec_23}
\hypertarget{fal_element_ref_fal_elem_ref_sec_23_1}{}\subsubsection{Purpose\+:}\label{fal_element_ref_fal_elem_ref_sec_23_1}
The {\ttfamily $<$Layer$>$} element is used to define layers of imagery within the definition of a State\+Imagery section.\hypertarget{fal_element_ref_fal_elem_ref_sec_23_2}{}\subsubsection{Attributes\+:}\label{fal_element_ref_fal_elem_ref_sec_23_2}
\begin{DoxyItemize}
\item {\ttfamily priority} specifies the priority for the layer. Higher priorities appear in front of lower priorities. Default priority is 0. Optional attribute.\end{DoxyItemize}
\hypertarget{fal_element_ref_fal_elem_ref_sec_23_3}{}\subsubsection{Usage\+:}\label{fal_element_ref_fal_elem_ref_sec_23_3}

\begin{DoxyItemize}
\item The {\ttfamily $<$Layer$>$} element may only appear as a sub-\/element of the {\ttfamily $<$State\+Imagery$>$} element. 
\item The {\ttfamily $<$Layer$>$} element may contain any number of {\ttfamily $<$Section$>$} sub-\/elements. 
\end{DoxyItemize}\hypertarget{fal_element_ref_fal_elem_ref_sec_23_4}{}\subsubsection{Examples\+:}\label{fal_element_ref_fal_elem_ref_sec_23_4}
Here we see a single layer with multiple sections included. This example was taken from the Taharez\+Look skin X\+ML file (List\+Header\+Segment widget)\+: 
\begin{DoxyCode}{0}
\DoxyCodeLine{<StateImagery name=\textcolor{stringliteral}{"Normal"}>}
\DoxyCodeLine{  <Layer>}
\DoxyCodeLine{    <Section section=\textcolor{stringliteral}{"segment\_normal"} />}
\DoxyCodeLine{    <Section section=\textcolor{stringliteral}{"splitter\_normal"} />}
\DoxyCodeLine{    <Section section=\textcolor{stringliteral}{"label"} />}
\DoxyCodeLine{  </Layer>}
\DoxyCodeLine{</StateImagery>}
\end{DoxyCode}
\hypertarget{fal_element_ref_fal_elem_ref_sec_24}{}\subsection{$<$\+Named\+Area$>$ Element}\label{fal_element_ref_fal_elem_ref_sec_24}
\hypertarget{fal_element_ref_fal_elem_ref_sec_24_1}{}\subsubsection{Purpose\+:}\label{fal_element_ref_fal_elem_ref_sec_24_1}
Defines an area that can be accessed via it\textquotesingle{}s name. Generally this this used by base widgets to obtain skin supplied areas for use in rendering or other widget specific operations.\hypertarget{fal_element_ref_fal_elem_ref_sec_24_2}{}\subsubsection{Attributes\+:}\label{fal_element_ref_fal_elem_ref_sec_24_2}
\begin{DoxyItemize}
\item {\ttfamily name} specifies a name for the area being defined. Required attribute.\end{DoxyItemize}
\hypertarget{fal_element_ref_fal_elem_ref_sec_24_3}{}\subsubsection{Usage\+:}\label{fal_element_ref_fal_elem_ref_sec_24_3}

\begin{DoxyItemize}
\item The {\ttfamily $<$Named\+Area$>$} element must contain only an {\ttfamily $<$Area$>$} sub-\/element defining the area rectangle for the named area. 
\item The {\ttfamily $<$Named\+Area$>$} element may only appear as a sub-\/element within {\ttfamily $<$Widget\+Look$>$} elements. 
\end{DoxyItemize}\hypertarget{fal_element_ref_fal_elem_ref_sec_24_4}{}\subsubsection{Examples\+:}\label{fal_element_ref_fal_elem_ref_sec_24_4}
This example defines a named area called \textquotesingle{}Text\+Area\textquotesingle{}. It is defined as being an area seven pixels inside the total area of the widget being defined\+: 
\begin{DoxyCode}{0}
\DoxyCodeLine{<NamedArea name=\textcolor{stringliteral}{"TextArea"}>}
\DoxyCodeLine{  <Area>}
\DoxyCodeLine{    <Dim type=\textcolor{stringliteral}{"LeftEdge"} >}
\DoxyCodeLine{      <AbsoluteDim value=\textcolor{stringliteral}{"7"} />}
\DoxyCodeLine{    </Dim>}
\DoxyCodeLine{    <Dim type=\textcolor{stringliteral}{"TopEdge"} >}
\DoxyCodeLine{      <AbsoluteDim value=\textcolor{stringliteral}{"7"} />}
\DoxyCodeLine{    </Dim>}
\DoxyCodeLine{    <Dim type=\textcolor{stringliteral}{"RightEdge"} >}
\DoxyCodeLine{      <UnifiedDim scale=\textcolor{stringliteral}{"1.0"} offset=\textcolor{stringliteral}{"-7"} type=\textcolor{stringliteral}{"RightEdge"} />}
\DoxyCodeLine{    </Dim>}
\DoxyCodeLine{    <Dim type=\textcolor{stringliteral}{"BottomEdge"} >}
\DoxyCodeLine{      <UnifiedDim scale=\textcolor{stringliteral}{"1.0"} offset=\textcolor{stringliteral}{"-7"} type=\textcolor{stringliteral}{"BottomEdge"} />}
\DoxyCodeLine{    </Dim>}
\DoxyCodeLine{  </Area>}
\DoxyCodeLine{</NamedArea>}
\end{DoxyCode}
\hypertarget{fal_element_ref_fal_elem_ref_sec_operatordim}{}\subsection{$<$\+Operator\+Dim$>$ Element}\label{fal_element_ref_fal_elem_ref_sec_operatordim}
\hypertarget{fal_element_ref_fal_elem_ref_sec_operatordim_1}{}\subsubsection{Purpose\+:}\label{fal_element_ref_fal_elem_ref_sec_operatordim_1}
The {\ttfamily $<$Operator\+Dim$>$} element allows you to use the result of performing a mathematical calculation on two two dimension elements. Since either dimension used as operands may also be an {\ttfamily $<$Operator\+Dim$>$} it is possible to create expression trees to perform any mathematical calculation that can be expressed using the supported operators.\hypertarget{fal_element_ref_fal_elem_ref_sec_operatordim_2}{}\subsubsection{Attributes\+:}\label{fal_element_ref_fal_elem_ref_sec_operatordim_2}
\begin{DoxyItemize}
\item {\ttfamily op} specifies one of the vales from the Dimension\+Operator enumeration indicating the mathematical operation to be performed. Required attribute.\end{DoxyItemize}
\hypertarget{fal_element_ref_fal_elem_ref_sec_operatordim_3}{}\subsubsection{Usage\+:}\label{fal_element_ref_fal_elem_ref_sec_operatordim_3}

\begin{DoxyItemize}
\item A single {\ttfamily $<$Operator\+Dim$>$} element may appear as a sub-\/element within the following\+: 
\begin{DoxyItemize}
\item {\ttfamily $<$Dim$>$} 
\item {\ttfamily $<$Operator\+Dim$>$} 
\end{DoxyItemize}


\item The {\ttfamily $<$Operator\+Dim$>$} element must contain any of two the following dimension elements\+: 
\begin{DoxyItemize}
\item {\ttfamily $<$Absolute\+Dim$>$} 
\item {\ttfamily $<$Font\+Dim$>$} 
\item {\ttfamily $<$Image\+Dim$>$} 
\item {\ttfamily $<$Image\+Property\+Dim$>$} 
\item {\ttfamily $<$Property\+Dim$>$} 
\item {\ttfamily $<$Unified\+Dim$>$} 
\item {\ttfamily $<$Widget\+Dim$>$} 
\item {\ttfamily $<$Operator\+Dim$>$} 
\end{DoxyItemize}
\end{DoxyItemize}\hypertarget{fal_element_ref_fal_elem_ref_sec_operatordim_4}{}\subsubsection{Examples\+:}\label{fal_element_ref_fal_elem_ref_sec_operatordim_4}
The following multiplies two simple Absolute\+Dim dimensions\+: 
\begin{DoxyCode}{0}
\DoxyCodeLine{...}
\DoxyCodeLine{<OperatorDim op=\textcolor{stringliteral}{"Multiply"}>}
\DoxyCodeLine{  <AbsoluteDim value=\textcolor{stringliteral}{"10"} />}
\DoxyCodeLine{  <AbsoluteDim value=\textcolor{stringliteral}{"4"} />}
\DoxyCodeLine{</OperatorDim>}
\DoxyCodeLine{...}
\end{DoxyCode}


The next example takes the height of the font used for the target window, adds four pixels and multiplies the result by two.

The operation performed will be\+:

$ (LineSpacing + 4) {*} 2 $


\begin{DoxyCode}{0}
\DoxyCodeLine{...}
\DoxyCodeLine{<OperatorDim op=\textcolor{stringliteral}{"Multiply"}>}
\DoxyCodeLine{  <OperatorDim op=\textcolor{stringliteral}{"Add"}>}
\DoxyCodeLine{    <FontDim type=\textcolor{stringliteral}{"LineSpacing"} />}
\DoxyCodeLine{    <AbsoluteDim value=\textcolor{stringliteral}{"4"} />}
\DoxyCodeLine{  </OperatorDim>}
\DoxyCodeLine{  <AbsoluteDim value=\textcolor{stringliteral}{"2"} />}
\DoxyCodeLine{</OperatorDim>}
\DoxyCodeLine{...}
\end{DoxyCode}
\hypertarget{fal_element_ref_fal_elem_ref_sec_25}{}\subsection{$<$\+Property$>$ Element}\label{fal_element_ref_fal_elem_ref_sec_25}
\hypertarget{fal_element_ref_fal_elem_ref_sec_25_1}{}\subsubsection{Purpose\+:}\label{fal_element_ref_fal_elem_ref_sec_25_1}
The {\ttfamily $<$Property$>$} element is used to initialise a property on a window or widget being defined.\hypertarget{fal_element_ref_fal_elem_ref_sec_25_2}{}\subsubsection{Attributes\+:}\label{fal_element_ref_fal_elem_ref_sec_25_2}
\begin{DoxyItemize}
\item {\ttfamily name} specifies the name of the property to be initialised. Required attribute. \item {\ttfamily value} specifies the value string to be used when initialising the property. Required attribute.\end{DoxyItemize}
\hypertarget{fal_element_ref_fal_elem_ref_sec_25_3}{}\subsubsection{Usage\+:}\label{fal_element_ref_fal_elem_ref_sec_25_3}

\begin{DoxyItemize}
\item The {\ttfamily $<$Property$>$} element may not contain any sub-\/elements. 
\item The {\ttfamily $<$Property$>$} element may appear as a sub-\/element in {\ttfamily $<$Widget\+Look$>$} elements to define property initialisers for the type being defined. 
\item The {\ttfamily $<$Property$>$} element may appear as a sub-\/element in {\ttfamily $<$Child$>$} elements to define property initialisers for the child widget being defined. 
\end{DoxyItemize}\hypertarget{fal_element_ref_fal_elem_ref_sec_25_4}{}\subsubsection{Examples\+:}\label{fal_element_ref_fal_elem_ref_sec_25_4}
In this extract from the definition for Taharez\+Look/\+Titlebar, we can see the {\ttfamily $<$Property$>$} element used to set the \textquotesingle{}Caption\+Colour\textquotesingle{} property; this establishes a default for all instances of this widget\+: 
\begin{DoxyCode}{0}
\DoxyCodeLine{<WidgetLook name=\textcolor{stringliteral}{"TaharezLook/Titlebar"}>}
\DoxyCodeLine{  <Property name=\textcolor{stringliteral}{"CaptionColour"} value=\textcolor{stringliteral}{"FFFFFFFF"} />}
\DoxyCodeLine{  <ImagerySection name=\textcolor{stringliteral}{"main"}>}
\DoxyCodeLine{    <ImageryComponent>}
\DoxyCodeLine{      <Area>}
\DoxyCodeLine{        <Dim type=\textcolor{stringliteral}{"LeftEdge"} ><AbsoluteDim value=\textcolor{stringliteral}{"0"} /></Dim>}
\DoxyCodeLine{        ...}
\DoxyCodeLine{      </Area>}
\DoxyCodeLine{      ...}
\DoxyCodeLine{    </ImageryComponent>}
\DoxyCodeLine{  </ImagerySection>}
\DoxyCodeLine{  ...}
\DoxyCodeLine{</WidgetLook>}
\end{DoxyCode}
\hypertarget{fal_element_ref_fal_elem_ref_sec_26}{}\subsection{$<$\+Property\+Definition$>$ Element}\label{fal_element_ref_fal_elem_ref_sec_26}
\hypertarget{fal_element_ref_fal_elem_ref_sec_26_1}{}\subsubsection{Purpose\+:}\label{fal_element_ref_fal_elem_ref_sec_26_1}
The {\ttfamily $<$Property\+Definition$>$} element creates a new named property for the widget being defined. The defined property may be accessed via any means that a \textquotesingle{}normal\textquotesingle{} property may.\hypertarget{fal_element_ref_fal_elem_ref_sec_26_2}{}\subsubsection{Attributes\+:}\label{fal_element_ref_fal_elem_ref_sec_26_2}
\begin{DoxyItemize}
\item {\ttfamily name} specifies the name to use for the new property. Required attribute. \item {\ttfamily initial\+Value} specifies the initial value to be assigned to the property. Optional attribute. \item {\ttfamily type} specifies the data type of the property. This should be one of the values defined for the Property\+Type enumeration. Defaults to {\ttfamily Generic}. Optional attribute. \item {\ttfamily redraw\+On\+Write} boolean setting specifies whether writing a new value to this property should cause the widget being defined to redraw itself. Optional attribute. \item {\ttfamily layout\+On\+Write} boolean setting specifies whether writing a new value to this property should cause the widget being defined to re-\/layout it\textquotesingle{}s defined child widgets. Optional attribute. \item {\ttfamily fire\+Event} specifies the name of an event to fire when this property is written to. The event is passed a C\+E\+G\+U\+I\+::\+Window\+Event\+Args reference with the window field initialised to the window upon which the property was written. For the C\+E\+G\+U\+I\+::\+Global\+Event\+Set, the event uses the name of the Widget\+Look being defined as the event namespace. Optional attribute.\end{DoxyItemize}
\hypertarget{fal_element_ref_fal_elem_ref_sec_26_3}{}\subsubsection{Usage\+:}\label{fal_element_ref_fal_elem_ref_sec_26_3}

\begin{DoxyItemize}
\item The {\ttfamily $<$Property\+Definition$>$} element may not contain sub-\/elements. 
\item The {\ttfamily $<$Property\+Definition$>$} element must appear as a sub-\/element within {\ttfamily $<$Widget\+Look$>$} elements. 
\end{DoxyItemize}\hypertarget{fal_element_ref_fal_elem_ref_sec_26_4}{}\subsubsection{Examples\+:}\label{fal_element_ref_fal_elem_ref_sec_26_4}
In this example, within the Widget\+Look we create a new property named \textquotesingle{}Scrollbar\+Width\textquotesingle{}. We then use this property to control the width of a component child widget. This effectively gives the user control over the width of the child scrollbar via the property\+: 
\begin{DoxyCode}{0}
\DoxyCodeLine{<WidgetLook name=\textcolor{stringliteral}{"PropertyDefExample"}>}
\DoxyCodeLine{  <PropertyDefinition}
\DoxyCodeLine{    name=\textcolor{stringliteral}{"ScrollbarWidth"}}
\DoxyCodeLine{    initialValue=\textcolor{stringliteral}{"12"}}
\DoxyCodeLine{    layoutOnWrite=\textcolor{stringliteral}{"true"}}
\DoxyCodeLine{  />}
\DoxyCodeLine{  ...}
\DoxyCodeLine{  <Child type=\textcolor{stringliteral}{"MyVertScrollbar"} nameSuffix=\textcolor{stringliteral}{"\_\_auto\_vscrollbar\_\_"}>}
\DoxyCodeLine{    <Area>}
\DoxyCodeLine{      <Dim type=\textcolor{stringliteral}{"LeftEdge"} ><AbsoluteDim value=\textcolor{stringliteral}{"0"} /></Dim>}
\DoxyCodeLine{      <Dim type=\textcolor{stringliteral}{"TopEdge"} ><AbsoluteDim value=\textcolor{stringliteral}{"0"} /></Dim>}
\DoxyCodeLine{      <Dim type=\textcolor{stringliteral}{"Width"} ><PropertyDim name=\textcolor{stringliteral}{"ScrollbarWidth"} /></Dim>}
\DoxyCodeLine{      <Dim type=\textcolor{stringliteral}{"Height"} ><UnifiedDim scale=\textcolor{stringliteral}{"1"} type=\textcolor{stringliteral}{"Height"} /></Dim>}
\DoxyCodeLine{    </Area>}
\DoxyCodeLine{    <HorzAlignment type=\textcolor{stringliteral}{"RightAligned"} />}
\DoxyCodeLine{  </Child>}
\DoxyCodeLine{  ...}
\DoxyCodeLine{</WidgetLook>}
\end{DoxyCode}
\hypertarget{fal_element_ref_fal_elem_ref_sec_27}{}\subsection{$<$\+Property\+Link\+Definition$>$ Element}\label{fal_element_ref_fal_elem_ref_sec_27}
\hypertarget{fal_element_ref_fal_elem_ref_sec_27_1}{}\subsubsection{Purpose\+:}\label{fal_element_ref_fal_elem_ref_sec_27_1}
The {\ttfamily $<$Property\+Link\+Definition$>$} element creates a new named property for the widget being defined that is linked to one or more properties on either child widget components, or on the parent widget. The target widgets and properties can be specified either as attributes (for example if the link is to be a one to one mapping), or via $<$Property\+Link\+Target$>$ elements (if there is to be a one to many mapping). This allows properties on child widgets to be directly exposed to clients of the widget being defined, as well as allowing the widget being defined to make use of properties defined on a parent widget. The defined property may be accessed via any means that a \textquotesingle{}normal\textquotesingle{} property may.\hypertarget{fal_element_ref_fal_elem_ref_sec_27_2}{}\subsubsection{Attributes\+:}\label{fal_element_ref_fal_elem_ref_sec_27_2}
\begin{DoxyItemize}
\item {\ttfamily name} specifies the name to use for the new property. Required attribute. \item {\ttfamily widget} specifies either the name suffix of the child widget containing the first property to be linked to the property being defined, or the special value {\ttfamily \char`\"{}\+\_\+\+\_\+parent\+\_\+\+\_\+\char`\"{}} to indicate a back-\/link to the parent. Optional attribute. \item {\ttfamily target\+Property} specifies the name of the property on the child widget that is to be the first property linked to the new property being defined. If this is omitted but {\ttfamily widget} is specified, it will be assumed that the target property has the same name as the property being defined. Optional attribute. \item {\ttfamily initial\+Value} specifies the initial value to be assigned to the property. Optional attribute. \item {\ttfamily type} specifies the data type of the property. This should be one of the values defined for the Property\+Type enumeration. Defaults to {\ttfamily Generic}. Optional attribute. \item {\ttfamily redraw\+On\+Write} boolean setting specifies whether writing a new value to this property should cause the widget being defined to redraw itself. Optional attribute. \item {\ttfamily layout\+On\+Write} boolean setting specifies whether writing a new value to this property should cause the widget being defined to re-\/layout it\textquotesingle{}s defined child widgets. Optional attribute. \item {\ttfamily fire\+Event} specifies the name of an event to fire when this property is written to. The event is passed a C\+E\+G\+U\+I\+::\+Window\+Event\+Args reference with the window field initialised to the window upon which the property was written. For the C\+E\+G\+U\+I\+::\+Global\+Event\+Set, the event uses the name of the Widget\+Look being defined as the event namespace. It is the responsibility of the Widget\+Look author to ensure that collisions with predefined events are avoided. Optional attribute.\end{DoxyItemize}
\hypertarget{fal_element_ref_fal_elem_ref_sec_27_3}{}\subsubsection{Usage\+:}\label{fal_element_ref_fal_elem_ref_sec_27_3}

\begin{DoxyItemize}
\item The {\ttfamily $<$Property\+Link\+Definition$>$} element may contain any number of {\ttfamily $<$Property\+Link\+Target$>$} sub-\/elements. 
\item The {\ttfamily $<$Property\+Link\+Definition$>$} element may only appear as a sub-\/element within {\ttfamily $<$Widget\+Look$>$} elements. 
\end{DoxyItemize}\hypertarget{fal_element_ref_fal_elem_ref_sec_27_4}{}\subsubsection{Examples\+:}\label{fal_element_ref_fal_elem_ref_sec_27_4}
In this example we create a new property named \textquotesingle{}Caption\+Text\+Colour\textquotesingle{}. This is linked to a property named \textquotesingle{}Caption\+Colour\textquotesingle{} on the child widget with name suffix \textquotesingle{}{\bfseries{auto\+\_\+titlebar}}\textquotesingle{}. Any access of the \textquotesingle{}Caption\+Text\+Colour\textquotesingle{} property on the widget will actually access the \textquotesingle{}Caption\+Colour\textquotesingle{} property on the specified child widget\+: 
\begin{DoxyCode}{0}
\DoxyCodeLine{<WidgetLook name=\textcolor{stringliteral}{"PropertyLinkExample"}>}
\DoxyCodeLine{  <PropertyLinkDefinition}
\DoxyCodeLine{    name=\textcolor{stringliteral}{"CaptionTextColour"}}
\DoxyCodeLine{    widget=\textcolor{stringliteral}{"\_\_auto\_titlebar\_\_"}}
\DoxyCodeLine{    targetProperty=\textcolor{stringliteral}{"CaptionColour"}}
\DoxyCodeLine{    initialValue=\textcolor{stringliteral}{"FFFF3333"}}
\DoxyCodeLine{    layoutOnWrite=\textcolor{stringliteral}{"true"}}
\DoxyCodeLine{  />}
\DoxyCodeLine{  ...}
\DoxyCodeLine{</WidgetLook>}
\end{DoxyCode}
\hypertarget{fal_element_ref_fal_elem_propertylinktarget}{}\subsection{$<$\+Property\+Link\+Target$>$ Element}\label{fal_element_ref_fal_elem_propertylinktarget}
\hypertarget{fal_element_ref_fal_elem_propertylinktargetr_1}{}\subsubsection{Purpose\+:}\label{fal_element_ref_fal_elem_propertylinktargetr_1}
The {\ttfamily $<$Property\+Link\+Target$>$} element specifies a target widget suffix and property name to be used with a {\ttfamily $<$Property\+Link\+Definition$>$}. Whenever the property defined by the enclosing {\ttfamily $<$Property\+Link\+Definition$>$} is written to, the property specified in the {\ttfamily $<$Property\+Link\+Target$>$} will be written also. If the {\ttfamily $<$Property\+Link\+Target$>$} is the first linked target it will also be used as the \textquotesingle{}master linked target\textquotesingle{} and will be used as the source for read accesses to the property defined by the enclosing {\ttfamily $<$Property\+Link\+Definition$>$} element.\hypertarget{fal_element_ref_fal_elem_propertylinktarget_2}{}\subsubsection{Attributes\+:}\label{fal_element_ref_fal_elem_propertylinktarget_2}
\begin{DoxyItemize}
\item {\ttfamily widget} specifies either the name suffix of the child widget containing the property that is to be linked to the property being defined by the enclosing {\ttfamily $<$Property\+Link\+Definition$>$} element, or the special value {\ttfamily \char`\"{}\+\_\+\+\_\+parent\+\_\+\+\_\+\char`\"{}} to indicate a back-\/link to the parent. Required attribute. \item {\ttfamily property} specifies the name of the property on the child widget that is to be linked to the property being defined by the enclosing {\ttfamily $<$Property\+Link\+Definition$>$}. If this is omitted, it will be assumed that the target property has the same name as the property being defined. Optional attribute.\end{DoxyItemize}
\hypertarget{fal_element_ref_fal_elem_propertylinktarget_3}{}\subsubsection{Usage\+:}\label{fal_element_ref_fal_elem_propertylinktarget_3}

\begin{DoxyItemize}
\item The {\ttfamily $<$Property\+Link\+Target$>$} element may not contain sub-\/elements. 
\item The {\ttfamily $<$Property\+Link\+Target$>$} element may only appear as a sub-\/element within {\ttfamily $<$Property\+Link\+Definition$>$} elements. 
\end{DoxyItemize}\hypertarget{fal_element_ref_fal_elem_propertylinktarget_4}{}\subsubsection{Examples\+:}\label{fal_element_ref_fal_elem_propertylinktarget_4}
In this example we create a new property named \textquotesingle{}Text\+Colour\textquotesingle{}. This is linked to separate properties on two different child components. A property named \textquotesingle{}Label\+Colour\textquotesingle{} on a child widget with name suffix \textquotesingle{}{\bfseries{auto\+\_\+button1}}\textquotesingle{}, and a property named \textquotesingle{}Content\+Colour\textquotesingle{} on a child widget with name suffix \textquotesingle{}{\bfseries{auto\+\_\+content\+\_\+panel}}. Writing to the property \textquotesingle{}Text\+Colour\textquotesingle{} will cause both \textquotesingle{}Label\+Colour\textquotesingle{} and \textquotesingle{}Content\+Colour\textquotesingle{} to be updated, while reading the property \textquotesingle{}Text\+Colour\textquotesingle{} will fetch the value of \textquotesingle{}Label\+Colour\textquotesingle{} (because it appears first, it is treated as the master target). 
\begin{DoxyCode}{0}
\DoxyCodeLine{<WidgetLook name=\textcolor{stringliteral}{"PropertyLinkTargetExample"}>}
\DoxyCodeLine{  <PropertyLinkDefinition name=\textcolor{stringliteral}{"TextColour"} initialValue=\textcolor{stringliteral}{"FFFF00FF"} >}
\DoxyCodeLine{    <PropertyLinkTarget widget=\textcolor{stringliteral}{"\_\_auto\_button1\_\_"} \textcolor{keyword}{property}=\textcolor{stringliteral}{"LabelColour"} />}
\DoxyCodeLine{    <PropertyLinkTarget widget=\textcolor{stringliteral}{"\_\_auto\_content\_panel\_\_"} \textcolor{keyword}{property}=\textcolor{stringliteral}{"ContentColour"} />}
\DoxyCodeLine{  </PropertyLinkDefinition>}
\DoxyCodeLine{  ...}
\DoxyCodeLine{</WidgetLook>}
\end{DoxyCode}
\hypertarget{fal_element_ref_fal_elem_ref_sec_28}{}\subsection{$<$\+Property\+Dim$>$ Element}\label{fal_element_ref_fal_elem_ref_sec_28}
\hypertarget{fal_element_ref_fal_elem_ref_sec_28_1}{}\subsubsection{Purpose\+:}\label{fal_element_ref_fal_elem_ref_sec_28_1}
The {\ttfamily $<$Property\+Dim$>$} element is used to define a component dimension for an area rectangle. {\ttfamily $<$Property\+Dim$>$} is used to specify either a U\+Dim or a simple floating point value -\/ accessed via a window property -\/ for use as an area dimension.\hypertarget{fal_element_ref_fal_elem_ref_sec_28_2}{}\subsubsection{Attributes\+:}\label{fal_element_ref_fal_elem_ref_sec_28_2}
\begin{DoxyItemize}
\item {\ttfamily widget} specifies the name suffix of a child window to access the property for. The final name used to access the widget will be that of the target window with this suffix appended. If this suffix is not specified, the target window itself is used. Optional attribute. \item {\ttfamily name} specifies the name of the property that will provide the value for this dimension. The property named should access either a U\+Dim value or a simple floating point numerical value -\/ the system determines which type of value to expect based upon the usage (or not) of the optional {\ttfamily type} attribute. \item {\ttfamily type} specifies the dimension for which the relative scale value of a U\+Dim value is a portion of, and as such should be set to either {\ttfamily \char`\"{}\+Width\char`\"{}} or {\ttfamily \char`\"{}\+Height\char`\"{}}. If the property specified by the attribute {\ttfamily name} is intended to access a simple floating point value rather than a U\+Dim value, this attribute should be omitted. Optional attribute.\end{DoxyItemize}
\hypertarget{fal_element_ref_fal_elem_ref_sec_28_3}{}\subsubsection{Usage\+:}\label{fal_element_ref_fal_elem_ref_sec_28_3}

\begin{DoxyItemize}
\item The {\ttfamily $<$Property\+Dim$>$} element can appear as a sub-\/element in {\ttfamily $<$Dim$>$} to form a dimension specification for an area. 
\item The {\ttfamily $<$Property\+Dim$>$} element can appear as a sub-\/element of {\ttfamily $<$Operator\+Dim$>$} to specify one of the operands for a dimension calculation. 
\end{DoxyItemize}\hypertarget{fal_element_ref_fal_elem_ref_sec_28_4}{}\subsubsection{Examples\+:}\label{fal_element_ref_fal_elem_ref_sec_28_4}
This example shows a dimension that uses {\ttfamily $<$Property\+Dim$>$} to fetch a property value to use as the dimension value. We are accessing the \textquotesingle{}Absolute\+Width\textquotesingle{} property, from an attached widget with the name suffix \textquotesingle{}{\bfseries{auto\+\_\+button}}\textquotesingle{}, that accesses a simple floating point value\+:


\begin{DoxyCode}{0}
\DoxyCodeLine{...}
\DoxyCodeLine{<Area>}
\DoxyCodeLine{  <Dim type=\textcolor{stringliteral}{"LeftEdge"}>}
\DoxyCodeLine{    <PropertyDim widget=\textcolor{stringliteral}{"\_\_auto\_button\_\_"} name=\textcolor{stringliteral}{"AbsoluteWidth"} />}
\DoxyCodeLine{  </Dim>}
\DoxyCodeLine{  ...}
\DoxyCodeLine{</Area>}
\DoxyCodeLine{...}
\end{DoxyCode}
 Here we show a dimension that uses {\ttfamily $<$Property\+Dim$>$} to access a property named \textquotesingle{}User\+Width\textquotesingle{} on the target window. We have specified the {\ttfamily type} attribute to be \textquotesingle{}Width\textquotesingle{}, which indicates to the system that the property will access a U\+Dim value and that the scale part of the U\+Dim is relative to the width of the target window.


\begin{DoxyCode}{0}
\DoxyCodeLine{...}
\DoxyCodeLine{<Area>}
\DoxyCodeLine{  ...}
\DoxyCodeLine{  <Dim type=\textcolor{stringliteral}{"Width"}>}
\DoxyCodeLine{    <PropertyDim name=\textcolor{stringliteral}{"UserWidth"} type=\textcolor{stringliteral}{"Width"} />}
\DoxyCodeLine{  </Dim>}
\DoxyCodeLine{  ...}
\DoxyCodeLine{</Area>}
\DoxyCodeLine{...}
\end{DoxyCode}
\hypertarget{fal_element_ref_fal_elem_ref_sec_29}{}\subsection{$<$\+Section$>$ Element}\label{fal_element_ref_fal_elem_ref_sec_29}
\hypertarget{fal_element_ref_fal_elem_ref_sec_29_1}{}\subsubsection{Purpose\+:}\label{fal_element_ref_fal_elem_ref_sec_29_1}
The {\ttfamily $<$Section$>$} element is used to name an Imagery\+Section to be included for rendering within a State\+Imagery Layer definition.\hypertarget{fal_element_ref_fal_elem_ref_sec_29_2}{}\subsubsection{Attributes\+:}\label{fal_element_ref_fal_elem_ref_sec_29_2}
\begin{DoxyItemize}
\item {\ttfamily look} specifies the name of a Widget\+Look that contains the Imagery\+Section to be referenced. If this is omitted, the Widget\+Look currently being defined is used. Optional attribute. \item {\ttfamily section} specifies the name of an Imagery\+Section from the chosen Widget\+Look to be referenced. Required attribute. \item {\ttfamily control\+Property} specifies the name of a property that will be accessed to determine whether or not to render this section (see {\ttfamily control\+Value} below for details of how it is determined whether or not the secion will be drawn). Optional attribute. \item {\ttfamily control\+Value} specifies a value that will be compared to the value fetched from {\ttfamily control\+Property} in order to decide whether to draw the section. If {\ttfamily control\+Value} is not specified, or is specified as the empty string, then the property named in {\ttfamily control\+Property} is treated as a boolean value. Otherwise the value fetched from the property named in {\ttfamily control\+Propery} must exactly match the value specified in {\ttfamily control\+Value} for the imagery section to be drawn. Optional attribute. \item {\ttfamily control\+Widget} specifies either the name suffix of a child widget, or the special value {\ttfamily \char`\"{}\+\_\+\+\_\+parent\+\_\+\+\_\+\char`\"{}}, to indicate a widget that will be used as the source of the property named in {\ttfamily control\+Property}. If {\ttfamily control\+Widget} is not specified, or is specified as the empty string, then the widget with Widget\+Look assigned to it will be used as the source (matching the previous / default behaviour). Optional attribute.\end{DoxyItemize}
\hypertarget{fal_element_ref_fal_elem_ref_sec_29_3}{}\subsubsection{Usage\+:}\label{fal_element_ref_fal_elem_ref_sec_29_3}

\begin{DoxyItemize}
\item The {\ttfamily $<$Section$>$} element may only appear as a sub-\/element within the {\ttfamily $<$Layer$>$} element.


\item The {\ttfamily $<$Section$>$} element may specify colours to be modulated with any current colours used for each component within the named Imagery\+Section, by optionally specifying one of the colour elements as a sub-\/element\+: 
\begin{DoxyItemize}
\item {\ttfamily $<$Colours$>$} 
\item {\ttfamily $<$Colour\+Property$>$} 
\item {\ttfamily $<$Colour\+Rect\+Property$>$} 
\end{DoxyItemize}
\end{DoxyItemize}\hypertarget{fal_element_ref_fal_elem_ref_sec_29_4}{}\subsubsection{Examples\+:}\label{fal_element_ref_fal_elem_ref_sec_29_4}
Here we see a state definition from a button widget. The state specifies to use the \textquotesingle{}normal\textquotesingle{} imagery section, and also the \textquotesingle{}label\textquotesingle{} imagery section. The \textquotesingle{}label\textquotesingle{} section is only drawn if the \textquotesingle{}Draw\+Text\textquotesingle{} property of the target window is \textquotesingle{}True\textquotesingle{}. Colours for \textquotesingle{}label\textquotesingle{} will be modulated with the colour obtained from the \textquotesingle{}Normal\+Text\+Colour\textquotesingle{} property of the target window\+: 
\begin{DoxyCode}{0}
\DoxyCodeLine{...}
\DoxyCodeLine{<StateImagery name=\textcolor{stringliteral}{"Normal"}>}
\DoxyCodeLine{  <Layer>}
\DoxyCodeLine{    <Section section=\textcolor{stringliteral}{"normal"} />}
\DoxyCodeLine{    <Section section=\textcolor{stringliteral}{"label"} controlProperty=\textcolor{stringliteral}{"DrawText"}>}
\DoxyCodeLine{      <ColourProperty name=\textcolor{stringliteral}{"NormalTextColour"} />}
\DoxyCodeLine{    </Section>}
\DoxyCodeLine{  </Layer>}
\DoxyCodeLine{</StateImagery>}
\DoxyCodeLine{...}
\end{DoxyCode}
\hypertarget{fal_element_ref_fal_elem_ref_sec_30}{}\subsection{$<$\+State\+Imagery$>$ Element}\label{fal_element_ref_fal_elem_ref_sec_30}
\hypertarget{fal_element_ref_fal_elem_ref_sec_30_1}{}\subsubsection{Purpose\+:}\label{fal_element_ref_fal_elem_ref_sec_30_1}
The {\ttfamily $<$State\+Imagery$>$} element defines imagery to be used when rendering a named state. The base widget type intended as a target for the Widget\+Look being defined will specify which states are required to be defined.\hypertarget{fal_element_ref_fal_elem_ref_sec_30_2}{}\subsubsection{Attributes\+:}\label{fal_element_ref_fal_elem_ref_sec_30_2}
\begin{DoxyItemize}
\item {\ttfamily name} specifies the name of the state being defined. Required attribute. \item {\ttfamily clipped} boolean setting that states whether imagery defined within this state should be clipped to the target window\textquotesingle{}s defined area. If this is specified and set to false, the state imagery will only be clipped to the display area. Optional attribute.\end{DoxyItemize}
\hypertarget{fal_element_ref_fal_elem_ref_sec_30_3}{}\subsubsection{Usage\+:}\label{fal_element_ref_fal_elem_ref_sec_30_3}

\begin{DoxyItemize}
\item The {\ttfamily $<$State\+Imagery$>$element} may contain any number of {\ttfamily $<$Layer$>$} sub-\/elements. 
\item The {\ttfamily $<$State\+Imagery$>$} element may only appear as a sub-\/element of the {\ttfamily $<$Widget\+Look$>$} element. 
\end{DoxyItemize}\hypertarget{fal_element_ref_fal_elem_ref_sec_30_4}{}\subsubsection{Examples\+:}\label{fal_element_ref_fal_elem_ref_sec_30_4}
The following is an extract of the Menu\+Item definition from Taharez\+Look.\+looknfeel. The except defines some of the states required for that widget. Note that, although not shown here, a required state can be empty if no rendering is needed for that state\+: 
\begin{DoxyCode}{0}
\DoxyCodeLine{...}
\DoxyCodeLine{<StateImagery name=\textcolor{stringliteral}{"EnabledNormal"}>}
\DoxyCodeLine{  <Layer>}
\DoxyCodeLine{    <Section section=\textcolor{stringliteral}{"label"} />}
\DoxyCodeLine{  </Layer>}
\DoxyCodeLine{</StateImagery>}
\DoxyCodeLine{<StateImagery name=\textcolor{stringliteral}{"EnabledHover"}>}
\DoxyCodeLine{  <Layer>}
\DoxyCodeLine{    <Section section=\textcolor{stringliteral}{"frame"} />}
\DoxyCodeLine{    <Section section=\textcolor{stringliteral}{"label"} />}
\DoxyCodeLine{  </Layer>}
\DoxyCodeLine{</StateImagery>}
\DoxyCodeLine{<StateImagery name=\textcolor{stringliteral}{"EnabledPushed"}>}
\DoxyCodeLine{  <Layer>}
\DoxyCodeLine{    <Section section=\textcolor{stringliteral}{"frame"} />}
\DoxyCodeLine{    <Section section=\textcolor{stringliteral}{"label"} />}
\DoxyCodeLine{  </Layer>}
\DoxyCodeLine{</StateImagery>}
\DoxyCodeLine{...}
\end{DoxyCode}
\hypertarget{fal_element_ref_fal_elem_ref_sec_31}{}\subsection{$<$\+Text$>$ Element}\label{fal_element_ref_fal_elem_ref_sec_31}
\hypertarget{fal_element_ref_fal_elem_ref_sec_31_1}{}\subsubsection{Purpose\+:}\label{fal_element_ref_fal_elem_ref_sec_31_1}
The {\ttfamily $<$Text$>$} element is used to define font and text string information within a Text\+Component.\hypertarget{fal_element_ref_fal_elem_ref_sec_31_2}{}\subsubsection{Attributes\+:}\label{fal_element_ref_fal_elem_ref_sec_31_2}
\begin{DoxyItemize}
\item {\ttfamily font} specifies the name of a font to use for this text. If this is omitted, the current font of the target window will be used instead. Optional attribute. \item {\ttfamily string} specifies a text string to be rendered. If this is omitted, the current window text for the target window will be used instead. Optional attribute.\end{DoxyItemize}
\hypertarget{fal_element_ref_fal_elem_ref_sec_31_3}{}\subsubsection{Usage\+:}\label{fal_element_ref_fal_elem_ref_sec_31_3}

\begin{DoxyItemize}
\item The {\ttfamily $<$Text$>$} element may not contain any sub-\/elements. 
\item The {\ttfamily $<$Text$>$} element should only appear as a sub-\/element within {\ttfamily $<$Text\+Component$>$} elements. 
\end{DoxyItemize}\hypertarget{fal_element_ref_fal_elem_ref_sec_31_4}{}\subsubsection{Examples\+:}\label{fal_element_ref_fal_elem_ref_sec_31_4}
In this simple example, we define a Text\+Component that renders some static text. The {\ttfamily $<$Text$>$} element is used to specify the font and string to be used\+: 
\begin{DoxyCode}{0}
\DoxyCodeLine{...}
\DoxyCodeLine{<TextComponent>}
\DoxyCodeLine{  <Area>}
\DoxyCodeLine{    <Dim type=\textcolor{stringliteral}{"LeftEdge"} ><AbsoluteDim value=\textcolor{stringliteral}{"0"} /></Dim>}
\DoxyCodeLine{    <Dim type=\textcolor{stringliteral}{"TopEdge"} ><AbsoluteDim value=\textcolor{stringliteral}{"0"} /></Dim>}
\DoxyCodeLine{    <Dim type=\textcolor{stringliteral}{"Width"} ><UnifiedDim scale=\textcolor{stringliteral}{"1"} type=\textcolor{stringliteral}{"Width"} /></Dim>}
\DoxyCodeLine{    <Dim type=\textcolor{stringliteral}{"Height"} ><UnifiedDim scale=\textcolor{stringliteral}{"1"} type=\textcolor{stringliteral}{"Height"} /></Dim>}
\DoxyCodeLine{  </Area>}
\DoxyCodeLine{  <Text font=\textcolor{stringliteral}{"Roman-18"} \textcolor{keywordtype}{string}=\textcolor{stringliteral}{"Render this text!"} />}
\DoxyCodeLine{</TextComponent>}
\DoxyCodeLine{...}
\end{DoxyCode}
\hypertarget{fal_element_ref_fal_elem_ref_sec_32}{}\subsection{$<$\+Text\+Component$>$ Element}\label{fal_element_ref_fal_elem_ref_sec_32}
\hypertarget{fal_element_ref_fal_elem_ref_sec_32_1}{}\subsubsection{Purpose\+:}\label{fal_element_ref_fal_elem_ref_sec_32_1}
The {\ttfamily $<$Text\+Component$>$} element defines a single item of text to be drawn within a given Imagery\+Section. The Text\+Component contains all information about the text that is to be drawn, where it should be drawn, which colours are to be used and how the text should be formatted within its area.

Note that if the {\ttfamily $<$Text$>$} element appears in addition to either of the {\ttfamily $<$Text\+Property$>$} or {\ttfamily $<$Font\+Property$>$} elements, the string and font specified within the {\ttfamily $<$Text$>$} element will act as default values if the properties referenced in {\ttfamily $<$Text\+Property$>$} or {\ttfamily $<$Font\+Property$>$} evaluate to empty strings.\hypertarget{fal_element_ref_fal_elem_ref_sec_32_2}{}\subsubsection{Attributes\+:}\label{fal_element_ref_fal_elem_ref_sec_32_2}
The {\ttfamily $<$Text\+Component$>$} element has no attributes defined.\hypertarget{fal_element_ref_fal_elem_ref_sec_32_3}{}\subsubsection{Usage\+:}\label{fal_element_ref_fal_elem_ref_sec_32_3}
Note\+: the sub-\/elements should appear in the order that they are defined here.


\begin{DoxyItemize}
\item {\ttfamily $<$Area$>$} defining the target area for the text. 
\item {\ttfamily $<$Text$>$} optional element specifying the font to be used and text string to be drawn. 
\item {\ttfamily $<$Text\+Property$>$} optional element specifying the name of a property that contains the text to be drawn. 
\item {\ttfamily $<$Font\+Property$>$} optional element specifying the name of a property that contains the name of the font to use when drawing the text.


\item Optionally specifying the colours for this text, one of the colour elements\+: 
\begin{DoxyItemize}
\item {\ttfamily $<$Colours$>$} 
\item {\ttfamily $<$Colour\+Property$>$} 
\item {\ttfamily $<$Colour\+Rect\+Property$>$} 
\end{DoxyItemize}


\item Optionally, to specify the vertical formatting to use, either of\+: 
\begin{DoxyItemize}
\item {\ttfamily $<$Vert\+Format$>$} 
\item {\ttfamily $<$Vert\+Format\+Property$>$} 
\end{DoxyItemize}


\item Optionally, to specify the horizontal formatting to use, either of\+: 
\begin{DoxyItemize}
\item {\ttfamily $<$Horz\+Format$>$} 
\item {\ttfamily $<$Horz\+Format\+Property$>$} 
\end{DoxyItemize}


\item The {\ttfamily $<$Text\+Component$>$} element may only appear as a sub-\/element of the element {\ttfamily $<$Imagery\+Section$>$}. 
\end{DoxyItemize}\hypertarget{fal_element_ref_fal_elem_ref_sec_32_4}{}\subsubsection{Examples\+:}\label{fal_element_ref_fal_elem_ref_sec_32_4}
The following example could be used to specify the caption text to appear within a Titlebar style widget\+: 
\begin{DoxyCode}{0}
\DoxyCodeLine{...}
\DoxyCodeLine{<ImagerySection name=\textcolor{stringliteral}{"caption"}>}
\DoxyCodeLine{  <TextComponent>}
\DoxyCodeLine{    <Area>}
\DoxyCodeLine{      <Dim type=\textcolor{stringliteral}{"LeftEdge"} ><AbsoluteDim value=\textcolor{stringliteral}{"10"} /></Dim>}
\DoxyCodeLine{      <Dim type=\textcolor{stringliteral}{"TopEdge"} ><AbsoluteDim value=\textcolor{stringliteral}{"2"} /></Dim>}
\DoxyCodeLine{      <Dim type=\textcolor{stringliteral}{"Width"} ><UnifiedDim scale=\textcolor{stringliteral}{"1"} type=\textcolor{stringliteral}{"Width"} /></Dim>}
\DoxyCodeLine{      <Dim type=\textcolor{stringliteral}{"Height"} ><UnifiedDim scale=\textcolor{stringliteral}{"1"} type=\textcolor{stringliteral}{"Height"} /></Dim>}
\DoxyCodeLine{    </Area>}
\DoxyCodeLine{    <ColourProperty name=\textcolor{stringliteral}{"CaptionColour"} />}
\DoxyCodeLine{    <VertFormat type=\textcolor{stringliteral}{"CentreAligned"} />}
\DoxyCodeLine{    <HorzFormat type=\textcolor{stringliteral}{"WordWrapLeftAligned"} />}
\DoxyCodeLine{  </TextComponent>}
\DoxyCodeLine{</ImagerySection>}
\DoxyCodeLine{...}
\end{DoxyCode}
\hypertarget{fal_element_ref_fal_elem_ref_sec_33}{}\subsection{$<$\+Text\+Property$>$ Element}\label{fal_element_ref_fal_elem_ref_sec_33}
\hypertarget{fal_element_ref_fal_elem_ref_sec_33_1}{}\subsubsection{Purpose\+:}\label{fal_element_ref_fal_elem_ref_sec_33_1}
The {\ttfamily $<$Text\+Property$>$} element is intended to allow the system to access a property on the target window to obtain the text to be used when rendering the Text\+Component being defined.\hypertarget{fal_element_ref_fal_elem_ref_sec_33_2}{}\subsubsection{Attributes\+:}\label{fal_element_ref_fal_elem_ref_sec_33_2}
\begin{DoxyItemize}
\item {\ttfamily name} specifies the name of the property to access. The value of the named property is taken as a string to be rendered. Required attribute.\end{DoxyItemize}
\hypertarget{fal_element_ref_fal_elem_ref_sec_33_3}{}\subsubsection{Usage\+:}\label{fal_element_ref_fal_elem_ref_sec_33_3}

\begin{DoxyItemize}
\item The {\ttfamily $<$Text\+Property$>$} element may not contain sub-\/elements. 
\item The {\ttfamily $<$Text\+Property$>$} element may appear as a sub-\/element only within the {\ttfamily $<$Text\+Component$>$} element. 
\end{DoxyItemize}\hypertarget{fal_element_ref_fal_elem_ref_sec_33_4}{}\subsubsection{Examples\+:}\label{fal_element_ref_fal_elem_ref_sec_33_4}
\hypertarget{fal_element_ref_fal_elem_ref_sec_34}{}\subsection{$<$\+Unified\+Dim$>$ Element}\label{fal_element_ref_fal_elem_ref_sec_34}
\hypertarget{fal_element_ref_fal_elem_ref_sec_34_1}{}\subsubsection{Purpose\+:}\label{fal_element_ref_fal_elem_ref_sec_34_1}
The {\ttfamily $<$Unified\+Dim$>$} element is used to define a component dimension for an area rectangle. {\ttfamily $<$Unified\+Dim$>$} is used to specify a value using the \textquotesingle{}unified\textquotesingle{} co-\/ordinate system.\hypertarget{fal_element_ref_fal_elem_ref_sec_34_2}{}\subsubsection{Attributes\+:}\label{fal_element_ref_fal_elem_ref_sec_34_2}
\begin{DoxyItemize}
\item {\ttfamily scale} specifies the relative scale component of the U\+Dim. Optional attribute. \item {\ttfamily offset} specifies the absolute pixel component of the U\+Dim. Optional attribute. \item {\ttfamily type} specifies what the dimension represents. This is needed so that the system knows how to interpret the \textquotesingle{}scale\textquotesingle{} component. Required attribute.\end{DoxyItemize}
\hypertarget{fal_element_ref_fal_elem_ref_sec_34_3}{}\subsubsection{Usage\+:}\label{fal_element_ref_fal_elem_ref_sec_34_3}

\begin{DoxyItemize}
\item The {\ttfamily $<$Unified\+Dim$>$} element can appear as a sub-\/element in {\ttfamily $<$Dim$>$} to form a dimension specification for an area. 
\item The {\ttfamily $<$Unified\+Dim$>$} element can appear as a sub-\/element of {\ttfamily $<$Operator\+Dim$>$} to specify one of the operands for a dimension calculation. 
\end{DoxyItemize}\hypertarget{fal_element_ref_fal_elem_ref_sec_34_4}{}\subsubsection{Examples\+:}\label{fal_element_ref_fal_elem_ref_sec_34_4}
This example shows a dimension that uses {\ttfamily $<$Unified\+Dim$>$} to specify a U\+Dim value to use as the dimension\textquotesingle{}s value\+: 
\begin{DoxyCode}{0}
\DoxyCodeLine{...}
\DoxyCodeLine{<Area>}
\DoxyCodeLine{  <Dim type=\textcolor{stringliteral}{"LeftEdge"}>}
\DoxyCodeLine{    <UnfiedDim scale=\textcolor{stringliteral}{"0.5"} offset=\textcolor{stringliteral}{"-8"} type=\textcolor{stringliteral}{"LeftEdge"} />}
\DoxyCodeLine{  </Dim>}
\DoxyCodeLine{  ...}
\DoxyCodeLine{</Area>}
\DoxyCodeLine{...}
\end{DoxyCode}
\hypertarget{fal_element_ref_fal_elem_ref_sec_35}{}\subsection{$<$\+Vert\+Alignment$>$ Element}\label{fal_element_ref_fal_elem_ref_sec_35}
\hypertarget{fal_element_ref_fal_elem_ref_sec_35_1}{}\subsubsection{Purpose\+:}\label{fal_element_ref_fal_elem_ref_sec_35_1}
The {\ttfamily $<$Vert\+Alignment$>$} element is used to specify the vertical alignment option required for a child window element.\hypertarget{fal_element_ref_fal_elem_ref_sec_35_2}{}\subsubsection{Attributes\+:}\label{fal_element_ref_fal_elem_ref_sec_35_2}
\begin{DoxyItemize}
\item {\ttfamily type} specifies one of the values from the Vertical\+Alignment enumeration indicating the desired vertical alignment.\end{DoxyItemize}
\hypertarget{fal_element_ref_fal_elem_ref_sec_35_3}{}\subsubsection{Usage\+:}\label{fal_element_ref_fal_elem_ref_sec_35_3}

\begin{DoxyItemize}
\item The {\ttfamily $<$Vert\+Alignment$>$} element may only appear as a sub-\/element of the {\ttfamily $<$Child$>$} element. 
\item The {\ttfamily $<$Vert\+Alignment$>$} element may not contain any sub-\/elements. 
\end{DoxyItemize}\hypertarget{fal_element_ref_fal_elem_ref_sec_35_4}{}\subsubsection{Examples\+:}\label{fal_element_ref_fal_elem_ref_sec_35_4}
This example defines a scrollbar type child widget. We have used the {\ttfamily $<$Vert\+Alignment$>$} element to specify that the scrollbar appear on the bottom edge of the component being defined\+: 
\begin{DoxyCode}{0}
\DoxyCodeLine{...}
\DoxyCodeLine{<Child type=\textcolor{stringliteral}{"MyLook/HorzScrollbar"} nameSuffix=\textcolor{stringliteral}{"\_\_auto\_hscrollbar\_\_"}>}
\DoxyCodeLine{  <Area>}
\DoxyCodeLine{    <Dim type=\textcolor{stringliteral}{"LeftEdge"} ><AbsoluteDim value=\textcolor{stringliteral}{"0"} /></Dim>}
\DoxyCodeLine{    <Dim type=\textcolor{stringliteral}{"TopEdge"} ><AbsoluteDim value=\textcolor{stringliteral}{"0"} /></Dim>}
\DoxyCodeLine{    <Dim type=\textcolor{stringliteral}{"Width"} ><UnifiedDim scale=\textcolor{stringliteral}{"1"} type=\textcolor{stringliteral}{"Width? /></Dim>}}
\DoxyCodeLine{\textcolor{stringliteral}{    <Dim type="}Height\textcolor{stringliteral}{" ><AbsoluteDim value="}15\textcolor{stringliteral}{" /></Dim>}}
\DoxyCodeLine{\textcolor{stringliteral}{  </Area>}}
\DoxyCodeLine{\textcolor{stringliteral}{  <VertAlignment type="}BottomAligned\textcolor{stringliteral}{" />}}
\DoxyCodeLine{\textcolor{stringliteral}{</Child>}}
\DoxyCodeLine{\textcolor{stringliteral}{...}}
\end{DoxyCode}
\hypertarget{fal_element_ref_fal_elem_ref_sec_36}{}\subsection{$<$\+Vert\+Format$>$ Element}\label{fal_element_ref_fal_elem_ref_sec_36}
\hypertarget{fal_element_ref_fal_elem_ref_sec_36_1}{}\subsubsection{Purpose\+:}\label{fal_element_ref_fal_elem_ref_sec_36_1}
The {\ttfamily $<$Vert\+Format$>$} element is used to specify the required vertical formatting for an image, frame, or text component.\hypertarget{fal_element_ref_fal_elem_ref_sec_36_2}{}\subsubsection{Attributes\+:}\label{fal_element_ref_fal_elem_ref_sec_36_2}

\begin{DoxyItemize}
\item {\ttfamily type} specifies the required vertical formatting option. 
\begin{DoxyItemize}
\item For use in Imagery\+Components and Frame\+Components, this will be one of the values from the Vertical\+Format enumeration. 
\item For use in Text\+Components, this will one of the values form the Vertical\+Text\+Format enumeration. 
\end{DoxyItemize}
\item {\ttfamily component} Only for Frame\+Component. Specifies the part of the frame that this formatting is to be used for. Should be \char`\"{}\+Left\+Edge\char`\"{}, \char`\"{}\+Right\+Edge\char`\"{} or \char`\"{}\+Background\char`\"{} from the Frame\+Image\+Component enumeration. Optional attribute, defaults to \char`\"{}\+Background\char`\"{}. 
\end{DoxyItemize}\hypertarget{fal_element_ref_fal_elem_ref_sec_36_3}{}\subsubsection{Usage\+:}\label{fal_element_ref_fal_elem_ref_sec_36_3}

\begin{DoxyItemize}
\item The {\ttfamily $<$Vert\+Format$>$} element may only appear as a sub-\/element of the elements\+: 
\begin{DoxyItemize}
\item {\ttfamily $<$Imagery\+Component$>$} 
\item {\ttfamily $<$Frame\+Component$>$} 
\item {\ttfamily $<$Text\+Component$>$} 
\end{DoxyItemize}


\item The {\ttfamily $<$Vert\+Format$>$} element may not contain any sub-\/elements. 
\end{DoxyItemize}\hypertarget{fal_element_ref_fal_elem_ref_sec_36_4}{}\subsubsection{Examples\+:}\label{fal_element_ref_fal_elem_ref_sec_36_4}
This first example shows an Imagery\+Component definition. We use {\ttfamily $<$Vert\+Format$>$} to specify that we want the image tiled to cover the entire width of the designated target area\+: 
\begin{DoxyCode}{0}
\DoxyCodeLine{...}
\DoxyCodeLine{<ImageryComponent>}
\DoxyCodeLine{  <Area>}
\DoxyCodeLine{    <Dim type=\textcolor{stringliteral}{"LeftEdge"} ><AbsoluteDim value=\textcolor{stringliteral}{"0"} /></Dim>}
\DoxyCodeLine{    <Dim type=\textcolor{stringliteral}{"TopEdge"} ><AbsoluteDim value=\textcolor{stringliteral}{"0"} /></Dim>}
\DoxyCodeLine{    <Dim type=\textcolor{stringliteral}{"Width"} ><AbsoluteDim value=\textcolor{stringliteral}{"25"} /></Dim>}
\DoxyCodeLine{    <Dim type=\textcolor{stringliteral}{"Height"} ><AbsoluteDim value=\textcolor{stringliteral}{"25"} /></Dim>}
\DoxyCodeLine{  </Area>}
\DoxyCodeLine{  <Image imageset=\textcolor{stringliteral}{"myImageset"} image=\textcolor{stringliteral}{"coolImage"} />}
\DoxyCodeLine{  <VertFormat type=\textcolor{stringliteral}{"Tiled"} />}
\DoxyCodeLine{  <HorzFormat type=\textcolor{stringliteral}{"Stretched"} />}
\DoxyCodeLine{</ImageryComponent>}
\DoxyCodeLine{...}
\end{DoxyCode}


This second example is for a Text\+Component. You can see {\ttfamily $<$Vert\+Format$>$} used here to specify that we want the text centred within the target area\+: 
\begin{DoxyCode}{0}
\DoxyCodeLine{...}
\DoxyCodeLine{<TextComponent>}
\DoxyCodeLine{  <Area>}
\DoxyCodeLine{    <Dim type=\textcolor{stringliteral}{"LeftEdge"} ><AbsoluteDim value=\textcolor{stringliteral}{"0"} /></Dim>}
\DoxyCodeLine{    <Dim type=\textcolor{stringliteral}{"TopEdge"} ><AbsoluteDim value=\textcolor{stringliteral}{"0"} /></Dim>}
\DoxyCodeLine{    <Dim type=\textcolor{stringliteral}{"RightEdge"} ><UnifiedDim scale=\textcolor{stringliteral}{"1"} type=\textcolor{stringliteral}{"Width"} /></Dim>}
\DoxyCodeLine{    <Dim type=\textcolor{stringliteral}{"Height"} ><UnifiedDim scale=\textcolor{stringliteral}{"1"} type=\textcolor{stringliteral}{"Height"} /></Dim>}
\DoxyCodeLine{  </Area>}
\DoxyCodeLine{  <VertFormat type=\textcolor{stringliteral}{"CentreAligned"} />}
\DoxyCodeLine{</TextComponent>}
\DoxyCodeLine{...}
\end{DoxyCode}
\hypertarget{fal_element_ref_fal_elem_ref_sec_37}{}\subsection{$<$\+Vert\+Format\+Property$>$ Element}\label{fal_element_ref_fal_elem_ref_sec_37}
\hypertarget{fal_element_ref_fal_elem_ref_sec_37_1}{}\subsubsection{Purpose\+:}\label{fal_element_ref_fal_elem_ref_sec_37_1}
The {\ttfamily $<$Vert\+Format\+Property$>$} element is intended to allow the system to access a property on the target window to obtain the vertical formatting to be used when drawing the component being defined.\hypertarget{fal_element_ref_fal_elem_ref_sec_37_2}{}\subsubsection{Attributes\+:}\label{fal_element_ref_fal_elem_ref_sec_37_2}
\begin{DoxyItemize}
\item {\ttfamily name} specifies the name of the property to access. The named property must access a string value that will be set to one of the enumeration values appropriate for the component being defined (Vertical\+Text\+Format for Text\+Component, and Vertical\+Format for Frame\+Component and Imagery\+Component). Required attribute. \item {\ttfamily component} Only for Frame\+Component. Specifies the part of the frame that this formatting is to be used for. Should be \char`\"{}\+Left\+Edge\char`\"{}, \char`\"{}\+Right\+Edge\char`\"{} or \char`\"{}\+Background\char`\"{} from the Frame\+Image\+Component enumeration. Optional attribute, defaults to \char`\"{}\+Background\char`\"{}.\end{DoxyItemize}
\hypertarget{fal_element_ref_fal_elem_ref_sec_37_3}{}\subsubsection{Usage\+:}\label{fal_element_ref_fal_elem_ref_sec_37_3}

\begin{DoxyItemize}
\item The {\ttfamily $<$Vert\+Format\+Property$>$} element may not contain sub-\/elements.


\item The {\ttfamily $<$Vert\+Format\+Property$>$} element may appear as a sub-\/element within any of the following elements\+: 
\begin{DoxyItemize}
\item {\ttfamily $<$Imagery\+Component$>$} to specify a vertical formatting to be used the the image. 
\item {\ttfamily $<$Frame\+Component$>$} to specify a vertical formatting to be used for the frame background. 
\item {\ttfamily $<$Text\+Component$>$} to specify a vertical formatting to be used for the text. 
\end{DoxyItemize}
\end{DoxyItemize}\hypertarget{fal_element_ref_fal_elem_ref_sec_37_4}{}\subsubsection{Examples\+:}\label{fal_element_ref_fal_elem_ref_sec_37_4}
\hypertarget{fal_element_ref_fal_elem_ref_sec_38}{}\subsection{$<$\+Widget\+Dim$>$ Element}\label{fal_element_ref_fal_elem_ref_sec_38}
\hypertarget{fal_element_ref_fal_elem_ref_sec_38_1}{}\subsubsection{Purpose\+:}\label{fal_element_ref_fal_elem_ref_sec_38_1}
The {\ttfamily $<$Widget\+Dim$>$} element is used to define a component dimension for an area rectangle. {\ttfamily $<$Widget\+Dim$>$} is used to specify some dimension of an attached child widget for use as an area dimension.\hypertarget{fal_element_ref_fal_elem_ref_sec_38_2}{}\subsubsection{Attributes\+:}\label{fal_element_ref_fal_elem_ref_sec_38_2}
\begin{DoxyItemize}
\item {\ttfamily widget} specifies a suffix which will be used when building the name of the widget to access. The final name of the child widget will be that of the parent with this suffix appended. If this is not specified, the target window itself is used. Optional attribute. \item {\ttfamily dimension} specifies the widget dimension to be used. This should be set to one of the values defined in the Dimension\+Type enumeration. Required attribute.\end{DoxyItemize}
\hypertarget{fal_element_ref_fal_elem_ref_sec_38_3}{}\subsubsection{Usage\+:}\label{fal_element_ref_fal_elem_ref_sec_38_3}

\begin{DoxyItemize}
\item The {\ttfamily $<$Widget\+Dim$>$} element can appear as a sub-\/element in {\ttfamily $<$Dim$>$} to form a dimension specification for an area. 
\item The {\ttfamily $<$Widget\+Dim$>$} element can appear as a sub-\/element of {\ttfamily $<$Operator\+Dim$>$} to specify one of the operands for a dimension calculation. 
\end{DoxyItemize}\hypertarget{fal_element_ref_fal_elem_ref_sec_38_4}{}\subsubsection{Examples\+:}\label{fal_element_ref_fal_elem_ref_sec_38_4}
This example shows using {\ttfamily $<$Widget\+Dim$>$} to obtain dimensions from an attached child widget \textquotesingle{}{\bfseries{auto\+\_\+titlebar}}\textquotesingle{}, and also from the target window itself\+: 
\begin{DoxyCode}{0}
\DoxyCodeLine{...}
\DoxyCodeLine{<Area>}
\DoxyCodeLine{  <Dim type=\textcolor{stringliteral}{"LeftEdge"} >}
\DoxyCodeLine{    <AbsoluteDim value=\textcolor{stringliteral}{"0"} />}
\DoxyCodeLine{  </Dim>}
\DoxyCodeLine{  <Dim type=\textcolor{stringliteral}{"TopEdge"} >}
\DoxyCodeLine{    <WidgetDim widget=\textcolor{stringliteral}{"\_\_auto\_titlebar\_\_"} dimension=\textcolor{stringliteral}{"BottomEdge"} />}
\DoxyCodeLine{  </Dim>}
\DoxyCodeLine{  <Dim type=\textcolor{stringliteral}{"Width"} >}
\DoxyCodeLine{    <UnifiedDim scale=\textcolor{stringliteral}{"1"} type=\textcolor{stringliteral}{"Width"} />}
\DoxyCodeLine{  </Dim>}
\DoxyCodeLine{  <Dim type=\textcolor{stringliteral}{"BottomEdge"} >}
\DoxyCodeLine{    <WidgetDim dimension=\textcolor{stringliteral}{"BottomEdge"} />}
\DoxyCodeLine{  </Dim>}
\DoxyCodeLine{</Area>}
\DoxyCodeLine{...}
\end{DoxyCode}
\hypertarget{fal_element_ref_fal_elem_ref_sec_39}{}\subsection{$<$\+Widget\+Look$>$ Element}\label{fal_element_ref_fal_elem_ref_sec_39}
\hypertarget{fal_element_ref_fal_elem_ref_sec_39_1}{}\subsubsection{Purpose\+:}\label{fal_element_ref_fal_elem_ref_sec_39_1}
The {\ttfamily $<$Widget\+Look$>$} element is the most important element within the system. It defines a complete widget \textquotesingle{}look\textquotesingle{} which can be assigned to one of the Falagard base widget classes to create what is essentially a new widget type.\hypertarget{fal_element_ref_fal_elem_ref_sec_39_2}{}\subsubsection{Attributes\+:}\label{fal_element_ref_fal_elem_ref_sec_39_2}
\begin{DoxyItemize}
\item {\ttfamily name} specifies the name of the Widget\+Look being defined. If a Widget\+Look with this name already exists within the system, it will be replaced with the new definition. Required attribute.\end{DoxyItemize}
\hypertarget{fal_element_ref_fal_elem_ref_sec_39_3}{}\subsubsection{Usage\+:}\label{fal_element_ref_fal_elem_ref_sec_39_3}
Note\+: the sub-\/elements should appear in the order that they are defined here.


\begin{DoxyItemize}
\item The {\ttfamily $<$Widget\+Look$>$} element can contain the following sub-\/elements\+: 
\begin{DoxyItemize}
\item Any number of {\ttfamily $<$Property\+Definition$>$} sub-\/elements defining new properties. 
\item Any number of {\ttfamily $<$Property\+Link\+Definition$>$} sub-\/elements defining new linked properties. 
\item Any number of {\ttfamily $<$Property$>$} sub-\/elements specifying default property values. 
\item Any number of {\ttfamily $<$Named\+Area$>$} sub-\/elements defining areas within the widget. 
\item Any number of {\ttfamily $<$Child$>$} sub-\/elements defining component child widgets. 
\item Any number of {\ttfamily $<$Imagery\+Section$>$} sub-\/elements defining imagery for the widget. 
\item Any number of {\ttfamily $<$State\+Imagery$>$} sub-\/elements defining what to draw for widget states. 
\item Any number of {\ttfamily $<$Animation\+Definition$>$} sub-\/elements defining animations for the Widget\+Look. See\+: \mbox{\hyperlink{xml_animation}{Animation X\+ML files.}} 
\end{DoxyItemize}


\item The {\ttfamily $<$Widget\+Look$>$} element may only appear as sub-\/elements of the root {\ttfamily $<$Falagard$>$} element. 
\end{DoxyItemize}\hypertarget{fal_element_ref_fal_elem_ref_sec_39_4}{}\subsubsection{Examples\+:}\label{fal_element_ref_fal_elem_ref_sec_39_4}
The following example is the complete definition for \textquotesingle{}Taharez\+Look/\+List\+Header\textquotesingle{}. This is a trivial example that actually does no rendering, it just specifies a required property\+: 
\begin{DoxyCode}{0}
\DoxyCodeLine{<WidgetLook name=\textcolor{stringliteral}{"TaharezLook/ListHeader"}>}
\DoxyCodeLine{  <Property}
\DoxyCodeLine{    name=\textcolor{stringliteral}{"SegmentWidgetType"}}
\DoxyCodeLine{    value=\textcolor{stringliteral}{"TaharezLook/ListHeaderSegment"}}
\DoxyCodeLine{  />}
\DoxyCodeLine{  <StateImagery name=\textcolor{stringliteral}{"Enabled"} />}
\DoxyCodeLine{  <StateImagery name=\textcolor{stringliteral}{"Disabled"} />}
\DoxyCodeLine{</WidgetLook>}
\end{DoxyCode}
 \hypertarget{fal_enum_ref}{}\section{Falagard X\+ML Enumeration Reference}\label{fal_enum_ref}
\hypertarget{fal_enum_ref_fal_enum_ref_sec_0}{}\subsection{Section Contents}\label{fal_enum_ref_fal_enum_ref_sec_0}
\mbox{\hyperlink{fal_enum_ref_fal_enum_ref_sec_1}{Dimension\+Operator Enumeration}} ~\newline
 \mbox{\hyperlink{fal_enum_ref_fal_enum_ref_sec_2}{Dimension\+Type Enumeration}} ~\newline
 \mbox{\hyperlink{fal_enum_ref_fal_enum_ref_sec_3}{Font\+Metric\+Type Enumeration}} ~\newline
 \mbox{\hyperlink{fal_enum_ref_fal_enum_ref_sec_4}{Frame\+Image\+Component Enumeration}} ~\newline
 \mbox{\hyperlink{fal_enum_ref_fal_enum_ref_sec_5}{Horizontal\+Alignment Enumeration}} ~\newline
 \mbox{\hyperlink{fal_enum_ref_fal_enum_ref_sec_6}{Horizontal\+Format Enumeration}} ~\newline
 \mbox{\hyperlink{fal_enum_ref_fal_enum_ref_sec_7}{Horizontal\+Text\+Format Enumeration}} ~\newline
 \mbox{\hyperlink{fal_enum_ref_fal_enum_ref_sec_8}{Property\+Type Enumeration}} ~\newline
 \mbox{\hyperlink{fal_enum_ref_fal_enum_ref_sec_9}{Vertical\+Alignment Enumeration}} ~\newline
 \mbox{\hyperlink{fal_enum_ref_fal_enum_ref_sec_10}{Vertical\+Format Enumeration}} ~\newline
 \mbox{\hyperlink{fal_enum_ref_fal_enum_ref_sec_11}{Vertical\+Text\+Format Enumeration}} ~\newline
 \mbox{\hyperlink{fal_enum_ref_fal_enum_ref_sec_12}{Child\+Event\+Action Enumeration}} ~\newline
\hypertarget{fal_enum_ref_fal_enum_ref_sec_1}{}\subsection{Dimension\+Operator Enumeration}\label{fal_enum_ref_fal_enum_ref_sec_1}
\begin{DoxyItemize}
\item {\ttfamily \char`\"{}\+Noop\char`\"{}} does nothing. \item {\ttfamily \char`\"{}\+Add\char`\"{}} Adds two dimensions. \item {\ttfamily \char`\"{}\+Subtract\char`\"{}} Subtracts two dimensions. \item {\ttfamily \char`\"{}\+Multiply\char`\"{}} Multiplies two dimensions. \item {\ttfamily \char`\"{}\+Divide\char`\"{}} Divides two dimensions.\end{DoxyItemize}
\hypertarget{fal_enum_ref_fal_enum_ref_sec_2}{}\subsection{Dimension\+Type Enumeration}\label{fal_enum_ref_fal_enum_ref_sec_2}
\begin{DoxyItemize}
\item {\ttfamily \char`\"{}\+Left\+Edge\char`\"{}} specifies the left edge of the target item. \item {\ttfamily \char`\"{}\+Top\+Edge\char`\"{}} specifies the top edge of the target item. \item {\ttfamily \char`\"{}\+Right\+Edge\char`\"{}} specifies the right edge of the target item. \item {\ttfamily \char`\"{}\+Bottom\+Edge\char`\"{}} specifies the bottom edge of the target item. \item {\ttfamily \char`\"{}\+X\+Position\char`\"{}} specifies the x position co-\/ordinate of the target item (same as ?Left\+Edge?). \item {\ttfamily \char`\"{}\+Y\+Position\char`\"{}} specifies the y position co-\/ordinate of the target item (same as ?Top\+Edge?). \item {\ttfamily \char`\"{}\+Width\char`\"{}} specifies the width of the target item. \item {\ttfamily \char`\"{}\+Height\char`\"{}} specifies the height of the target item. \item {\ttfamily \char`\"{}\+X\+Offset\char`\"{}} specifies the x offset of the target item (only applies to Images). \item {\ttfamily \char`\"{}\+Y\+Offset\char`\"{}} specifies the y offset of the target item (only applies to Images).\end{DoxyItemize}
\hypertarget{fal_enum_ref_fal_enum_ref_sec_3}{}\subsection{Font\+Metric\+Type Enumeration}\label{fal_enum_ref_fal_enum_ref_sec_3}
\begin{DoxyItemize}
\item {\ttfamily \char`\"{}\+Line\+Spacing\char`\"{}} gets the vertical line spacing value of the font. \item {\ttfamily \char`\"{}\+Baseline\char`\"{}} get the vertical baseline value of the font. \item {\ttfamily \char`\"{}\+Horz\+Extent\char`\"{}} gets the horizontal extent of a string of text.\end{DoxyItemize}
\hypertarget{fal_enum_ref_fal_enum_ref_sec_4}{}\subsection{Frame\+Image\+Component Enumeration}\label{fal_enum_ref_fal_enum_ref_sec_4}
\begin{DoxyItemize}
\item {\ttfamily \char`\"{}\+Top\+Left\+Corner\char`\"{}} specifies the image be used for the frame\textquotesingle{}s top-\/left corner. \item {\ttfamily \char`\"{}\+Top\+Right\+Corner\char`\"{}} specifies the image be used for the frame\textquotesingle{}s top-\/right corner. \item {\ttfamily \char`\"{}\+Bottom\+Left\+Corner\char`\"{}} specifies the image be used for the frame\textquotesingle{}s bottom-\/left corner. \item {\ttfamily \char`\"{}\+Bottom\+Right\+Corner\char`\"{}} specifies the image be used for the frame\textquotesingle{}s bottom-\/right corner. \item {\ttfamily \char`\"{}\+Left\+Edge\char`\"{}} specifies the image be used for the frame\textquotesingle{} left edge. \item {\ttfamily \char`\"{}\+Right\+Edge\char`\"{}} specifies the image be used for the frame\textquotesingle{}s right edge. \item {\ttfamily \char`\"{}\+Top\+Edge\char`\"{}} specifies the image be used for the frame\textquotesingle{}s top edge. \item {\ttfamily \char`\"{}\+Bottom\+Edge\char`\"{}} specifies the image be used for the frame bottom edge. \item {\ttfamily \char`\"{}\+Background\char`\"{}} specifies the image be used for the frame\textquotesingle{}s background (area formed within all edges).\end{DoxyItemize}
\hypertarget{fal_enum_ref_fal_enum_ref_sec_5}{}\subsection{Horizontal\+Alignment Enumeration}\label{fal_enum_ref_fal_enum_ref_sec_5}
\begin{DoxyItemize}
\item {\ttfamily \char`\"{}\+Left\+Aligned\char`\"{}} x position is an offset of element\textquotesingle{}s left edges. \item {\ttfamily \char`\"{}\+Centre\+Aligned\char`\"{}} x position is an offset of element\textquotesingle{}s horizontal centre points. \item {\ttfamily \char`\"{}\+Right\+Aligned\char`\"{}} x position is an offset of element\textquotesingle{}s right edges.\end{DoxyItemize}
\hypertarget{fal_enum_ref_fal_enum_ref_sec_6}{}\subsection{Horizontal\+Format Enumeration}\label{fal_enum_ref_fal_enum_ref_sec_6}
\begin{DoxyItemize}
\item {\ttfamily \char`\"{}\+Left\+Aligned\char`\"{}} Image is left aligned within the prescribed area. \item {\ttfamily \char`\"{}\+Centre\+Aligned\char`\"{}} Image is horizontally centred within the prescribed area. \item {\ttfamily \char`\"{}\+Right\+Aligned\char`\"{}} Image is right aligned within the prescribed area. \item {\ttfamily \char`\"{}\+Stretched\char`\"{}} Image is horizontally stretched to fill the prescribed area. \item {\ttfamily \char`\"{}\+Tiled\char`\"{}} Image is horizontally tiled to fill the prescribed area.\end{DoxyItemize}
\hypertarget{fal_enum_ref_fal_enum_ref_sec_7}{}\subsection{Horizontal\+Text\+Format Enumeration}\label{fal_enum_ref_fal_enum_ref_sec_7}
\begin{DoxyItemize}
\item {\ttfamily \char`\"{}\+Left\+Aligned\char`\"{}} lines of text are left aligned within the prescribed area. \item {\ttfamily \char`\"{}\+Centre\+Aligned\char`\"{}} lines of text are horizontally centred within the prescribed area. \item {\ttfamily \char`\"{}\+Right\+Aligned\char`\"{}} lines of text are right aligned within the prescribed area. \item {\ttfamily \char`\"{}\+Justified\char`\"{}} lines of text are justified to the prescribed area. \item {\ttfamily \char`\"{}\+Word\+Wrap\+Left\+Aligned\char`\"{}} text wraps, with lines left aligned within the prescribed area. \item {\ttfamily \char`\"{}\+Word\+Wrap\+Centre\+Aligned\char`\"{}} text wraps, with lines horizontally centred in the prescribed area. \item {\ttfamily \char`\"{}\+Word\+Wrap\+Right\+Aligned\char`\"{}} text wraps, with lines right aligned within the prescribed area. \item {\ttfamily \char`\"{}\+Word\+Wrap\+Justified\char`\"{}} text wraps, within lines justified to the prescribed area.\end{DoxyItemize}
\hypertarget{fal_enum_ref_fal_enum_ref_sec_8}{}\subsection{Property\+Type Enumeration}\label{fal_enum_ref_fal_enum_ref_sec_8}
\begin{DoxyItemize}
\item {\ttfamily \char`\"{}\+Generic\char`\"{}} specifies a general purpose property (accessed as C\+E\+G\+U\+I\+::\+String). \item {\ttfamily \char`\"{}\+Colour\char`\"{}} specifies a property that accesses a single C\+E\+G\+U\+I\+::\+Colour value. \item {\ttfamily \char`\"{}\+Colour\+Rect\char`\"{}} specifies a property that accesses a C\+E\+G\+U\+I\+::\+Colour\+Rect value. \item {\ttfamily \char`\"{}\+U\+Box\char`\"{}} specifies a property that accesses a C\+E\+G\+U\+I\+::\+U\+Box value. \item {\ttfamily \char`\"{}\+U\+Rect\char`\"{}} specifies a property that accesses a C\+E\+G\+U\+I\+::\+U\+Rect value. \item {\ttfamily \char`\"{}\+U\+Size\char`\"{}} specifies a property that accesses a C\+E\+G\+U\+I\+::\+U\+Size value. \item {\ttfamily \char`\"{}\+U\+Dim\char`\"{}} specifies a property that accesses a C\+E\+G\+U\+I\+::\+U\+Dim value. \item {\ttfamily \char`\"{}\+U\+Vector2\char`\"{}} specifies a property that accesses a C\+E\+G\+U\+I\+::\+U\+Vector2 value. \item {\ttfamily \char`\"{}\+Sizef\char`\"{}} specifies a property that accesses a C\+E\+G\+U\+I\+::\+Sizef value. \item {\ttfamily \char`\"{}vec2\char`\"{}} specifies a property that accesses a glm\+::vec2 value. \item {\ttfamily \char`\"{}vec3\char`\"{}} specifies a property that accesses a glm\+::vec3 value. \item {\ttfamily \char`\"{}\+Rectf\char`\"{}} specifies a property that accesses a C\+E\+G\+U\+I\+::\+Rectf value. \item {\ttfamily \char`\"{}\+Font\char`\"{}} specifies a property that accesses a font via its name string. \item {\ttfamily \char`\"{}\+Image\char`\"{}} specifies a property that accesses an image via its name string. \item {\ttfamily \char`\"{}quat\char`\"{}} specifies a property that accesses a glm\+::quat value. \item {\ttfamily \char`\"{}\+Aspect\+Mode\char`\"{}} specifies a property that accesses a C\+E\+G\+U\+I\+::\+Aspect\+Mode value. \item {\ttfamily \char`\"{}\+Horizontal\+Alignment\char`\"{}} specifies a property that accesses a C\+E\+G\+U\+I\+::\+Horizontal\+Alignment enumerated value. \item {\ttfamily \char`\"{}\+Vertical\+Alignment\char`\"{}} specifies a property that accesses a C\+E\+G\+U\+I\+::\+Vertical\+Alignment enumerated value. \item {\ttfamily \char`\"{}\+Horizontal\+Text\+Formatting\char`\"{}} specifies a property that accesses a C\+E\+G\+U\+I\+::\+Horizontal\+Text\+Formatting enumerated value. \item {\ttfamily \char`\"{}\+Vertical\+Text\+Formatting\char`\"{}} specifies a property that accesses a C\+E\+G\+U\+I\+::\+Vertical\+Text\+Formatting enumerated value. \item {\ttfamily \char`\"{}\+Window\+Update\+Mode\char`\"{}} specifies a property that accesses a C\+E\+G\+U\+I\+::\+Window\+Update\+Mode enumerated value. \item {\ttfamily \char`\"{}bool\char`\"{}} specifies a property that accesses a simple boolean value. \item {\ttfamily \char`\"{}uint\char`\"{}} specifies a property that accesses a simple unsigned integer value. \item {\ttfamily \char`\"{}unsigned long\char`\"{}} specifies a property that accesses a simple unsigned long value. \item {\ttfamily \char`\"{}int\char`\"{}} specifies a property that accesses a simple signed integer value. \item {\ttfamily \char`\"{}float\char`\"{}} specifies a property that accesses a simple single precision floating point value. \item {\ttfamily \char`\"{}double\char`\"{}} specifies a property that accesses a simple double precision floating point value. \item {\ttfamily \char`\"{}\+Tab\+Control\+::\+Tab\+Pane\+Position\char`\"{}} specifies a property that accesses a C\+E\+G\+U\+I\+::\+Tab\+Control\+::\+Tab\+Pane\+Position enumerated value. \item {\ttfamily \char`\"{}\+Spinner\+::\+Text\+Input\+Mode\char`\"{}} specifies a property that accesses a C\+E\+G\+U\+I\+::\+Spinner\+::\+Text\+Input\+Mode enumerated value. \item {\ttfamily \char`\"{}\+Item\+List\+Base\+::\+Sort\+Mode\char`\"{}} specifies a property that accesses a C\+E\+G\+U\+I\+::\+Item\+List\+Base\+::\+Sort\+Direction enumerated value. \item {\ttfamily \char`\"{}\+List\+Header\+Segment\+::\+Sort\+Direction\char`\"{}} specifies a property that accesses a C\+E\+G\+U\+I\+::\+List\+Header\+Segment\+::\+Sort\+Direction enumerated value. \item {\ttfamily \char`\"{}\+Multi\+Column\+List\+::\+Selection\+Mode\char`\"{}} specifies a property that accesses a C\+E\+G\+U\+I\+::\+Multi\+Column\+List\+::\+Selection\+Mode enumerated value. \item {\ttfamily \char`\"{}\+String\char`\"{}} specifies a property that accesses a C\+E\+G\+U\+I\+::\+String value. \item {\ttfamily \char`\"{}\+Vert\+Formatting\char`\"{}} specifies a property that accesses a C\+E\+G\+U\+I\+::\+Vert\+Formatting enumerated value. \item {\ttfamily \char`\"{}\+Horz\+Formatting\char`\"{}} specifies a property that accesses a C\+E\+G\+U\+I\+::\+Horz\+Formatting enumerated value.\end{DoxyItemize}
\hypertarget{fal_enum_ref_fal_enum_ref_sec_9}{}\subsection{Vertical\+Alignment Enumeration}\label{fal_enum_ref_fal_enum_ref_sec_9}
\begin{DoxyItemize}
\item {\ttfamily \char`\"{}\+Top\+Aligned\char`\"{}} y position is an offset of element\textquotesingle{}s top edges. \item {\ttfamily \char`\"{}\+Centre\+Aligned\char`\"{}} y position is an offset of element\textquotesingle{}s vertical centre points. \item {\ttfamily \char`\"{}\+Bottom\+Aligned\char`\"{}} y position is an offset of element\textquotesingle{}s bottom edges.\end{DoxyItemize}
\hypertarget{fal_enum_ref_fal_enum_ref_sec_10}{}\subsection{Vertical\+Format Enumeration}\label{fal_enum_ref_fal_enum_ref_sec_10}
\begin{DoxyItemize}
\item {\ttfamily \char`\"{}\+Top\+Aligned\char`\"{}} Image is aligned with the top of the prescribed area. \item {\ttfamily \char`\"{}\+Centre\+Aligned\char`\"{}} Image is vertically centred within the prescribed area. \item {\ttfamily \char`\"{}\+Bottom\+Aligned\char`\"{}} Image is aligned with the bottom of the prescribed area. \item {\ttfamily \char`\"{}\+Stretched\char`\"{}} Image is vertically stretched to fill the prescribed area. \item {\ttfamily \char`\"{}\+Tiled\char`\"{}} Image is vertically tiled to fill the prescribed area.\end{DoxyItemize}
\hypertarget{fal_enum_ref_fal_enum_ref_sec_11}{}\subsection{Vertical\+Text\+Format Enumeration}\label{fal_enum_ref_fal_enum_ref_sec_11}
\begin{DoxyItemize}
\item {\ttfamily \char`\"{}\+Top\+Aligned\char`\"{}} Text line block is aligned with the top of the prescribed area. \item {\ttfamily \char`\"{}\+Centre\+Aligned\char`\"{}} Text line block is vertically centred within the prescribed area. \item {\ttfamily \char`\"{}\+Bottom\+Aligned\char`\"{}} Text line block is aligned with the bottom of the prescribed area.\end{DoxyItemize}
\hypertarget{fal_enum_ref_fal_enum_ref_sec_12}{}\subsection{Child\+Event\+Action Enumeration}\label{fal_enum_ref_fal_enum_ref_sec_12}
\begin{DoxyItemize}
\item {\ttfamily \char`\"{}\+Redraw\char`\"{}} When the event is fired on the child widget, the containing widget will be redrawn (non-\/recursive). \item {\ttfamily \char`\"{}\+Layout\char`\"{}} When the event is fired on the child widget, the containing widget will re-\/layout its child widget content. \end{DoxyItemize}
\hypertarget{fal_baseclass_ref}{}\section{C\+E\+G\+UI Widget Base Type Requirements}\label{fal_baseclass_ref}
The following is a reference to the required elements in a Widget\+Look as dictated by the widget base classes available within C\+E\+G\+UI. We also state the recommended window renderer to be mapped from the Falagard\+W\+R\+Base module, though you are free to use a custom window renderer as your needs dictate.\hypertarget{fal_baseclass_ref_fal_baseclass_ref_sec_0}{}\subsection{Section Contents}\label{fal_baseclass_ref_fal_baseclass_ref_sec_0}
\mbox{\hyperlink{fal_baseclass_ref_fal_baseclass_ref_sec_1}{Default\+Window}} ~\newline
 \mbox{\hyperlink{fal_baseclass_ref_fal_baseclass_ref_sec_2}{C\+E\+G\+U\+I/\+Checkbox}} ~\newline
 \mbox{\hyperlink{fal_baseclass_ref_fal_baseclass_ref_sec_3}{C\+E\+G\+U\+I/\+Combo\+Drop\+List}} ~\newline
 \mbox{\hyperlink{fal_baseclass_ref_fal_baseclass_ref_sec_4}{C\+E\+G\+U\+I/\+Combobox}} ~\newline
 \mbox{\hyperlink{fal_baseclass_ref_fal_baseclass_ref_sec_5}{C\+E\+G\+U\+I/\+Drag\+Container}} ~\newline
 \mbox{\hyperlink{fal_baseclass_ref_fal_baseclass_ref_sec_6}{C\+E\+G\+U\+I/\+Editbox}} ~\newline
 \mbox{\hyperlink{fal_baseclass_ref_fal_baseclass_ref_sec_7}{C\+E\+G\+U\+I/\+Frame\+Window}} ~\newline
 \mbox{\hyperlink{fal_baseclass_ref_fal_baseclass_ref_sec_8}{C\+E\+G\+U\+I/\+Item\+Entry}} ~\newline
 \mbox{\hyperlink{fal_baseclass_ref_fal_baseclass_ref_sec_9}{C\+E\+G\+U\+I/\+List\+Header}} ~\newline
 \mbox{\hyperlink{fal_baseclass_ref_fal_baseclass_ref_sec_10}{C\+E\+G\+U\+I/\+List\+Header\+Segment}} ~\newline
 \mbox{\hyperlink{fal_baseclass_ref_fal_baseclass_ref_sec_11}{C\+E\+G\+U\+I/\+List\+View}} ~\newline
 \mbox{\hyperlink{fal_baseclass_ref_fal_baseclass_ref_sec_12}{C\+E\+G\+U\+I/\+List\+Widget}} ~\newline
 \mbox{\hyperlink{fal_baseclass_ref_fal_baseclass_ref_sec_13}{C\+E\+G\+U\+I/\+Menu\+Item}} ~\newline
 \mbox{\hyperlink{fal_baseclass_ref_fal_baseclass_ref_sec_14}{C\+E\+G\+U\+I/\+Menubar}} ~\newline
 \mbox{\hyperlink{fal_baseclass_ref_fal_baseclass_ref_sec_15}{C\+E\+G\+U\+I/\+Multi\+Column\+List}} ~\newline
 \mbox{\hyperlink{fal_baseclass_ref_fal_baseclass_ref_sec_16}{C\+E\+G\+U\+I/\+Multi\+Line\+Editbox}} ~\newline
 \mbox{\hyperlink{fal_baseclass_ref_fal_baseclass_ref_sec_17}{C\+E\+G\+U\+I/\+Popup\+Menu}} ~\newline
 \mbox{\hyperlink{fal_baseclass_ref_fal_baseclass_ref_sec_18}{C\+E\+G\+U\+I/\+Progress\+Bar}} ~\newline
 \mbox{\hyperlink{fal_baseclass_ref_fal_baseclass_ref_sec_19}{C\+E\+G\+U\+I/\+Push\+Button}} ~\newline
 \mbox{\hyperlink{fal_baseclass_ref_fal_baseclass_ref_sec_20}{C\+E\+G\+U\+I/\+Radio\+Button}} ~\newline
 \mbox{\hyperlink{fal_baseclass_ref_fal_baseclass_ref_sec_21}{C\+E\+G\+U\+I/\+Scrollable\+Pane}} ~\newline
 \mbox{\hyperlink{fal_baseclass_ref_fal_baseclass_ref_sec_22}{C\+E\+G\+U\+I/\+Scrollbar}} ~\newline
 \mbox{\hyperlink{fal_baseclass_ref_fal_baseclass_ref_sec_23}{C\+E\+G\+U\+I/\+Slider}} ~\newline
 \mbox{\hyperlink{fal_baseclass_ref_fal_baseclass_ref_sec_24}{C\+E\+G\+U\+I/\+Spinner}} ~\newline
 \mbox{\hyperlink{fal_baseclass_ref_fal_baseclass_ref_sec_25}{C\+E\+G\+U\+I/\+Tab\+Button}} ~\newline
 \mbox{\hyperlink{fal_baseclass_ref_fal_baseclass_ref_sec_26}{C\+E\+G\+U\+I/\+Tab\+Control}} ~\newline
 \mbox{\hyperlink{fal_baseclass_ref_fal_baseclass_ref_sec_27}{C\+E\+G\+U\+I/\+Thumb}} ~\newline
 \mbox{\hyperlink{fal_baseclass_ref_fal_baseclass_ref_sec_28}{C\+E\+G\+U\+I/\+Titlebar}} ~\newline
 \mbox{\hyperlink{fal_baseclass_ref_fal_baseclass_ref_sec_29}{C\+E\+G\+U\+I/\+Tooltip}} ~\newline
 \mbox{\hyperlink{fal_baseclass_ref_fal_baseclass_ref_sec_30}{C\+E\+G\+U\+I/\+Tree\+View}} ~\newline
 \mbox{\hyperlink{fal_baseclass_ref_fal_baseclass_ref_sec_31}{C\+E\+G\+U\+I/\+Tree\+Widget}} ~\newline
\hypertarget{fal_baseclass_ref_fal_baseclass_ref_sec_1}{}\subsection{Default\+Window}\label{fal_baseclass_ref_fal_baseclass_ref_sec_1}
Base class intended to be used as a simple, generic container window. The logic for this class does nothing.

You should use a \char`\"{}\+Core/\+Default\char`\"{} window renderer for this widget.

Assigned Widget\+Look should provide the following\+: 
\begin{DoxyItemize}
\item This class currently has no Widget\+Look requirements. 
\end{DoxyItemize}\hypertarget{fal_baseclass_ref_fal_baseclass_ref_sec_2}{}\subsection{C\+E\+G\+U\+I/\+Checkbox}\label{fal_baseclass_ref_fal_baseclass_ref_sec_2}
Base class providing logic for Checkbox / toggle button widgets.

You should use a \char`\"{}\+Core/\+Toggle\+Button\char`\"{} window renderer for this widget.

Assigned Widget\+Look should provide the following\+: 
\begin{DoxyItemize}
\item This class currently has no Widget\+Look requirements. 
\end{DoxyItemize}\hypertarget{fal_baseclass_ref_fal_baseclass_ref_sec_3}{}\subsection{C\+E\+G\+U\+I/\+Combo\+Drop\+List}\label{fal_baseclass_ref_fal_baseclass_ref_sec_3}
Base class providing logic for the combo box drop down list sub-\/widget. This is a specialisation of the \char`\"{}\+C\+E\+G\+U\+I/\+List\+Widget\char`\"{} class.

You should use a \char`\"{}\+Core/\+List\+View\char`\"{} window renderer for this widget.

Assigned Widget\+Look should provide the following\+: 
\begin{DoxyItemize}
\item Child widget definitions\+: 
\begin{DoxyItemize}
\item Scrollbar based widget with name suffix \char`\"{}\+\_\+\+\_\+auto\+\_\+vscrollbar\+\_\+\+\_\+\char`\"{}. This widget will be used to control vertical scroll position. 
\item Scrollbar based widget with name suffix \char`\"{}\+\_\+\+\_\+auto\+\_\+hscrollbar\+\_\+\+\_\+\char`\"{}. This widget will be used to control horizontal scroll position. 
\end{DoxyItemize}
\end{DoxyItemize}\hypertarget{fal_baseclass_ref_fal_baseclass_ref_sec_4}{}\subsection{C\+E\+G\+U\+I/\+Combobox}\label{fal_baseclass_ref_fal_baseclass_ref_sec_4}
Base class providing logic for the combo box widget.

You should use a \char`\"{}\+Core/\+Default\char`\"{} window renderer for this widget.

Assigned Widget\+Look should provide the following\+: 
\begin{DoxyItemize}
\item Child widget definitions\+: 
\begin{DoxyItemize}
\item Editbox based widget with name suffix \char`\"{}\+\_\+\+\_\+auto\+\_\+editbox\+\_\+\+\_\+\char`\"{} 
\item Combo\+Drop\+List based widget with name suffix \char`\"{}\+\_\+\+\_\+auto\+\_\+droplist\+\_\+\+\_\+\char`\"{} 
\item Push\+Button based widget with name suffix \char`\"{}\+\_\+\+\_\+auto\+\_\+button\+\_\+\+\_\+\char`\"{} 
\end{DoxyItemize}
\end{DoxyItemize}\hypertarget{fal_baseclass_ref_fal_baseclass_ref_sec_5}{}\subsection{C\+E\+G\+U\+I/\+Drag\+Container}\label{fal_baseclass_ref_fal_baseclass_ref_sec_5}
Base class providing logic for a generic container that supports drag and drop.

You should use a \char`\"{}\+Core/\+Default\char`\"{} window renderer for this widget.

Assigned Widget\+Look should provide the following\+: 
\begin{DoxyItemize}
\item This class currently has no Widget\+Look requirements. 
\end{DoxyItemize}\hypertarget{fal_baseclass_ref_fal_baseclass_ref_sec_6}{}\subsection{C\+E\+G\+U\+I/\+Editbox}\label{fal_baseclass_ref_fal_baseclass_ref_sec_6}
Base class providing logic for a basic, single line, editbox / textbox widget.

You should use a \char`\"{}\+Core/\+Editbox\char`\"{} window renderer for this widget.

Assigned Widget\+Look should provide the following\+: 
\begin{DoxyItemize}
\item This class currently has no Widget\+Look requirements. 
\end{DoxyItemize}\hypertarget{fal_baseclass_ref_fal_baseclass_ref_sec_7}{}\subsection{C\+E\+G\+U\+I/\+Frame\+Window}\label{fal_baseclass_ref_fal_baseclass_ref_sec_7}
Base class providing logic for a window that is movable, sizable, and has a title-\/bar, frame, and a close button.

You should use a \char`\"{}\+Core/\+Frame\+Window\char`\"{} window renderer for this widget.

Assigned Widget\+Look should provide the following\+: 
\begin{DoxyItemize}
\item Child widget definitions\+: 
\begin{DoxyItemize}
\item Titlebar based widget with name suffix \char`\"{}\+\_\+\+\_\+auto\+\_\+titlebar\+\_\+\+\_\+\char`\"{}. This widget will be used as the title bar for the frame window. 
\item Push\+Button based widget with name suffix \char`\"{}\+\_\+\+\_\+auto\+\_\+closebutton\+\_\+\+\_\+\char`\"{}. This widget will be used as the close button for the frame window. 
\end{DoxyItemize}
\end{DoxyItemize}\hypertarget{fal_baseclass_ref_fal_baseclass_ref_sec_8}{}\subsection{C\+E\+G\+U\+I/\+Item\+Entry}\label{fal_baseclass_ref_fal_baseclass_ref_sec_8}
Base class providing logic for entries in supporting list widgets such as Item\+List\+Base.

You should use a \char`\"{}\+Core/\+Item\+Entry\char`\"{} window renderer for this widget.

Assigned Widget\+Look should provide the following\+: 
\begin{DoxyItemize}
\item This class currently has no Widget\+Look requirements. 
\end{DoxyItemize}\hypertarget{fal_baseclass_ref_fal_baseclass_ref_sec_9}{}\subsection{C\+E\+G\+U\+I/\+List\+Header}\label{fal_baseclass_ref_fal_baseclass_ref_sec_9}
Base class providing logic for a multi columned header widget -\/ intended for use on the multi column list.

You should use a \char`\"{}\+Core/\+List\+Header\char`\"{} window renderer for this widget.

Assigned Widget\+Look should provide the following\+: 
\begin{DoxyItemize}
\item This class currently has no Widget\+Look requirements. 
\end{DoxyItemize}\hypertarget{fal_baseclass_ref_fal_baseclass_ref_sec_10}{}\subsection{C\+E\+G\+U\+I/\+List\+Header\+Segment}\label{fal_baseclass_ref_fal_baseclass_ref_sec_10}
Base class providing logic for a widget representing single segment / column of the List\+Header widget.

You should use a \char`\"{}\+Core/\+List\+Header\+Segment\char`\"{} window renderer for this widget.

Assigned Widget\+Look should provide the following\+: 
\begin{DoxyItemize}
\item This class currently has no Widget\+Look requirements. 
\end{DoxyItemize}\hypertarget{fal_baseclass_ref_fal_baseclass_ref_sec_11}{}\subsection{C\+E\+G\+U\+I/\+List\+View}\label{fal_baseclass_ref_fal_baseclass_ref_sec_11}
Base class providing logic for a simple single column list view.

You should use a \char`\"{}\+Core/\+List\+View\char`\"{} window renderer for this view.

Assigned Widget\+Look should provide the following\+: 
\begin{DoxyItemize}
\item Child widget definitions\+: 
\begin{DoxyItemize}
\item Scrollbar based widget with name suffix \char`\"{}\+\_\+\+\_\+auto\+\_\+vscrollbar\+\_\+\+\_\+\char`\"{}. This widget will be used to control vertical scroll position. 
\item Scrollbar based widget with name suffix \char`\"{}\+\_\+\+\_\+auto\+\_\+hscrollbar\+\_\+\+\_\+\char`\"{}. This widget will be used to control horizontal scroll position. 
\end{DoxyItemize}
\end{DoxyItemize}\hypertarget{fal_baseclass_ref_fal_baseclass_ref_sec_12}{}\subsection{C\+E\+G\+U\+I/\+List\+Widget}\label{fal_baseclass_ref_fal_baseclass_ref_sec_12}
Base class providing logic for a simple single column list widget. This is a List\+View, with additional convenience methods, making it closer in functionality to the old List\+Box widget.

You should use a \char`\"{}\+Core/\+List\+Widget\char`\"{} window renderer for this widget.

Assigned Widget\+Look should provide the following\+: 
\begin{DoxyItemize}
\item Child widget definitions\+: 
\begin{DoxyItemize}
\item Scrollbar based widget with name suffix \char`\"{}\+\_\+\+\_\+auto\+\_\+vscrollbar\+\_\+\+\_\+\char`\"{}. This widget will be used to control vertical scroll position. 
\item Scrollbar based widget with name suffix \char`\"{}\+\_\+\+\_\+auto\+\_\+hscrollbar\+\_\+\+\_\+\char`\"{}. This widget will be used to control horizontal scroll position. 
\end{DoxyItemize}
\end{DoxyItemize}\hypertarget{fal_baseclass_ref_fal_baseclass_ref_sec_13}{}\subsection{C\+E\+G\+U\+I/\+Menu\+Item}\label{fal_baseclass_ref_fal_baseclass_ref_sec_13}
Base class providing logic for a Menu\+Item -\/ intended for attaching to Menubar and Popup\+Menu based widgets.

You should use a \char`\"{}\+Core/\+Menu\+Item\char`\"{} window renderer for this widget.

Assigned Widget\+Look should provide the following\+: 
\begin{DoxyItemize}
\item This class currently has no Widget\+Look requirements. 
\end{DoxyItemize}\hypertarget{fal_baseclass_ref_fal_baseclass_ref_sec_14}{}\subsection{C\+E\+G\+U\+I/\+Menubar}\label{fal_baseclass_ref_fal_baseclass_ref_sec_14}
Base class providing logic for a menu bar.

You should use a \char`\"{}\+Core/\+Menubar\char`\"{} window renderer for this widget.

Assigned Widget\+Look should provide the following\+: 
\begin{DoxyItemize}
\item This class currently has no Widget\+Look requirements. 
\end{DoxyItemize}\hypertarget{fal_baseclass_ref_fal_baseclass_ref_sec_15}{}\subsection{C\+E\+G\+U\+I/\+Multi\+Column\+List}\label{fal_baseclass_ref_fal_baseclass_ref_sec_15}
Base class providing logic for a multi-\/column list / grid widget supporting simple items based on non-\/window class Listbox\+Item.

You should use a \char`\"{}\+Core/\+Multi\+Column\+List\char`\"{} window renderer for this widget.

Assigned Widget\+Look should provide the following\+: 
\begin{DoxyItemize}
\item Child widget definitions\+: 
\begin{DoxyItemize}
\item Scrollbar based widget with name suffix \char`\"{}\+\_\+\+\_\+auto\+\_\+vscrollbar\+\_\+\+\_\+\char`\"{}. This widget will be used to control vertical scroll position. 
\item Scrollbar based widget with name suffix \char`\"{}\+\_\+\+\_\+auto\+\_\+hscrollbar\+\_\+\+\_\+\char`\"{}. This widget will be used to control horizontal scroll position. 
\item List\+Header based widget with name suffix \char`\"{}\+\_\+\+\_\+auto\+\_\+listheader\+\_\+\+\_\+\char`\"{}. This widget will be used for the header (though technically, you can place it anywhere). 
\end{DoxyItemize}
\end{DoxyItemize}\hypertarget{fal_baseclass_ref_fal_baseclass_ref_sec_16}{}\subsection{C\+E\+G\+U\+I/\+Multi\+Line\+Editbox}\label{fal_baseclass_ref_fal_baseclass_ref_sec_16}
Base class providing logic for a more advanced editbox / text box with support for multiple lines of text, word-\/wrapping, and so on.

You should use a \char`\"{}\+Core/\+Multi\+Line\+Editbox\char`\"{} window renderer for this widget.

Assigned Widget\+Look should provide the following\+: 
\begin{DoxyItemize}
\item Child widget definitions\+: 
\begin{DoxyItemize}
\item Scrollbar based widget with name suffix \char`\"{}\+\_\+\+\_\+auto\+\_\+vscrollbar\+\_\+\+\_\+\char`\"{}. This widget will be used to control vertical scroll position. 
\item Scrollbar based widget with name suffix \char`\"{}\+\_\+\+\_\+auto\+\_\+hscrollbar\+\_\+\+\_\+\char`\"{}. This widget will be used to control horizontal scroll position. 
\end{DoxyItemize}


\item Property initialiser definitions\+: 
\begin{DoxyItemize}
\item Selection\+Brush\+Image -\/ defines name of image that will be painted for the text selection (this is applied on a per-\/line basis). 
\end{DoxyItemize}
\end{DoxyItemize}\hypertarget{fal_baseclass_ref_fal_baseclass_ref_sec_17}{}\subsection{C\+E\+G\+U\+I/\+Popup\+Menu}\label{fal_baseclass_ref_fal_baseclass_ref_sec_17}
Base class providing logic for a pop-\/up menu.

You should use a \char`\"{}\+Core/\+Popup\+Menu\char`\"{} window renderer for this widget.

Assigned Widget\+Look should provide the following\+: 
\begin{DoxyItemize}
\item This class currently has no Widget\+Look requirements. 
\end{DoxyItemize}\hypertarget{fal_baseclass_ref_fal_baseclass_ref_sec_18}{}\subsection{C\+E\+G\+U\+I/\+Progress\+Bar}\label{fal_baseclass_ref_fal_baseclass_ref_sec_18}
Base class providing logic for progress bar widgets.

You should use a \char`\"{}\+Core/\+Progress\+Bar\char`\"{} window renderer for this widget.

Assigned Widget\+Look should provide the following\+: 
\begin{DoxyItemize}
\item This class currently has no Widget\+Look requirements. 
\end{DoxyItemize}\hypertarget{fal_baseclass_ref_fal_baseclass_ref_sec_19}{}\subsection{C\+E\+G\+U\+I/\+Push\+Button}\label{fal_baseclass_ref_fal_baseclass_ref_sec_19}
Base class providing logic for a simple push button type widget.

You should use a \char`\"{}\+Core/\+Button\char`\"{} window renderer for this widget.

Assigned Widget\+Look should provide the following\+: 
\begin{DoxyItemize}
\item This class currently has no Widget\+Look requirements. 
\end{DoxyItemize}\hypertarget{fal_baseclass_ref_fal_baseclass_ref_sec_20}{}\subsection{C\+E\+G\+U\+I/\+Radio\+Button}\label{fal_baseclass_ref_fal_baseclass_ref_sec_20}
Base class providing logic for radio button style widgets.

You should use a \char`\"{}\+Core/\+Toggle\+Button\char`\"{} window renderer for this widget.

Assigned Widget\+Look should provide the following\+: 
\begin{DoxyItemize}
\item This class currently has no Widget\+Look requirements. 
\end{DoxyItemize}\hypertarget{fal_baseclass_ref_fal_baseclass_ref_sec_21}{}\subsection{C\+E\+G\+U\+I/\+Scrollable\+Pane}\label{fal_baseclass_ref_fal_baseclass_ref_sec_21}
Base class providing logic for a widget that can scroll the content attached to it -\/ which may cover an area much larger than the viewable area.

You should use a \char`\"{}\+Core/\+Scrollable\+Pane\char`\"{} window renderer for this widget.

Assigned Widget\+Look should provide the following\+: 
\begin{DoxyItemize}
\item Child widget definitions\+: 
\begin{DoxyItemize}
\item Scrollbar based widget with name suffix \char`\"{}\+\_\+\+\_\+auto\+\_\+vscrollbar\+\_\+\+\_\+\char`\"{}. This widget will be used to control vertical scroll position. 
\item Scrollbar based widget with name suffix \char`\"{}\+\_\+\+\_\+auto\+\_\+hscrollbar\+\_\+\+\_\+\char`\"{}. This widget will be used to control horizontal scroll position. 
\end{DoxyItemize}
\end{DoxyItemize}\hypertarget{fal_baseclass_ref_fal_baseclass_ref_sec_22}{}\subsection{C\+E\+G\+U\+I/\+Scrollbar}\label{fal_baseclass_ref_fal_baseclass_ref_sec_22}
Base class providing logic for a scrollbar type widget with a movable thumb and increase / decrease buttons.

You should use a \char`\"{}\+Core/\+Scrollbar\char`\"{} window renderer for this widget.

Assigned Widget\+Look should provide the following\+: 
\begin{DoxyItemize}
\item Child widget definitions\+: 
\begin{DoxyItemize}
\item Thumb based widget with name suffix \char`\"{}\+\_\+\+\_\+auto\+\_\+thumb\+\_\+\+\_\+\char`\"{}. This widget will be used for the scrollbar thumb. 
\item Push\+Button based widget with name suffix \char`\"{}\+\_\+\+\_\+auto\+\_\+incbtn\+\_\+\+\_\+\char`\"{}. This widget will be used as the increase button. 
\item Push\+Button based widget with name suffix \char`\"{}\+\_\+\+\_\+auto\+\_\+decbtn\+\_\+\+\_\+\char`\"{}. This widget will be used as the decrease button. 
\end{DoxyItemize}
\end{DoxyItemize}\hypertarget{fal_baseclass_ref_fal_baseclass_ref_sec_23}{}\subsection{C\+E\+G\+U\+I/\+Slider}\label{fal_baseclass_ref_fal_baseclass_ref_sec_23}
Base class providing logic for a simple slider widget with a movable thumb.

You should use a \char`\"{}\+Core/\+Slider\char`\"{} window renderer for this widget.

Assigned Widget\+Look should provide the following\+: 
\begin{DoxyItemize}
\item Child widget definitions\+: 
\begin{DoxyItemize}
\item Thumb based widget with name suffix \char`\"{}\+\_\+\+\_\+auto\+\_\+thumb\+\_\+\+\_\+\char`\"{}. This widget will be used for the slider thumb. 
\end{DoxyItemize}
\end{DoxyItemize}\hypertarget{fal_baseclass_ref_fal_baseclass_ref_sec_24}{}\subsection{C\+E\+G\+U\+I/\+Spinner}\label{fal_baseclass_ref_fal_baseclass_ref_sec_24}
Base class providing logic for a numerical spinner widget, with a text entry box and increase / decrease buttons.

You should use a \char`\"{}\+Core/\+Default\char`\"{} window renderer for this widget.

Assigned Widget\+Look should provide the following\+: 
\begin{DoxyItemize}
\item Child widget definitions\+: 
\begin{DoxyItemize}
\item Editbox based widget with name suffix \char`\"{}\+\_\+\+\_\+auto\+\_\+editbox\+\_\+\+\_\+\char`\"{}. This widget will be used as the text box / display portion of the widget. 
\item Push\+Button based widget with name suffix \char`\"{}\+\_\+\+\_\+auto\+\_\+incbtn\+\_\+\+\_\+\char`\"{}. This widget will be used as the increase button. 
\item Push\+Button based widget with name suffix \char`\"{}\+\_\+\+\_\+auto\+\_\+decbtn\+\_\+\+\_\+\char`\"{}. This widget will be used as the decrease button. 
\end{DoxyItemize}
\end{DoxyItemize}\hypertarget{fal_baseclass_ref_fal_baseclass_ref_sec_25}{}\subsection{C\+E\+G\+U\+I/\+Tab\+Button}\label{fal_baseclass_ref_fal_baseclass_ref_sec_25}
Base class providing logic for the tabs within a Tab\+Control widget.

You should use a \char`\"{}\+Core/\+Tab\+Button\char`\"{} window renderer for this widget.

Assigned Widget\+Look should provide the following\+: 
\begin{DoxyItemize}
\item This class currently has no Widget\+Look requirements. 
\end{DoxyItemize}\hypertarget{fal_baseclass_ref_fal_baseclass_ref_sec_26}{}\subsection{C\+E\+G\+U\+I/\+Tab\+Control}\label{fal_baseclass_ref_fal_baseclass_ref_sec_26}
Base class providing logic for a widget supporting multiple tabbed content pages.

You should use a \char`\"{}\+Core/\+Tab\+Control\char`\"{} window renderer for this widget.

Assigned Widget\+Look should provide the following\+: 
\begin{DoxyItemize}
\item Child widget definitions\+: 
\begin{DoxyItemize}
\item Tab\+Pane based widget with name suffix \char`\"{}\+\_\+\+\_\+auto\+\_\+\+Tab\+Pane\+\_\+\+\_\+\char`\"{}. This widget will be used as the content viewing pane. 
\item Default\+Window based widget with name suffix \char`\"{}\+\_\+\+\_\+auto\+\_\+\+Tab\+Pane\+\_\+\+\_\+\+Buttons\char`\"{}. This widget will be used as a container for the tab buttons. Optional. 
\item Push\+Button based widget with name suffix \char`\"{}\+\_\+\+\_\+auto\+\_\+\+Tab\+Pane\+\_\+\+\_\+\+Scroll\+Left\char`\"{}. This widget is used to scroll the tab bar buttons left. Optional. 
\item Push\+Button based widget with name suffix \char`\"{}\+\_\+\+\_\+auto\+\_\+\+Tab\+Pane\+\_\+\+\_\+\+Scroll\+Right\char`\"{}. This widget is used to scroll the tab bar buttons right. Optional. 
\end{DoxyItemize}
\end{DoxyItemize}\hypertarget{fal_baseclass_ref_fal_baseclass_ref_sec_27}{}\subsection{C\+E\+G\+U\+I/\+Thumb}\label{fal_baseclass_ref_fal_baseclass_ref_sec_27}
Base class providing logic for a movable \textquotesingle{}tumb\textquotesingle{} button; for use as a component in other widgets such as scrollbars and sliders.

You should use a \char`\"{}\+Core/\+Button\char`\"{} window renderer for this widget.

Assigned Widget\+Look should provide the following\+: 
\begin{DoxyItemize}
\item This class currently has no Widget\+Look requirements. 
\end{DoxyItemize}\hypertarget{fal_baseclass_ref_fal_baseclass_ref_sec_28}{}\subsection{C\+E\+G\+U\+I/\+Titlebar}\label{fal_baseclass_ref_fal_baseclass_ref_sec_28}
Base class providing logic for a title / caption bar. This should only be used as a component of the Frame\+Window widget.

You should use a \char`\"{}\+Core/\+Titlebar\char`\"{} window renderer for this widget.

Assigned Widget\+Look should provide the following\+: 
\begin{DoxyItemize}
\item This class currently has no Widget\+Look requirements. 
\end{DoxyItemize}\hypertarget{fal_baseclass_ref_fal_baseclass_ref_sec_29}{}\subsection{C\+E\+G\+U\+I/\+Tooltip}\label{fal_baseclass_ref_fal_baseclass_ref_sec_29}
Base class providing logic for a simple tooltip type widget.

You should use a \char`\"{}\+Core/\+Tooltip\char`\"{} window renderer for this widget.

Assigned Widget\+Look should provide the following\+: 
\begin{DoxyItemize}
\item This class currently has no Widget\+Look requirements. 
\end{DoxyItemize}\hypertarget{fal_baseclass_ref_fal_baseclass_ref_sec_30}{}\subsection{C\+E\+G\+U\+I/\+Tree\+View}\label{fal_baseclass_ref_fal_baseclass_ref_sec_30}
Base class providing logic for a basic Tree type view.

You should use a \char`\"{}\+Core/\+Tree\+View\char`\"{} window renderer for this view.

Assigned Widget\+Look should provide the following\+: 
\begin{DoxyItemize}
\item Child view definitions\+: 
\begin{DoxyItemize}
\item Scrollbar based widget with name suffix \char`\"{}\+\_\+\+\_\+auto\+\_\+vscrollbar\+\_\+\+\_\+\char`\"{}. This widget will be used to control vertical scroll position. 
\item Scrollbar based widget with name suffix \char`\"{}\+\_\+\+\_\+auto\+\_\+hscrollbar\+\_\+\+\_\+\char`\"{}. This widget will be used to control horizontal scroll position. 
\end{DoxyItemize}


\item Imagery section definitions\+: 
\begin{DoxyItemize}
\item Imagery\+Section named \char`\"{}\+Subtree\+Expander\char`\"{}. This imagery is diplayed for the root of a subtree that is closed / collapsed to indicate that the subtree may be opened / expanded. 
\item Imagery\+Section named \char`\"{}\+Subtree\+Collapser\char`\"{}. This imagery is diplayed for the root of a subtree that is opened / expanded to indicate that the subtree may be closed / collapsed. 
\end{DoxyItemize}
\end{DoxyItemize}\hypertarget{fal_baseclass_ref_fal_baseclass_ref_sec_31}{}\subsection{C\+E\+G\+U\+I/\+Tree\+Widget}\label{fal_baseclass_ref_fal_baseclass_ref_sec_31}
Base class providing logic for a basic Tree type widget. This is a Tree\+View, with additional convenience methods, making it closer in functionality to the old Tree widget.

You should use a \char`\"{}\+Core/\+Tree\+Widget\char`\"{} window renderer for this widget.

Assigned Widget\+Look should provide the following\+: 
\begin{DoxyItemize}
\item Child widget definitions\+: 
\begin{DoxyItemize}
\item Scrollbar based widget with name suffix \char`\"{}\+\_\+\+\_\+auto\+\_\+vscrollbar\+\_\+\+\_\+\char`\"{}. This widget will be used to control vertical scroll position. 
\item Scrollbar based widget with name suffix \char`\"{}\+\_\+\+\_\+auto\+\_\+hscrollbar\+\_\+\+\_\+\char`\"{}. This widget will be used to control horizontal scroll position. 
\end{DoxyItemize}


\item Imagery section definitions\+: 
\begin{DoxyItemize}
\item Imagery\+Section named \char`\"{}\+Subtree\+Expander\char`\"{}. This imagery is diplayed for the root of a subtree that is closed / collapsed to indicate that the subtree may be opened / expanded. 
\item Imagery\+Section named \char`\"{}\+Subtree\+Collapser\char`\"{}. This imagery is diplayed for the root of a subtree that is opened / expanded to indicate that the subtree may be closed / collapsed. 
\end{DoxyItemize}
\end{DoxyItemize}\hypertarget{fal_wr_ref}{}\section{Falagard Window Renderer Requirements}\label{fal_wr_ref}
\hypertarget{fal_wr_ref_fal_wr_ref_sec_0}{}\subsection{Section Contents}\label{fal_wr_ref_fal_wr_ref_sec_0}
\mbox{\hyperlink{fal_wr_ref_fal_wr_ref_sec_1}{Falagard/\+Button}} ~\newline
 \mbox{\hyperlink{fal_wr_ref_fal_wr_ref_sec_2}{Falagard/\+Default}} ~\newline
 \mbox{\hyperlink{fal_wr_ref_fal_wr_ref_sec_3}{Falagard/\+Editbox}} ~\newline
 \mbox{\hyperlink{fal_wr_ref_fal_wr_ref_sec_4}{Falagard/\+Frame\+Window}} ~\newline
 \mbox{\hyperlink{fal_wr_ref_fal_wr_ref_sec_5}{Falagard/\+Item\+Entry}} ~\newline
 \mbox{\hyperlink{fal_wr_ref_fal_wr_ref_sec_6}{Falagard/\+List\+View}} ~\newline
 \mbox{\hyperlink{fal_wr_ref_fal_wr_ref_sec_7}{Falagard/\+List\+Widget}} ~\newline
 \mbox{\hyperlink{fal_wr_ref_fal_wr_ref_sec_8}{Falagard/\+List\+Header}} ~\newline
 \mbox{\hyperlink{fal_wr_ref_fal_wr_ref_sec_9}{Falagard/\+List\+Header\+Segment}} ~\newline
 \mbox{\hyperlink{fal_wr_ref_fal_wr_ref_sec_10}{Falagard/\+Menubar}} ~\newline
 \mbox{\hyperlink{fal_wr_ref_fal_wr_ref_sec_11}{Falagard/\+Menu\+Item}} ~\newline
 \mbox{\hyperlink{fal_wr_ref_fal_wr_ref_sec_12}{Falagard/\+Multi\+Column\+List}} ~\newline
 \mbox{\hyperlink{fal_wr_ref_fal_wr_ref_sec_13}{Falagard/\+Multi\+Line\+Editbox}} ~\newline
 \mbox{\hyperlink{fal_wr_ref_fal_wr_ref_sec_14}{Falagard/\+Popup\+Menu}} ~\newline
 \mbox{\hyperlink{fal_wr_ref_fal_wr_ref_sec_15}{Falagard/\+Progress\+Bar}} ~\newline
 \mbox{\hyperlink{fal_wr_ref_fal_wr_ref_sec_16}{Falagard/\+Toggle\+Button}} ~\newline
 \mbox{\hyperlink{fal_wr_ref_fal_wr_ref_sec_17}{Falagard/\+Scrollable\+Pane}} ~\newline
 \mbox{\hyperlink{fal_wr_ref_fal_wr_ref_sec_18}{Falagard/\+Scrollbar}} ~\newline
 \mbox{\hyperlink{fal_wr_ref_fal_wr_ref_sec_19}{Falagard/\+Slider}} ~\newline
 \mbox{\hyperlink{fal_wr_ref_fal_wr_ref_sec_20}{Falagard/\+Static}} ~\newline
 \mbox{\hyperlink{fal_wr_ref_fal_wr_ref_sec_21}{Falagard/\+Static\+Image}} ~\newline
 \mbox{\hyperlink{fal_wr_ref_fal_wr_ref_sec_22}{Falagard/\+Static\+Text}} ~\newline
 \mbox{\hyperlink{fal_wr_ref_fal_wr_ref_sec_23}{Falagard/\+System\+Button}} ~\newline
 \mbox{\hyperlink{fal_wr_ref_fal_wr_ref_sec_24}{Falagard/\+Tab\+Button}} ~\newline
 \mbox{\hyperlink{fal_wr_ref_fal_wr_ref_sec_25}{Falagard/\+Tab\+Control}} ~\newline
 \mbox{\hyperlink{fal_wr_ref_fal_wr_ref_sec_26}{Falagard/\+Titlebar}} ~\newline
 \mbox{\hyperlink{fal_wr_ref_fal_wr_ref_sec_27}{Falagard/\+Tooltip}} ~\newline
 \mbox{\hyperlink{fal_wr_ref_fal_wr_ref_sec_28}{Falagard/\+Tree\+View}} ~\newline
 \mbox{\hyperlink{fal_wr_ref_fal_wr_ref_sec_29}{Falagard/\+Tree\+Widget}} ~\newline
\hypertarget{fal_wr_ref_fal_wr_ref_sec_1}{}\subsection{Falagard/\+Button}\label{fal_wr_ref_fal_wr_ref_sec_1}
General purpose push button widget class.

Assigned Widget\+Look should provide the following\+: 
\begin{DoxyItemize}
\item State\+Imagery definitions (missing states will default to \textquotesingle{}Normal\textquotesingle{})\+: 
\begin{DoxyItemize}
\item Normal -\/ Imagery used when the widget is neither pushed nor has the mouse hovering over it. 
\item Hover -\/ Imagery used when the widget is not pushed and has the mouse hovering over it. 
\item Pushed -\/ Imagery used when the widget is pushed and the mouse is over the widget. 
\item Pushed\+Off -\/ Imagery used when the widget is pushed and the mouse is not over the widget. 
\item Disabled -\/ Imagery used when the widget is disabled. 
\end{DoxyItemize}
\end{DoxyItemize}\hypertarget{fal_wr_ref_fal_wr_ref_sec_2}{}\subsection{Falagard/\+Default}\label{fal_wr_ref_fal_wr_ref_sec_2}
Generic window which can be used as a container window, amongst other uses.

Assigned Widget\+Look should provide the following\+: 
\begin{DoxyItemize}
\item State\+Imagery definitions\+: 
\begin{DoxyItemize}
\item Enabled -\/ General imagery for when the widget is enabled. 
\item Disabled -\/ General imagery for when the widget is disabled. 
\end{DoxyItemize}
\end{DoxyItemize}\hypertarget{fal_wr_ref_fal_wr_ref_sec_3}{}\subsection{Falagard/\+Editbox}\label{fal_wr_ref_fal_wr_ref_sec_3}
General purpose single-\/line text box widget.

Assigned Widget\+Look should provide the following\+: 
\begin{DoxyItemize}
\item State\+Imagery definitions\+: 
\begin{DoxyItemize}
\item Enabled -\/ Imagery used when widget is enabled. 
\item Disabled -\/ Imagery used when widget is disabled. 
\item Read\+Only -\/ Imagery used when widget is in \textquotesingle{}Read Only\textquotesingle{} state. 
\item Active\+Selection -\/ Additional imagery used when a text selection is defined and the widget is active. The imagery for this state will be rendered within the selection area. 
\item Inactive\+Selection -\/ Additional imagery used when a text selection is defined and the widget is not active. The imagery for this state will be rendered within the selection area. 
\end{DoxyItemize}


\item Named\+Area definitions\+: 
\begin{DoxyItemize}
\item Text\+Area -\/ Defines the area where the text, carat, and any selection imagery will appear. 
\end{DoxyItemize}


\item Property\+Definition specifications (optional, defaults will be black)\+: 
\begin{DoxyItemize}
\item Normal\+Text\+Colour -\/ property that accesses a colour value to be used to render normal unselected text. 
\item Selected\+Text\+Colour -\/ property that accesses a colour value to be used to render selected text. 
\end{DoxyItemize}


\item Imagery\+Section definitions\+: 
\begin{DoxyItemize}
\item Carat -\/ Additional imagery used to display the insertion position carat. 
\end{DoxyItemize}
\end{DoxyItemize}\hypertarget{fal_wr_ref_fal_wr_ref_sec_4}{}\subsection{Falagard/\+Frame\+Window}\label{fal_wr_ref_fal_wr_ref_sec_4}
General purpose window type which can be sized and moved.

Assigned Widget\+Look should provide the following\+: 
\begin{DoxyItemize}
\item State\+Imagery definitions\+: 
\begin{DoxyItemize}
\item Active\+With\+Title\+With\+Frame -\/ Imagery used when the widget has its title bar enabled, has its frame enabled, and is active. 
\item Inactive\+With\+Title\+With\+Frame -\/ Imagery used when the widget has its title bar enabled, has its frame enabled, and is inactive. 
\item Disabled\+With\+Title\+With\+Frame -\/ Imagery used when the widget has its title bar enabled, has its frame enabled, and is disabled. 
\item Active\+With\+Title\+No\+Frame -\/ Imagery used when the widget has its title bar enabled, has its frame disabled, and is active. 
\item Inactive\+With\+Title\+No\+Frame -\/ Imagery used when the widget has its title bar enabled, has its frame disabled, and is inactive. 
\item Disabled\+With\+Title\+No\+Frame -\/ Imagery used when the widget has its title bar enabled, has its frame disabled, and is disabled. 
\item Active\+No\+Title\+With\+Frame -\/ Imagery used when the widget has its title bar disabled, has its frame enabled, and is active. 
\item Inactive\+No\+Title\+With\+Frame -\/ Imagery used when the widget has its title bar disabled, has its frame enabled, and is inactive. 
\item Disabled\+No\+Title\+With\+Frame -\/ Imagery used when the widget has its title bar disabled, has its frame enabled, and is disabled. 
\item Active\+No\+Title\+No\+Frame -\/ Imagery used when the widget has its title bar disabled, has its frame disabled, and is active. 
\item Inactive\+No\+Title\+No\+Frame -\/ Imagery used when the widget has its title bar disabled, has its frame disabled, and is inactive. 
\item Disabled\+No\+Title\+No\+Frame -\/ Imagery used when the widget has its title bar disabled, has its frame disabled, and is disabled. 
\end{DoxyItemize}


\item Named\+Area definitions\+: 
\begin{DoxyItemize}
\item Client\+With\+Title\+With\+Frame -\/ Area that defines the clipping region for the client area when the widget has its title bar enabled, and has its frame enabled. 
\item Client\+With\+Title\+No\+Frame -\/ Area that defines the clipping region for the client area when the widget has its title bar enabled, and has its frame disabled. 
\item Client\+No\+Title\+With\+Frame -\/ Area that defines the clipping region for the client area when the widget has its title bar disabled, and has its frame enabled. 
\item Client\+No\+Title\+No\+Frame -\/ Area that defines the clipping region for the client area when the widget has its title bar disabled, and has its frame disabled. 
\end{DoxyItemize}
\end{DoxyItemize}\hypertarget{fal_wr_ref_fal_wr_ref_sec_5}{}\subsection{Falagard/\+Item\+Entry}\label{fal_wr_ref_fal_wr_ref_sec_5}
Basic class that may be added to any of the Item\+List\+Base base classes.

Assigned Widget\+Look should provide the following\+: 
\begin{DoxyItemize}
\item State\+Imagery definitions\+: 
\begin{DoxyItemize}
\item Enabled -\/ General imagery for when the widget is enabled. 
\item Disabled -\/ General imagery for when the widget is disabled. 
\end{DoxyItemize}


\item Named\+Area definitions\+: 
\begin{DoxyItemize}
\item Content\+Size -\/ Area defining the size of the item content. Required. 
\end{DoxyItemize}
\end{DoxyItemize}\hypertarget{fal_wr_ref_fal_wr_ref_sec_6}{}\subsection{Falagard/\+List\+View}\label{fal_wr_ref_fal_wr_ref_sec_6}
General purpose single column list view.

Assigned Widget\+Look should provide the following\+: 
\begin{DoxyItemize}
\item State\+Imagery definitions\+: 
\begin{DoxyItemize}
\item Enabled -\/ General imagery for when the view is enabled. 
\item Disabled -\/ General imagery for when the view is disabled. 
\item {\itshape Optional} Enabled\+Focused -\/ General imagery for when the view is enabled {\itshape and} focused. 
\end{DoxyItemize}


\item Named\+Area definitions (you should choose one set, or the other)\+: 
\begin{DoxyItemize}
\item Item\+Render\+Area -\/ Target area where list items will appear when no scrollbars are visible (also acts as default area). Required. 
\item Item\+Render\+Area\+H\+Scroll -\/ Target area where list items will appear when the horizontal scrollbar is visible. Optional. 
\item Item\+Render\+Area\+V\+Scroll -\/ Target area where list items will appear when the vertical scrollbar is visible. Optional. 
\item Item\+Render\+Area\+H\+V\+Scroll -\/ Target area where list items will appear when both the horizontal and vertical scrollbars are visible. Optional. 
\end{DoxyItemize}
\item OR\+: 
\begin{DoxyItemize}
\item Item\+Rendering\+Area -\/ Target area where list items will appear when no scrollbars are visible (also acts as default area). Required. 
\item Item\+Rendering\+Area\+H\+Scroll -\/ Target area where list items will appear when the horizontal scrollbar is visible. Optional. 
\item Item\+Rendering\+Area\+V\+Scroll -\/ Target area where list items will appear when the vertical scrollbar is visible. Optional. 
\item Item\+Rendering\+Area\+H\+V\+Scroll -\/ Target area where list items will appear when both the horizontal and vertical scrollbars are visible. Optional. 
\end{DoxyItemize}
\end{DoxyItemize}\hypertarget{fal_wr_ref_fal_wr_ref_sec_7}{}\subsection{Falagard/\+List\+Widget}\label{fal_wr_ref_fal_wr_ref_sec_7}
General purpose single column list widget.

Assigned Widget\+Look should provide the following\+: 
\begin{DoxyItemize}
\item State\+Imagery definitions\+: 
\begin{DoxyItemize}
\item Enabled -\/ General imagery for when the widget is enabled. 
\item Disabled -\/ General imagery for when the widget is disabled. 
\item \+\_\+\+Optional\+\_\+ Enabled\+Focused -\/ General imagery for when the widget is enabled {\itshape and} focused. 
\end{DoxyItemize}


\item Named\+Area definitions (you should choose one set, or the other)\+: 
\begin{DoxyItemize}
\item Item\+Rendering\+Area -\/ Target area where list items will appear when no scrollbars are visible (also acts as default area). Required. 
\item Item\+Rendering\+Area\+H\+Scroll -\/ Target area where list items will appear when the horizontal scrollbar is visible. Optional. 
\item Item\+Rendering\+Area\+V\+Scroll -\/ Target area where list items will appear when the vertical scrollbar is visible. Optional. 
\item Item\+Rendering\+Area\+H\+V\+Scroll -\/ Target area where list items will appear when both the horizontal and vertical scrollbars are visible. Optional. 
\end{DoxyItemize}
\item OR\+: 
\begin{DoxyItemize}
\item Item\+Render\+Area -\/ Target area where list items will appear when no scrollbars are visible (also acts as default area). Required. 
\item Item\+Render\+Area\+H\+Scroll -\/ Target area where list items will appear when the horizontal scrollbar is visible. Optional. 
\item Item\+Render\+Area\+V\+Scroll -\/ Target area where list items will appear when the vertical scrollbar is visible. Optional. 
\item Item\+Render\+Area\+H\+V\+Scroll -\/ Target area where list items will appear when both the horizontal and vertical scrollbars are visible. Optional. 
\end{DoxyItemize}
\end{DoxyItemize}\hypertarget{fal_wr_ref_fal_wr_ref_sec_8}{}\subsection{Falagard/\+List\+Header}\label{fal_wr_ref_fal_wr_ref_sec_8}
List header widget. Acts as a container for List\+Header\+Segment based widgets. Usually used as a component part widget for multi-\/column list widgets.

Assigned Widget\+Look should provide the following\+: 
\begin{DoxyItemize}
\item State\+Imagery definitions\+: 
\begin{DoxyItemize}
\item Enabled -\/ General imagery for when the widget is enabled. 
\item Disabled -\/ General imagery for when the widget is disabled. 
\end{DoxyItemize}


\item Property initialiser definitions\+: 
\begin{DoxyItemize}
\item Segment\+Widget\+Type -\/ specifies the name of a \char`\"{}\+List\+Header\+Segment\char`\"{} based widget type; an instance of which will be created for each column within the header. (Required) 
\end{DoxyItemize}
\end{DoxyItemize}\hypertarget{fal_wr_ref_fal_wr_ref_sec_9}{}\subsection{Falagard/\+List\+Header\+Segment}\label{fal_wr_ref_fal_wr_ref_sec_9}
Widget type intended for use as a single column header within a list header widget.

Assigned Widget\+Look should provide the following\+: 
\begin{DoxyItemize}
\item State\+Imagery definitions\+: 
\begin{DoxyItemize}
\item Disabled -\/ Imagery to use when the widget is disabled. 
\item Normal -\/ Imagery to use when the widget is enabled and the mouse is not within any part of the segment widget. 
\item Hover -\/ Imagery to use when the widget is enabled and the mouse is within the main area of the widget (not the drag-\/sizing \textquotesingle{}splitter\textquotesingle{} area). 
\item Splitter\+Hover -\/ Imagery to use when the widget is enabled and the mouse is within the drag-\/sizing \textquotesingle{}splitter\textquotesingle{} area. 
\item Drag\+Ghost -\/ Imagery to use for the drag-\/moving \textquotesingle{}ghost\textquotesingle{} of the segment. This state should specify that its imagery be render unclipped. 
\item Ascending\+Sort\+Icon -\/ Additional imagery used when the segment has the ascending sort direction set. 
\item Descending\+Sort\+Down -\/ Additional imagery used when the segment has the descending sort direction set. 
\item Ghost\+Ascending\+Sort\+Icon -\/ Additional imagery used for the drag-\/moving \textquotesingle{}ghost\textquotesingle{} when the segment has the ascending sort direction set. 
\item Ghost\+Descending\+Sort\+Down -\/ Additional imagery used for the drag-\/moving \textquotesingle{}ghost\textquotesingle{} when the segment has the descending sort direction set. 
\end{DoxyItemize}


\item Property initialiser definitions\+: 
\begin{DoxyItemize}
\item Moving\+Cursor\+Image -\/ Property to define a mouse cursor image to use when drag-\/moving the widget. (Optional). 
\item Sizing\+Cursor\+Image -\/ Property to define a mouse cursor image to use when drag-\/sizing the widget. (Optional). 
\end{DoxyItemize}
\end{DoxyItemize}\hypertarget{fal_wr_ref_fal_wr_ref_sec_10}{}\subsection{Falagard/\+Menubar}\label{fal_wr_ref_fal_wr_ref_sec_10}
General purpose horizontal menu bar widget.

Assigned Widget\+Look should provide the following\+: 
\begin{DoxyItemize}
\item State\+Imagery definitions\+: 
\begin{DoxyItemize}
\item Enabled -\/ General imagery for when the widget is enabled. 
\item Disabled -\/ General imagery for when the widget is disabled. 
\end{DoxyItemize}


\item Named\+Area definitions\+: 
\begin{DoxyItemize}
\item Item\+Render\+Area -\/ Target area where menu items will appear. 
\end{DoxyItemize}
\end{DoxyItemize}\hypertarget{fal_wr_ref_fal_wr_ref_sec_11}{}\subsection{Falagard/\+Menu\+Item}\label{fal_wr_ref_fal_wr_ref_sec_11}
General purpose textual menu item widget.

Assigned Widget\+Look should provide the following\+: 
\begin{DoxyItemize}
\item State\+Imagery definitions\+: 
\begin{DoxyItemize}
\item Enabled\+Normal -\/ Imagery used when the item is enabled and the mouse is not within its area. 
\item Enabled\+Hover -\/ Imagery used when the item is enabled and the mouse is within its area. 
\item Enabled\+Pushed -\/ Imagery used when the item is enabled and user has pushed the mouse button over it. 
\item Enabled\+Popup\+Open -\/ Imagery used when the item is enabled and attached popup menu is opened. 
\item Disabled\+Normal -\/ Imagery used when the item is disabled and the mouse is not within its area. 
\item Disabled\+Hover -\/ Imagery used when the item is disabled and the mouse is within its area. 
\item Disabled\+Pushed -\/ Imagery used when the item is disabled and user has pushed the mouse button over it. 
\item Disabled\+Popup\+Open -\/ Imagery used when the item is disabled and attached popup menu is opened. 
\item Popup\+Closed\+Icon -\/ Additional imagery used when the item is attached to a popup menu widget and has a a \textquotesingle{}sub\textquotesingle{} popup menu attached to itself, and that popup is closed. 
\item Popup\+Open\+Icon -\/ Additional imagery used when the item is attached to a popup menu widget and has a a \textquotesingle{}sub\textquotesingle{} popup menu attached to itself, and that popup is open. 
\end{DoxyItemize}


\item Named\+Area definitions\+: 
\begin{DoxyItemize}
\item Content\+Size -\/ Area defining the size of this item\textquotesingle{}s content. Required. 
\item Has\+Popup\+Content\+Size -\/ Area defining the size of this item\textquotesingle{}s content if the item has an attached popup menu and is not attached to a Menubar (basically the content size with allowance for the \textquotesingle{}popup icon\textquotesingle{}. Optional. 
\end{DoxyItemize}
\end{DoxyItemize}\hypertarget{fal_wr_ref_fal_wr_ref_sec_12}{}\subsection{Falagard/\+Multi\+Column\+List}\label{fal_wr_ref_fal_wr_ref_sec_12}
General purpose multi-\/column list / grid widget.

Assigned Widget\+Look should provide the following\+: 
\begin{DoxyItemize}
\item State\+Imagery definitions\+: 
\begin{DoxyItemize}
\item Enabled -\/ General imagery for when the widget is enabled. 
\item Disabled -\/ General imagery for when the widget is disabled. 
\end{DoxyItemize}


\item Named\+Area definitions\+: 
\begin{DoxyItemize}
\item Item\+Rendering\+Area -\/ Target area where list items will appear when no scrollbars are visible (also acts as default area). Required. 
\item Item\+Rendering\+Area\+H\+Scroll -\/ Target area where list items will appear when the horizontal scrollbar is visible. Optional. 
\item Item\+Rendering\+Area\+V\+Scroll -\/ Target area where list items will appear when the vertical scrollbar is visible. Optional. 
\item Item\+Rendering\+Area\+H\+V\+Scroll -\/ Target area where list items will appear when both the horizontal and vertical scrollbars are visible. Optional. 
\end{DoxyItemize}
\end{DoxyItemize}\hypertarget{fal_wr_ref_fal_wr_ref_sec_13}{}\subsection{Falagard/\+Multi\+Line\+Editbox}\label{fal_wr_ref_fal_wr_ref_sec_13}
General purpose multi-\/line text box widget.

Assigned Widget\+Look should provide the following\+: 
\begin{DoxyItemize}
\item State\+Imagery definitions\+: 
\begin{DoxyItemize}
\item Enabled -\/ Imagery used when widget is enabled. 
\item Disabled -\/ Imagery used when widget is disabled. 
\item Read\+Only -\/ Imagery used when widget is in \textquotesingle{}Read Only\textquotesingle{} state. 
\end{DoxyItemize}


\item Named\+Area definitions\+: 
\begin{DoxyItemize}
\item Text\+Area -\/ Target area where text lines will appear when no scrollbars are visible (also acts as default area). Required. 
\item Text\+Area\+H\+Scroll -\/ Target area where text lines will appear when the horizontal scrollbar is visible. Optional. 
\item Text\+Area\+V\+Scroll -\/ Target area where text lines will appear when the vertical scrollbar is visible. Optional. 
\item Text\+Area\+H\+V\+Scroll -\/ Target area where text lines will appear when both the horizontal and vertical scrollbars are visible. Optional. 
\end{DoxyItemize}


\item Imagery\+Section definitions\+: 
\begin{DoxyItemize}
\item Carat -\/ Additional imagery used to display the insertion position carat. 
\end{DoxyItemize}


\item Property\+Definition specifications (optional, defaults will be black)\+: 
\begin{DoxyItemize}
\item Normal\+Text\+Colour -\/ property that accesses a colour value to be used to render normal unselected text. 
\item Selected\+Text\+Colour -\/ property that accesses a colour value to be used to render selected text. 
\item Active\+Selection\+Colour -\/ property that accesses a colour value to be used to render active selection highlight. 
\item Inactive\+Selection\+Colour -\/ property that accesses a colour value to be used to render inactive selection highlight. 
\end{DoxyItemize}
\end{DoxyItemize}\hypertarget{fal_wr_ref_fal_wr_ref_sec_14}{}\subsection{Falagard/\+Popup\+Menu}\label{fal_wr_ref_fal_wr_ref_sec_14}
General purpose popup menu widget.

Assigned Widget\+Look should provide the following\+: 
\begin{DoxyItemize}
\item State\+Imagery definitions\+: 
\begin{DoxyItemize}
\item Enabled -\/ General imagery for when the widget is enabled. 
\item Disabled -\/ General imagery for when the widget is disabled. 
\end{DoxyItemize}


\item Named\+Area definitions\+: 
\begin{DoxyItemize}
\item Item\+Render\+Area -\/ Target area where menu items will appear. 
\end{DoxyItemize}
\end{DoxyItemize}\hypertarget{fal_wr_ref_fal_wr_ref_sec_15}{}\subsection{Falagard/\+Progress\+Bar}\label{fal_wr_ref_fal_wr_ref_sec_15}
General purpose progress widget.

Assigned Widget\+Look should provide the following\+: 
\begin{DoxyItemize}
\item State\+Imagery definitions\+: 
\begin{DoxyItemize}
\item Enabled -\/ General imagery used when widget is enabled. 
\item Disabled -\/ General imagery used when widget is disabled. 
\item Enabled\+Progress -\/ imagery for 100\textbackslash{} progress used when widget is enabled. The drawn imagery will appear in named area \char`\"{}\+Progress\+Area\char`\"{} and will be clipped appropriately according to widget settings and the current progress value. 
\item Disabled\+Progress -\/ imagery for 100\textbackslash{} progress used when widget is disabled. The drawn imagery will appear in named area \char`\"{}\+Progress\+Area\char`\"{} and will be clipped appropriately according to widget settings and the current progress value. 
\end{DoxyItemize}


\item Named\+Area definitions\+: 
\begin{DoxyItemize}
\item Progress\+Area -\/ Target area where progress imagery will appear. 
\end{DoxyItemize}


\item Property initialiser definitions\+: 
\begin{DoxyItemize}
\item Vertical\+Progress -\/ boolean property. Determines whether the progress widget is horizontal or vertical. Default is horizontal. Optional. 
\item Reversed\+Progress -\/ boolean property. Determines whether the progress grows in the opposite direction to what is considered \textquotesingle{}usual\textquotesingle{}. Set to \char`\"{}\+True\char`\"{} to have progress grow towards the left or bottom of the progress area. Optional. 
\end{DoxyItemize}
\end{DoxyItemize}\hypertarget{fal_wr_ref_fal_wr_ref_sec_16}{}\subsection{Falagard/\+Toggle\+Button}\label{fal_wr_ref_fal_wr_ref_sec_16}
General purpose radio button style widget.

Assigned Widget\+Look should provide the following\+: 
\begin{DoxyItemize}
\item State\+Imagery definitions (missing states will default to \textquotesingle{}Normal\textquotesingle{} or \textquotesingle{}Selected\+Normal\textquotesingle{})\+: 
\begin{DoxyItemize}
\item Normal -\/ Imagery used when the widget is in the deselected / off state, and is neither pushed nor has the mouse hovering over it. 
\item Hover -\/ Imagery used when the widget is in the deselected / off state, and has the mouse hovering over it. 
\item Pushed -\/ Imagery used when the widget is in the deselected / off state, is pushed and has mouse over the widget. 
\item Pushed\+Off -\/ Imagery used when the widget is in the deselected / off state, is pushed and does not have the mouse over the widget. 
\item Disabled -\/ Imagery used when the widget is in the deselected / off state, and is disabled. 
\item Selected\+Normal -\/ Imagery used when the widget is in the selected / on state, and is neither pushed nor has the mouse hovering over it. 
\item Selected\+Hover -\/ Imagery used when the widget is in the selected / on state, and has the mouse hovering over it. 
\item Selected\+Pushed -\/ Imagery used when the widget is in the selected / on state, is pushed and has the mouse over the widget. 
\item Selected\+Pushed\+Off -\/ Imagery used when the widget is in the selected / on state, is pushed and does not have the mouse over the widget. 
\item Selected\+Disabled -\/ Imagery used when the widget is in the selected / on state, and is disabled. 
\end{DoxyItemize}
\end{DoxyItemize}\hypertarget{fal_wr_ref_fal_wr_ref_sec_17}{}\subsection{Falagard/\+Scrollable\+Pane}\label{fal_wr_ref_fal_wr_ref_sec_17}
General purpose scrollable pane widget.

Assigned Widget\+Look should provide the following\+: 
\begin{DoxyItemize}
\item State\+Imagery definitions\+: 
\begin{DoxyItemize}
\item Enabled -\/ General imagery for when the widget is enabled. 
\item Disabled -\/ General imagery for when the widget is disabled. 
\end{DoxyItemize}


\item Named\+Area definitions\+: 
\begin{DoxyItemize}
\item Viewable\+Area -\/ Target area where visible content will appear when no scrollbars are visible (also acts as default area). Required. 
\item Viewable\+Area\+H\+Scroll -\/ Target area where visible content will appear when the horizontal scrollbar is visible. Optional. 
\item Viewable\+Area\+V\+Scroll -\/ Target area where visible content will appear when the vertical scrollbar is visible. Optional. 
\item Viewable\+Area\+H\+V\+Scroll -\/ Target area where visible content will appear when both the horizontal and vertical scrollbars are visible. Optional. 
\end{DoxyItemize}
\end{DoxyItemize}\hypertarget{fal_wr_ref_fal_wr_ref_sec_18}{}\subsection{Falagard/\+Scrollbar}\label{fal_wr_ref_fal_wr_ref_sec_18}
General purpose scrollbar widget.

Assigned Widget\+Look should provide the following\+: 
\begin{DoxyItemize}
\item State\+Imagery definitions\+: 
\begin{DoxyItemize}
\item Enabled -\/ General imagery for when the widget is enabled. 
\item Disabled -\/ General imagery for when the widget is disabled. 
\end{DoxyItemize}


\item Named\+Area definitions\+: 
\begin{DoxyItemize}
\item Thumb\+Track\+Area -\/ Target area in which thumb may be moved. 
\end{DoxyItemize}


\item Property initialiser definitions\+: 
\begin{DoxyItemize}
\item Vertical\+Scrollbar -\/ boolean property. Indicates whether this scrollbar will operate in the vertical or horizontal direction. Default is for horizontal. Optional. 
\end{DoxyItemize}
\end{DoxyItemize}\hypertarget{fal_wr_ref_fal_wr_ref_sec_19}{}\subsection{Falagard/\+Slider}\label{fal_wr_ref_fal_wr_ref_sec_19}
General purpose slider widget.

Assigned Widget\+Look should provide the following\+: 
\begin{DoxyItemize}
\item State\+Imagery definitions\+: 
\begin{DoxyItemize}
\item Enabled -\/ General imagery for when the widget is enabled. 
\item Disabled -\/ General imagery for when the widget is disabled. 
\end{DoxyItemize}


\item Named\+Area definitions\+: 
\begin{DoxyItemize}
\item Thumb\+Track\+Area -\/ Target area in which thumb may be moved. 
\end{DoxyItemize}


\item Property initialiser definitions\+: 
\begin{DoxyItemize}
\item Vertical\+Slider -\/ boolean property. Indicates whether this slider will operate in the vertical or horizontal direction. Default is for horizontal. Optional. 
\end{DoxyItemize}
\end{DoxyItemize}\hypertarget{fal_wr_ref_fal_wr_ref_sec_20}{}\subsection{Falagard/\+Static}\label{fal_wr_ref_fal_wr_ref_sec_20}
Generic non-\/interactive \textquotesingle{}static\textquotesingle{} widget. Used as a base class for Falagard/\+Static\+Image and Falagard/\+Static\+Text.

Assigned Widget\+Look should provide the following\+: 
\begin{DoxyItemize}
\item State\+Imagery definitions\+: 
\begin{DoxyItemize}
\item Enabled -\/ General imagery for when the widget is enabled. 
\item Disabled -\/ General imagery for when the widget is disabled. 
\item Enabled\+Frame -\/ Additional imagery used when the widget is enabled and the widget frame is enabled. 
\item Disabled\+Frame -\/ Additional imagery used when the widget is disabled and the widget frame is enabled. 
\item With\+Frame\+Enabled\+Background -\/ Additional imagery used when the widget is enabled, the widget frame is enabled, and the widget background is enabled. 
\item With\+Frame\+Disabled\+Background -\/ Additional imagery used when the widget is disabled, the widget frame is enabled, and the widget background is enabled. 
\item No\+Frame\+Enabled\+Background -\/ Additional imagery used when the widget is enabled, the widget frame is disabled, and the widget background is enabled. 
\item No\+Frame\+Disabled\+Background -\/ Additional imagery used when the widget is disabled, the widget frame is disabled, and the widget background is enabled. 
\end{DoxyItemize}
\end{DoxyItemize}\hypertarget{fal_wr_ref_fal_wr_ref_sec_21}{}\subsection{Falagard/\+Static\+Image}\label{fal_wr_ref_fal_wr_ref_sec_21}
Static widget that displays a configurable image.

Assigned Widget\+Look should provide the following\+: 
\begin{DoxyItemize}
\item State\+Imagery definitions\+: 
\begin{DoxyItemize}
\item Enabled -\/ General imagery for when the widget is enabled. 
\item Disabled -\/ General imagery for when the widget is disabled. 
\item Enabled\+Frame -\/ Additional imagery used when the widget is enabled and the widget frame is enabled. 
\item Disabled\+Frame -\/ Additional imagery used when the widget is disabled and the widget frame is enabled. 
\item With\+Frame\+Enabled\+Background -\/ Additional imagery used when the widget is enabled, the widget frame is enabled, and the widget background is enabled. 
\item With\+Frame\+Disabled\+Background -\/ Additional imagery used when the widget is disabled, the widget frame is enabled, and the widget background is enabled. 
\item No\+Frame\+Enabled\+Background -\/ Additional imagery used when the widget is enabled, the widget frame is disabled, and the widget background is enabled. 
\item No\+Frame\+Disabled\+Background -\/ Additional imagery used when the widget is disabled, the widget frame is disabled, and the widget background is enabled. 
\item With\+Frame\+Image -\/ Image rendering when the frame is enabled. 
\item No\+Frame\+Image -\/ Image rendering when the frame is disabled. 
\end{DoxyItemize}
\end{DoxyItemize}\hypertarget{fal_wr_ref_fal_wr_ref_sec_22}{}\subsection{Falagard/\+Static\+Text}\label{fal_wr_ref_fal_wr_ref_sec_22}
Static widget that displays configurable text.

Assigned Widget\+Look should provide the following\+: 
\begin{DoxyItemize}
\item State\+Imagery definitions\+: 
\begin{DoxyItemize}
\item Enabled -\/ General imagery for when the widget is enabled. 
\item Disabled -\/ General imagery for when the widget is disabled. 
\item Enabled\+Frame -\/ Additional imagery used when the widget is enabled and the widget frame is enabled. 
\item Disabled\+Frame -\/ Additional imagery used when the widget is disabled and the widget frame is enabled. 
\item With\+Frame\+Enabled\+Background -\/ Additional imagery used when the widget is enabled, the widget frame is enabled, and the widget background is enabled. 
\item With\+Frame\+Disabled\+Background -\/ Additional imagery used when the widget is disabled, the widget frame is enabled, and the widget background is enabled. 
\item No\+Frame\+Enabled\+Background -\/ Additional imagery used when the widget is enabled, the widget frame is disabled, and the widget background is enabled. 
\item No\+Frame\+Disabled\+Background -\/ Additional imagery used when the widgetis disabled, the widget frame is disabled, and the widget background is enabled. 
\end{DoxyItemize}


\item Named\+Area definitions (missing areas will default to With\+Frame\+Text\+Render\+Area)\+: 
\begin{DoxyItemize}
\item With\+Frame\+Text\+Render\+Area -\/ Target area where text will appear when the frame is enabled and no scrollbars are visible (also acts as default area). Required. 
\item With\+Frame\+Text\+Render\+Area\+H\+Scroll -\/ Target area where text will appear when the frame is enabled and the horizontal scrollbar is visible. Optional. 
\item With\+Frame\+Text\+Render\+Area\+V\+Scroll -\/ Target area where text will appear when the frame is enabled and the vertical scrollbar is visible. Optional. 
\item With\+Frame\+Text\+Render\+Area\+H\+V\+Scroll -\/ Target area where text will appear when the frame is enabled and both the horizontal and vertical scrollbars are visible. Optional. 
\item No\+Frame\+Text\+Render\+Area -\/ Target area where text will appear when the frame is disabled and no scrollbars are visible (also acts as default area). Optional. 
\item No\+Frame\+Text\+Render\+Area\+H\+Scroll -\/ Target area where text will appear when the frame is disabled and the horizontal scrollbar is visible. Optional. 
\item No\+Frame\+Text\+Render\+Area\+V\+Scroll -\/ Target area where text will appear when the frame is disabled and the vertical scrollbar is visible. Optional. 
\item No\+Frame\+Text\+Render\+Area\+H\+V\+Scroll -\/ Target area where text will appear when the frame is disabled and both the horizontal and vertical scrollbars are visible. Optional. 
\end{DoxyItemize}


\item Child widget definitions\+: 
\begin{DoxyItemize}
\item Scrollbar based widget with name suffix \char`\"{}\+\_\+\+\_\+auto\+\_\+vscrollbar\+\_\+\+\_\+\char`\"{}. This widget will be used to control vertical scroll position. 
\item Scrollbar based widget with name suffix \char`\"{}\+\_\+\+\_\+auto\+\_\+hscrollbar\+\_\+\+\_\+\char`\"{}. This widget will be used to control horizontal scroll position. 
\end{DoxyItemize}
\end{DoxyItemize}\hypertarget{fal_wr_ref_fal_wr_ref_sec_23}{}\subsection{Falagard/\+System\+Button}\label{fal_wr_ref_fal_wr_ref_sec_23}
Specialised push button widget intended to be used for \textquotesingle{}system\textquotesingle{} buttons appearing outside of the client area of a frame window style widget.

Assigned Widget\+Look should provide the following\+: 
\begin{DoxyItemize}
\item State\+Imagery definitions (missing states will default to \textquotesingle{}Normal\textquotesingle{})\+: 
\begin{DoxyItemize}
\item Normal -\/ Imagery used when the widget is neither pushed nor has the mouse hovering over it. 
\item Hover -\/ Imagery used when the widget is not pushed and has the mouse hovering over it. 
\item Pushed -\/ Imagery used when the widget is pushed and the mouse is over the widget. 
\item Pushed\+Off -\/ Imagery used when the widget is pushed and the mouse is not over the widget. 
\item Disabled -\/ Imagery used when the widget is disabled. 
\end{DoxyItemize}
\end{DoxyItemize}\hypertarget{fal_wr_ref_fal_wr_ref_sec_24}{}\subsection{Falagard/\+Tab\+Button}\label{fal_wr_ref_fal_wr_ref_sec_24}
Special widget type used for tab buttons within a tab control based widget.

Assigned Widget\+Look should provide the following\+: 
\begin{DoxyItemize}
\item State\+Imagery definitions (missing states will default to \textquotesingle{}Normal\textquotesingle{})\+: 
\begin{DoxyItemize}
\item Normal -\/ Imagery used when the widget is neither selected nor has the mouse hovering over it. 
\item Hover -\/ Imagery used when the widget has the mouse hovering over it. 
\item Selected -\/ Imagery used when the widget is the active / selected tab. 
\item Disabled -\/ Imagery used when the widget is disabled. 
\end{DoxyItemize}
\end{DoxyItemize}\hypertarget{fal_wr_ref_fal_wr_ref_sec_25}{}\subsection{Falagard/\+Tab\+Control}\label{fal_wr_ref_fal_wr_ref_sec_25}
General purpose tab control widget.

The current Tab\+Control base class enforces a fairly strict layout, so while imagery can be customised as desired, the general layout of the component widgets is, at least for the time being, mostly fixed.

Assigned Widget\+Look should provide the following\+: 
\begin{DoxyItemize}
\item State\+Imagery definitions\+: 
\begin{DoxyItemize}
\item Enabled -\/ General imagery for when the widget is enabled. 
\item Disabled -\/ General imagery for when the widget is disabled. 
\end{DoxyItemize}


\item Property initialiser definitions\+: 
\begin{DoxyItemize}
\item Tab\+Button\+Type -\/ specifies a Tab\+Button based widget type to be created each time a new tab button is required. 
\end{DoxyItemize}
\end{DoxyItemize}\hypertarget{fal_wr_ref_fal_wr_ref_sec_26}{}\subsection{Falagard/\+Titlebar}\label{fal_wr_ref_fal_wr_ref_sec_26}
Title bar widget intended for use as the title bar of a frame window widget.

Assigned Widget\+Look should provide the following\+: 
\begin{DoxyItemize}
\item State\+Imagery definitions (missing states will default to \textquotesingle{}Normal\textquotesingle{})\+: 
\begin{DoxyItemize}
\item Active -\/ Imagery used when the widget is active. 
\item Inactive -\/ Imagery used when the widget is inactive. 
\item Disabled -\/ Imagery used when the widget is disabled. 
\end{DoxyItemize}
\end{DoxyItemize}\hypertarget{fal_wr_ref_fal_wr_ref_sec_27}{}\subsection{Falagard/\+Tooltip}\label{fal_wr_ref_fal_wr_ref_sec_27}
General purpose tool-\/tip widget.

Assigned Widget\+Look should provide the following\+: 
\begin{DoxyItemize}
\item State\+Imagery definitions\+: 
\begin{DoxyItemize}
\item Enabled -\/ General imagery for when the widget is enabled. 
\item Disabled -\/ General imagery for when the widget is disabled. 
\end{DoxyItemize}


\item Named\+Area definitions\+: 
\begin{DoxyItemize}
\item Text\+Area -\/ Typically this would be the same area as the Text\+Component you define to receive the tool-\/tip text. This named area is used when deciding how to dynamically size the tool-\/tip so that text is not clipped. 
\end{DoxyItemize}
\end{DoxyItemize}\hypertarget{fal_wr_ref_fal_wr_ref_sec_28}{}\subsection{Falagard/\+Tree\+View}\label{fal_wr_ref_fal_wr_ref_sec_28}
Basic Tree type view.

Assigned Widget\+Look should provide the following\+: 
\begin{DoxyItemize}
\item State\+Imagery definitions\+: 
\begin{DoxyItemize}
\item Enabled -\/ General imagery for when the view is enabled. 
\item Disabled -\/ General imagery for when the view is disabled. 
\end{DoxyItemize}


\item Named\+Area definitions\+: 
\begin{DoxyItemize}
\item Item\+Rendering\+Area -\/ Target area where tree items will appear when no scrollbars are visible (also acts as default area). Required. 
\item Item\+Rendering\+Area\+H\+Scroll -\/ Target area where tree items will appear when the horizontal scrollbar is visible. Optional. 
\item Item\+Rendering\+Area\+V\+Scroll -\/ Target area where tree items will appear when the vertical scrollbar is visible. Optional. 
\item Item\+Rendering\+Area\+H\+V\+Scroll -\/ Target area where tree items will appear when both the horizontal and vertical scrollbars are visible. Optional. 
\end{DoxyItemize}
\end{DoxyItemize}\hypertarget{fal_wr_ref_fal_wr_ref_sec_29}{}\subsection{Falagard/\+Tree\+Widget}\label{fal_wr_ref_fal_wr_ref_sec_29}
Basic Tree type widget.

Assigned Widget\+Look should provide the following\+: 
\begin{DoxyItemize}
\item State\+Imagery definitions\+: 
\begin{DoxyItemize}
\item Enabled -\/ General imagery for when the widget is enabled. 
\item Disabled -\/ General imagery for when the widget is disabled. 
\end{DoxyItemize}


\item Named\+Area definitions\+: 
\begin{DoxyItemize}
\item Item\+Rendering\+Area -\/ Target area where tree items will appear when no scrollbars are visible (also acts as default area). Required. 
\item Item\+Rendering\+Area\+H\+Scroll -\/ Target area where tree items will appear when the horizontal scrollbar is visible. Optional. 
\item Item\+Rendering\+Area\+V\+Scroll -\/ Target area where tree items will appear when the vertical scrollbar is visible. Optional. 
\item Item\+Rendering\+Area\+H\+V\+Scroll -\/ Target area where tree items will appear when both the horizontal and vertical scrollbars are visible. Optional. 
\end{DoxyItemize}
\end{DoxyItemize}