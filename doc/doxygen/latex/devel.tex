\begin{DoxyAuthor}{Author}
Paul D Turner
\end{DoxyAuthor}
The C\+E\+G\+UI developers are always happy to consider code and other materials contributed from users out in the community for inclusion with C\+E\+G\+UI. Many of the existing parts of C\+E\+G\+UI are -\/ or started out as -\/ user contributed material.

In order to aid people wishing to contribute to the development of C\+E\+G\+UI, we\textquotesingle{}ve gathered together some basic guidelines that will both maximise the productivity of yourselves and the C\+E\+G\+UI developers, and also maximise the chances that your contribution will be accepted into the code base. Certain points here also serve as a guide to C\+E\+G\+UI developers when assessing the overall quality of a submission.


\begin{DoxyItemize}
\item Do not submit code that you personally did not create, do not own, or are otherwise unauthorised to contribute or give away.
\item Unless clearly stated otherwise on your submission, all contributed materials will be considered to be gifted to us, for use as we see fit. Typically this just means we\textquotesingle{}ll be releasing the contributed materials under our M\+IT license.
\item Before you start work, especially on anything major, you should definitely join our developer channel \#cegui-\/devel on irc.\+freenode.\+net and discuss your ideas and intentions (alternatively, post on the forums at \href{http://forums.cegui.org.uk/}{\texttt{ http\+://forums.\+cegui.\+org.\+uk/}}). There is nothing worse in our eyes than somebody spending valuable time working on something that turns out to be unsuitable when completed -\/ whether due to a design choice we could have advised about, something conflicting with existing or upcoming work, or some other reason.
\item We presently accept modifications exclusively via mercurial pull request -\/ so this means no patches, no archives containing files or anything else of that nature.
\item Ideally your clone would be hosted at bitbucket.\+org and made via the fork mechanism that exists there. This way your pull requests can be handled in the most effient manner.
\item If your fork is not hosted at bitbucket.\+org, please add a ticket to the appropriate sub-\/project on the C\+E\+G\+UI mantis tracker (\href{http://mantis.cegui.org.uk/}{\texttt{ http\+://mantis.\+cegui.\+org.\+uk/}}). Posting pull requests on irc or on the forums is not advised and will virtually always result in your request being lost and forgotten about.
\item It should be clearly stated on your pull request what precisely the change or changes are for -\/ including links to forum discussions and/or other mantis tickets where appropriate.
\item When working on changes, try to keep to small granular commits that each {\itshape do one thing only.} If your commit message states that the change fixes a bug, or adds some new feature, the commit should not additionally contain changes to other non related parts of the system. Pull requests with commits that fall into this category are likely to be rejected.
\item When developing code for C\+E\+G\+UI -\/ whether you\textquotesingle{}re modifying existing code, or developing new code -\/ you should ensure that the code conforms to the existing style and idioms in use. The required code style is documented in \mbox{\hyperlink{code_standards}{Coding Standards in use for C\+E\+G\+UI}}, and your contributed code should follow this as closely as possible. Contributed code that deviates too much from these guidelines will be rejected as a matter of course.
\item Ensure all new classes and/or functions are clearly documented and that any documentation for modified classes and/or functions is updated to be correct. Documentation should use the same doxygen style as is used elsewhere. If we see new undocumented code or clearly incorrect documentation in your pull request will likely be rejected.
\item Where possible you should test your code on multiple platforms, and update any build mechanisms if appropriate. If you\textquotesingle{}re unable to test on all platforms, your pull request should clearly state which systems have been tested and which have not.
\item Modifications should be complete -\/ especially those that affect the abstract classes and/or interfaces. For example, if you\textquotesingle{}re adding to or changing the functionality of the rendering system, you should include the necessary modifications and implementations for {\itshape all supported renderers} (or as many as possible, perhaps with an explanatory note). Changes that effectively break all implementation modules bar one can usually expect to be rejected, the reason being is that accepting such modifications effectively leaves the C\+E\+G\+UI developers to do 75\% or more of the work -\/ since the remaining modules would have to be updated by us -\/ which defeats the purpose of having people submit code in the first place.
\item Lastly, don\textquotesingle{}t forget to update our equivalent of the A\+U\+T\+H\+O\+RS file at {\ttfamily doc/doxygen/authors.\+dox} with your name and contribution. If you don\textquotesingle{}t do this, we\textquotesingle{}ll assume you wish your contribution to be uncredited. 
\end{DoxyItemize}