\begin{DoxyAuthor}{Author}
Paul D Turner
\end{DoxyAuthor}
This tutorial introduces the C\+E\+G\+U\+I\+::\+Resource\+Provider concept, explains how to set up resource groups and directories in the Default\+Resource\+Provider and how to specify the default resource groups to be used by various parts of the system.

~\newline
 \hypertarget{resprov_tutorial_resprov_tutorial_intro}{}\section{What is a Resource\+Provider?}\label{resprov_tutorial_resprov_tutorial_intro}
C\+E\+G\+UI uses a Resource\+Provider object to provide a bridge between the C\+E\+G\+UI library and external file loading systems -\/ whether this might be the basic underlying file system of the machine executing the code, or something a bit more exotic such as the resource management sub-\/systems offered by the Ogre3D and Irrlicht engines; by providing custom implementations of the Resource\+Provider interface, it is possible for C\+E\+G\+UI to seamlessly integrate with all of these systems.

~\newline
 \hypertarget{resprov_tutorial_resprov_tutorial_default_rp}{}\section{Default\+Resource\+Provider Explained}\label{resprov_tutorial_resprov_tutorial_default_rp}
C\+E\+G\+UI\textquotesingle{}s default resource provider, Default\+Resource\+Provider, supplies some basic functionality for those who do not yet have, or do not need, a more sophisticated alternative. As well as supplying the functions required by C\+E\+G\+UI for actually loading file data, Default\+Resource\+Provider also has some rudimentary support for \textquotesingle{}resource groups\textquotesingle{}. Here, a resource group is basically a label given to a directory location on the system. This allows one to have logical grouping of files within directories and then to be able to refer to those locations via a simple label rather than hard coded paths. This then means that should you need to change the location of some data files, you just need to update the resource group location instead of updating path information throughout your code and X\+ML files.

For users of the Ogre3D library -\/ or users with other custom resource providers not derived from Default\+Resource\+Provider -\/ {\itshape you should not follow the parts of this tutorial that deal with defining resouce groups and their directories} (you should especially ignore any example code that casts to Default\+Resouce\+Provider). These users should follow the usual procedures for using the underlying resouce system in use (for example, in Ogre3D you might set up your resource locations in code using the Ogre\+::\+Resource\+Manager or via a resources.\+cfg file).

This said, {\itshape all users} should follow the parts of this tutorial dealing with setting the default resouce groups for the various resource classes, as detailed in \mbox{\hyperlink{resprov_tutorial_resprov_tutorial_default_resource_groups}{Specifying Default Resource Groups}}.\hypertarget{resprov_tutorial_resprov_tutorial_default_rp_groups}{}\subsection{Specifying Resource Groups and Directories}\label{resprov_tutorial_resprov_tutorial_default_rp_groups}
The Default\+Resource\+Provider allows you to define any number of named resource groups and to specify a directory to be accessed by each of those groups. What this means is that you can create a resource group, say {\ttfamily \char`\"{}imagesets\char`\"{}}, and assign a directory to that group, for example {\ttfamily \char`\"{}./mygame/datafiles/gui/imagesets/\char`\"{}}. When loading an Imageset through the Image\+Manager, you might then specify the resource group to be used as {\ttfamily \char`\"{}imagesets\char`\"{}} and the system will look in the predefined location {\ttfamily \char`\"{}./mygame/datafiles/gui/imagesets/\char`\"{}}. Note that at present each resource group may only have a single directory associated with it.

A small code example is in order to clarify what has been said. Instead of loading resources by giving an explicit path, like this\+: 
\begin{DoxyCode}{0}
\DoxyCodeLine{ImageManager::getSingleton().loadImageset(}
\DoxyCodeLine{    \textcolor{stringliteral}{"./mygame/datafiles/gui/imagesets/WindowsLook.imageset"});}
\end{DoxyCode}


At initialisation time, you set up a resource group in the default resource provider, like this\+: 
\begin{DoxyCode}{0}
\DoxyCodeLine{DefaultResourceProvider* rp = static\_cast<DefaultResourceProvider*>(}
\DoxyCodeLine{    CEGUI::System::getSingleton().getResourceProvider());}
\DoxyCodeLine{}
\DoxyCodeLine{rp->setResourceGroupDirectory(\textcolor{stringliteral}{"imagesets"}, \textcolor{stringliteral}{"./mygame/datafiles/gui/imagesets/"});}
\end{DoxyCode}


Then later on in the code, when you need to load an imageset, just do this\+: 
\begin{DoxyCode}{0}
\DoxyCodeLine{ImageManager::getSingleton().loadImageset(}
\DoxyCodeLine{    \textcolor{stringliteral}{"WindowsLook.imageset"}, \textcolor{stringliteral}{"imagesets"});}
\end{DoxyCode}


Note how when loading the imageset from a file we do not specify any path information; the path information is obtained from the resource group specified, in the example this is {\ttfamily \char`\"{}imagesets\char`\"{}}. We will later show you how you set default resource groups for each of the resource types -\/ then you do not have to specify the group when you load a resource (unless you\textquotesingle{}re loading it from a group other than the default, of course).

Something important to consider is that when using this resource group approach, data files that contain references to other data files should not contain relative path information -\/ they should, in general, just have the actual file name of the file being referred to -\/ this way the resource group system can be put to better use, and it also makes it easier to move files around later -\/ since rather than having to \textquotesingle{}fix up\textquotesingle{} all the relative paths, you just have to update the resource group paths instead.

~\newline
 \hypertarget{resprov_tutorial_resprov_tutorial_default_resource_groups}{}\section{Specifying Default Resource Groups}\label{resprov_tutorial_resprov_tutorial_default_resource_groups}
Each of the core system classes that represents a loadable resource has static members to set and get a default resource group. This resource group will be used when loading the specific data files needed by a given class.

For each of the resource consuming classes, the static members are named the same (special exceptions are the Image\+Manager and the the xerces-\/c based X\+ML paser -\/ see below)\+: 
\begin{DoxyCode}{0}
\DoxyCodeLine{\textcolor{keyword}{const} String\& getDefaultResourceGroup();}
\DoxyCodeLine{\textcolor{keywordtype}{void} setDefaultResourceGroup(\textcolor{keyword}{const} String\& groupname);}
\end{DoxyCode}


The following is a list of the core resource loading classes and the resources that they load\+:
\begin{DoxyItemize}
\item {\ttfamily C\+E\+G\+U\+I\+::\+Font} \+: Font xml and freetype loadable font files.
\item {\ttfamily C\+E\+G\+U\+I\+::\+Scheme} \+: Scheme xml files.
\item {\ttfamily C\+E\+G\+U\+I\+::\+Window\+Manager} \+: Window layout xml files.
\item {\ttfamily C\+E\+G\+U\+I\+::\+Widget\+Look\+Manager} \+: Look\+N\+Feel xml files
\item {\ttfamily C\+E\+G\+U\+I\+::\+Script\+Module} \+: Script files in whichever scripted langauge.
\end{DoxyItemize}

\begin{DoxyNote}{Note}
One final thing to consider, is that the Resource\+Provider class also has a default resource group. This should be considered a global or master default; it is used whenever a specific resource loading class has no default of it\textquotesingle{}s own specified. This would prove useful if you have all your data in a single directory.
\end{DoxyNote}
There are a couple of special exceptions, as mentioned above, these are the Image\+Manager and the Xerces-\/\+C++ based X\+ML parser.\hypertarget{resprov_tutorial_resprov_tutorial_default_resource_groups_im}{}\subsection{Image\+Manager Default Resource Group}\label{resprov_tutorial_resprov_tutorial_default_resource_groups_im}
Since there is no acutal Imageset class (an Imageset is now just a set of images loaded together from an xml file), the default group used for loading these files is set on the C\+E\+G\+U\+I\+::\+Image\+Manager class and uses functions named differently from the \textquotesingle{}standard\textquotesingle{} mentined above, the functions you need are named\+: 
\begin{DoxyCode}{0}
\DoxyCodeLine{\textcolor{keyword}{const} String\& ImageManager::getImagesetDefaultResourceGroup();}
\DoxyCodeLine{\textcolor{keywordtype}{void} ImageManager::setImagesetDefaultResourceGroup(\textcolor{keyword}{const} String\& resourceGroup);}
\end{DoxyCode}


The resource group specified will be used as the default for loading imageset X\+ML files as well as image texture files.\hypertarget{resprov_tutorial_resprov_tutorial_default_resource_groups_xc}{}\subsection{Xerces\+Parser Default Resource Group}\label{resprov_tutorial_resprov_tutorial_default_resource_groups_xc}
For the Xerces\+Parser there is a special resource group setting to specify where the .xsd schema files -\/ used for X\+ML schema validation -\/ can be found. For this special case, you instead use the Property\+Set interface and access a property named {\ttfamily Schema\+Default\+Resource\+Group}. The use of the property interface mainly serves to eliminate the need to link directly with the xerces xml based parser module just to set the default schema resource group.

Because you may not know ahead of time which X\+ML parser module is actually being used -\/ and therefore may not know whether the property exists -\/ you should check the existence of the propery before setting it. This is better than checking the X\+ML parser ID string for \char`\"{}\+Xerces\char`\"{} because it allows you to seamlessly work with any future parser module that may also offer validation (so long as it exposes the same property).

For example\+: 
\begin{DoxyCode}{0}
\DoxyCodeLine{<<<<<<< local}
\DoxyCodeLine{CEGUI::System::getSingleton().getDefaultGUIContext().}
\DoxyCodeLine{    getCursor().setDefaultImage( \textcolor{stringliteral}{"TaharezLook/MouseArrow"} );}
\end{DoxyCode}


Finally, if you intend to use tool tips, you should specify the type of the Tool\+Tip based widget that you want used for that purpose. It is actually possible to set this on a per-\/window basis, though this is not normally required, and is beyond the scope of this introductory tutorial. The code to set the default tool tip window type for the initial, default G\+U\+I\+Context looks like this\+: 
\begin{DoxyCode}{0}
\DoxyCodeLine{CEGUI::System::getSingleton().getDefaultGUIContext().}
\DoxyCodeLine{    setDefaultTooltipType( \textcolor{stringliteral}{"TaharezLook/Tooltip"} );}
\DoxyCodeLine{=======}
\DoxyCodeLine{\textcolor{comment}{// setup default group for validation schemas}}
\DoxyCodeLine{CEGUI::XMLParser* parser = CEGUI::System::getSingleton().getXMLParser();}
\DoxyCodeLine{\textcolor{keywordflow}{if} (parser->isPropertyPresent(\textcolor{stringliteral}{"SchemaDefaultResourceGroup"}))}
\DoxyCodeLine{    parser->setProperty(\textcolor{stringliteral}{"SchemaDefaultResourceGroup"}, \textcolor{stringliteral}{"schemas"});}
\DoxyCodeLine{>>>>>>> other}
\end{DoxyCode}


~\newline
 \hypertarget{resprov_tutorial_resprov_tutorial_example}{}\section{A final, Complete Example}\label{resprov_tutorial_resprov_tutorial_example}
To close, we will show how you might perform the initialisation of resource groups and their target directories to access the data files within the datafiles directory that comes with C\+E\+G\+UI, and how we assign the default groups to be used for all of the resource types.

After initialising the core C\+E\+G\+U\+I\+::\+System object as usual, we then specify a set of resource groups and their target directories (this assumes the working directory will be the \textquotesingle{}bin\textquotesingle{} directory within the C\+E\+G\+UI package\+: 
\begin{DoxyCode}{0}
\DoxyCodeLine{\textcolor{comment}{// initialise the required dirs for the DefaultResourceProvider}}
\DoxyCodeLine{CEGUI::DefaultResourceProvider* rp = static\_cast<CEGUI::DefaultResourceProvider*>}
\DoxyCodeLine{    (CEGUI::System::getSingleton().getResourceProvider());}
\DoxyCodeLine{}
\DoxyCodeLine{rp->setResourceGroupDirectory(\textcolor{stringliteral}{"schemes"}, \textcolor{stringliteral}{"../datafiles/schemes/"});}
\DoxyCodeLine{rp->setResourceGroupDirectory(\textcolor{stringliteral}{"imagesets"}, \textcolor{stringliteral}{"../datafiles/imagesets/"});}
\DoxyCodeLine{rp->setResourceGroupDirectory(\textcolor{stringliteral}{"fonts"}, \textcolor{stringliteral}{"../datafiles/fonts/"});}
\DoxyCodeLine{rp->setResourceGroupDirectory(\textcolor{stringliteral}{"layouts"}, \textcolor{stringliteral}{"../datafiles/layouts/"});}
\DoxyCodeLine{rp->setResourceGroupDirectory(\textcolor{stringliteral}{"looknfeels"}, \textcolor{stringliteral}{"../datafiles/looknfeel/"});}
\DoxyCodeLine{rp->setResourceGroupDirectory(\textcolor{stringliteral}{"lua\_scripts"}, \textcolor{stringliteral}{"../datafiles/lua\_scripts/"});}
\DoxyCodeLine{}
\DoxyCodeLine{\textcolor{comment}{// This is only really needed if you are using Xerces and need to}}
\DoxyCodeLine{\textcolor{comment}{// specify the schemas location}}
\DoxyCodeLine{rp->setResourceGroupDirectory(\textcolor{stringliteral}{"schemas"}, \textcolor{stringliteral}{"../datafiles/xml\_schemas/"});}
\end{DoxyCode}
 Now that is done, we have a nice set of resource groups defined with their target directories set. Finally, to get the system to use these directories, we set the default resource groups to be used\+: 
\begin{DoxyCode}{0}
\DoxyCodeLine{\textcolor{comment}{// set the default resource groups to be used}}
\DoxyCodeLine{CEGUI::ImageManager::setImagesetDefaultResourceGroup(\textcolor{stringliteral}{"imagesets"});}
\DoxyCodeLine{CEGUI::Font::setDefaultResourceGroup(\textcolor{stringliteral}{"fonts"});}
\DoxyCodeLine{CEGUI::Scheme::setDefaultResourceGroup(\textcolor{stringliteral}{"schemes"});}
\DoxyCodeLine{CEGUI::WidgetLookManager::setDefaultResourceGroup(\textcolor{stringliteral}{"looknfeels"});}
\DoxyCodeLine{CEGUI::WindowManager::setDefaultResourceGroup(\textcolor{stringliteral}{"layouts"});}
\DoxyCodeLine{CEGUI::ScriptModule::setDefaultResourceGroup(\textcolor{stringliteral}{"lua\_scripts"});}
\DoxyCodeLine{}
\DoxyCodeLine{\textcolor{comment}{// setup default group for validation schemas}}
\DoxyCodeLine{CEGUI::XMLParser* parser = CEGUI::System::getSingleton().getXMLParser();}
\DoxyCodeLine{\textcolor{keywordflow}{if} (parser->isPropertyPresent(\textcolor{stringliteral}{"SchemaDefaultResourceGroup"}))}
\DoxyCodeLine{    parser->setProperty(\textcolor{stringliteral}{"SchemaDefaultResourceGroup"}, \textcolor{stringliteral}{"schemas"});}
\end{DoxyCode}
\hypertarget{resprov_tutorial_resprov_tutorial_conclusion}{}\section{Conclusion}\label{resprov_tutorial_resprov_tutorial_conclusion}
This has been a brief introduction to the Resouce\+Provider system used by C\+E\+G\+UI. You have seen how to create and use resource groups with C\+E\+G\+UI\textquotesingle{}s Default\+Resource\+Provider, and how to specify default resource groups for each resource type used by C\+E\+G\+UI. 